\documentclass[kapital_02.tex]{subfiles}

\begin{document}

Če \KPEstran označimo s \( \KPECk \)  celotni proces cirkulacije, lahko prikažemo tri like takole:

\[
    \begin{array}{r l}
        \textrm{I. } & \KPED \KPEcrta \KPEB \KPEpike \KPEP \KPEpike \KPEB' \KPEcrta \KPED' \\
        \textrm{II. } & \KPEP \KPEpike \KPECk \KPEpike \KPEP \\
        \textrm{III. } & \KPECk \KPEpike \KPEP (\KPEB') \\
    \end{array}
\]

Če pogledamo vse tri oblike skupaj, se pokažejo vse predpostavke procesa za njegov rezultat, za predpostavko, ki jo ustvarja sam. Vsaka točka se pokaže za izhodiščno, prehodno in povratno. Celotni proces se pokaže kot enotnost produkcijskega in cirkulacijskega procesa; produkcijski proces postane posredovalec cirkulacijskega, in narobe.

Vsem trem krožnim tokom je skupno, da je povečanje vrednosti določilni smoter, gonilni motiv vseh. V I izraža to oblika. Oblika II se začne s \( \KPEP \), procesom večanja vrednosti samim. V III se prične krožni tok s povečano vrednostjo in se sklene z novo povečano vrednostjo, celo če se gibanje ponovi v nespremenjenem obsegu.

Kolikor je \( \KPEB \KPEcrta \KPED \) za kupca \( \KPED \KPEcrta \KPEB \) in \( \KPED \KPEcrta \KPEB \) za prodajalca \( \KPEB \KPEcrta \KPED \), je cirkulacija kapitala zgolj navadna blagovna metamorfoza in veljajo v zvezi z njo razloženi zakoni o količini krožečega denarja (I.\ knjiga, 3.\ poglavje, 2). Če pa se ne držimo te formalne plati, ampak raziščemo stvarno povezanost metamorfoz različnih posameznih kapitalov, dejansko torej povezanost krožnih tokov posameznih kapilalov kot delnih gibanj reprodukcijskega procesa celotnega družbenega \KPEstran kapitala, pa ga ne moremo pojasniti z enostavno menjavo oblik denarja in blaga.

V krogu, ki se nenehno vrti, je vsaka točka hkrati izhodiščna in povratna. Če vrtenje prekinemo, ni več vsaka izhodiščna točka tudi povratna. Tako smo videli ne le, da vsak posebni krožni tok (implicite) predpostavlja drugega, ampak tudi, da vključuje ponovitev krožnega toka v eni obliki pot krožnega toka v drugih oblikah. Tako se pokaže vsa razlika kot izključno formalna ali tudi kot izključno subjektivna, takšna, ki obstaja samo za opazovalca.

Kolikor opazujemo vsakega teh krožnih tokov kot posebno obliko gibanja, v katerem so različni individualni industrijski kapitali, obstaja tudi ta različnost vedno le kot individualna. V resnici pa je vsak posamezni industrijski kapital hkrati v vseh treh krožnih tokih. Vsi trije krožni toki, reprodukcijske oblike treh podob kapitala, potekajo neprekinjeno drug poleg drugega. Del vrednosti kapitala, na primer, ki deluje sedaj kot blagovni kapital, se spremeni v denarni kapital, istočasno pa vstopi drug del iz produkcijskega procesa v cirkulacijo kot nov blagovni kapital. Tako se krožna oblika \( \KPEB' \KPEpike \KPEB' \) nikoli ne pretrga; prav tako ne drugi dve obliki. Reprodukcija kapitala v vsaki od njegovih oblik in v vsakem njegovem stadiju je prav tako neprekinjena kakor metamorfoza teh oblik in zaporedni pretek skozi tri stadije. Tu je torej celotni krožni tok dejanska enotnost njegovih treh oblik.

V naši raziskavi smo predpostavljali, da nastopa vrednost kapitala v svoji celotni velikosti vsa bodisi kot denarni kapital, bodisi kot produktivni kapital, bodisi kot blagovni kapital. Tako smo npr. imeli 422\ f.\ št. najprej v celoti kot denarni kapital, potem v vsem svojem obsegu spremenjene v produktivni kapital, nazadnje pa kot blagovni kapital: prejo v vrednosti 500\ f.\ št. (od tega 78\ f.\ št. presežna vrednost). Tu je prav toliko prekinitev, kolikor je različnih stadijev. Dokler na primer ostane 422\ f.\ št. v denarni obliki, se pravi, dokler niso opravljeni nakupi \( \KPED \KPEcrta \KPEB (\KPEDs \) + \( \KPEPs) \), obstaja in deluje celotni kapital samo kot denarni kapital. Kakor hitro se spremeni v produktivni kapital, ne deluje niti \KPEstran kot denarni niti kot blagovni. Njegov celotni proces cirkulacije se prekine, prav tako kakor se prekine po drugi strani njegov celotni produkcijski proces, če deluje v enem od obeh cirkulacijskih stadijev bodisi kot \( \KPED \) ali kot \( \KPEB' \). Tako krožni tok \( \KPEP \KPEpike \KPEP \) torej ne bi pomenil samo občasne obnovitve produktivnega kapitala, ampak prav tako prekinitev njegovega delovanja, produkcijskega procesa, vse dotlej, dokler ne bi bil opravljen cirkulacijski proces; namesto neprekinjeno bi potekala produkcija sunkoma in se obnavljala le po naključno dolgih časovnih obdobjih, odvisnih od tega, ali se dokončata oba stadija cirkulacijskega procesa hitreje ali počasneje. Tako je na primer pri kitajskem rokodelcu, ki dela le za zasebne odjemalce: njegov produkcijski proces se prekine, dokler ne dobi novega naročila.

V resnici velja to za vsak posamezni del kapitala v gibanju; vsi deli kapitala po vrsti opravijo to gibanje. Vzemimo, da je tedenski produkt kakega predilca 10.000 funtov preje. Teh 10.000 funtov preje v celoti izstopi iz produkcijske sfere v cirkulacijsko; v njih vsebovana vrednost kapitala se mora v celoti spremeniti v denarni kapital, in dokler ostane v obliki denarnega kapitala, ne more znova vstopiti v produkcijski proces; poprej mora v cirkulacijo in se spremeniti nazaj v elemente produktivnega kapitala \( \KPEDs \) + \( \KPEPs \). Krožni tok kapitala je neprestano prekinjanje, zapuščanje enega stadija, vstopanje v naslednjega; odlaganje ene oblike, bivanje v drugi; vsak teh stadijev je ne le pogoj naslednjega, ampak ga tudi izključuje.

Vendar je neprekinjenost kljub temu značilna za kapitalistično produkcijo. Čeravno je ne more doseči vedno in brezpogojno, jo omogoča njena tehnična podlaga. Poglejmo torej, kako poteka zadeva v resničnosti. Medtem ko stopa na primer 10.000 funtov preje kot blagovni kapital na trg in se spreminja v denar (bodi to plačilno ali kupno sredstvo ali celo le računski denar), stopa na njihovo mesto v produkcijskem procesu nov bombaž, premog ipd., se je torej že povrnil iz denarne in blagovne oblike v obliko produktivnega kapitala in pričenja kot tak svojo funkcijo. Vtem ko \KPEstran se prvih 10.000 funtov preje spreminja v denar, opravlja prejšnjih 10.000 funtov preje že drugi stadij svoje cirkulacije in se spreminja iz denarja nazaj v elemente produktivnega kapitala. Vsi deli kapitala opravijo proces krožnega toka po vrsti, so hkrati v različnih njegovih stadijih. Tako je industrijski kapital v svojem nepretrganem krožnem toku istočasno v vseh svojih stadijih in njim ustreznih različnih funkcijskih oblikah. Za del, ki se prvič spreminja iz blagovnega kapitala v denar, se krožni tok \( \KPEB' \KPEpike \KPEB' \) pričenja, medtem ko je za industrijski kapital kot gibljivo celoto krožni tok \( \KPEB' \KPEpike \KPEB' \) že opravljen. Z eno roko se denar zalaga, z drugo sprejema; začetek krožnega toka \( \KPED \KPEpike \KPED' \) na eni točki je hkrati njegova vrnitev na drugi. Isto velja za produktivni kapital.

Dejanski krožni tok industrijskega kapitala v svoji nepretrganosti torej ni samo enotnost cirkulacijskega in produkcijskega procesa, ampak tudi enotnost vseh treh njegovih krožnih tokov. Ta pa more biti takšna enotnost samo, če lahko preteče vsak različni del kapitala zaporedoma drugo drugi sledeče si faze krožnega toka, če lahko preide iz ene faze, ene funkcijske oblike v drugo, če je torej industrijski kapital kot celota teh delov istočasno v različnih fazah in funkcijah in tako istočasno opisuje vse tri krožne toke. Vsak del lahko tu zaporedoma sledi drugemu zaradi sočasnega obstoja vseh delov drugega ob drugem, se pravi zaradi delitve kapitala. Tako je v razčlenjenem tovarniškem sistemu produkt prav tako nenehno na različnih stopnjah procesa svojega tvorjenja kakor pri prehajanju iz ene produkcijske faze v drugo. Ker ima posamezni industrijski kapital določeno velikost, ki je odvisna od kapitalistovih sredstev in ki ima za vsako industrijsko panogo neki minimalen obseg, se mora deliti v določenih sorazmerjih. Velikost razpoložljivega kapitala določa obseg produkcijskega procesa, ta določa obseg blagovnega in denarnega kapitala, kolikor delujeta poleg produkcijskega procesa. Sočasni obstoj, od katerega je odvisna nepretrganost produkcije, pa je mogoč le s takšnim gibanjem delov kapitala, v katerem drug za drugim opišejo različne stadije. Sočasni \KPEstran obstoj sam je le rezultat zaporednosti. Če se npr. zaustavi \( \KPEB' \KPEcrta \KPED' \) enega dela, če se blago ne more prodati, tedaj se krožni tok tega dela pretrga in njegova produkcijska sredstva se ne nadomeste; naslednji deli, ki prihajajo kot \( \KPEB' \) iz produkcijskega procesa, vidijo, da so jim zaprli njihovi predniki menjavo njihove funkcije. Če to traja nekaj časa, se produkcija skrči in ves proces se zaustavi. Vsak zastoj v zaporednosti spravi v nered sočasni obstoj, vsak zastoj v enem stadiju povzroči večji ali manjši zastoj v celotnem krožnem toku ne samo zastalega dela kapitala, ampak tudi celotnega individualnega kapitala.

Naslednja oblika, v kateri se pokaže proces, je oblika zaporedja faz: prehod kapitala v novo fazo je mogoč samo, če zapusti prejšnjo. Zato tudi ima vsak posebni krožni tok eno od funkcijskih oblik kapitala za izhodiščno in povratno točko. Na drugi strani je celotni proces dejansko enota ireh krožnih tokov, ki so tiste različne oblike, v katerih se izraža nepretrganost procesa. Za vsako od funkcijskih oblik kapitala je celotni krožni tok njen posebni krožni tok; pri tem je nepretrganost celotnega procesa odvisna od vsakega teh krožnih tokov: krožni tok ene funkcionalne oblike omogoča krožni tok druge. Nujni pogoj celotnega produkcijskega procesa, zlasti družbenega kapitala, je, da je hkrati reprodukcijski proces in zato tudi krožni tok vsakega od njegovih momentov. Različni deli kapitala gredo postopoma skozi različne stadije in funkcijske oblike. Čeprav je v njej vsak čas drug del kapitala, opravlja tako vsaka funkcijska oblika svoj lastni krožni tok hkrati z drugimi. Del kapitala, vendar neprenehoma menjujoč se, neprenehoma reproduciran, obstaja kot blagovni kapital, ki se spreminja v denar; drugi del kot denarni kapital, ki se spreminja v produktivnega; tretji kot produktivni kapital, ki se spreminja v blagovnega. Krožni tok celotnega kapitala prav preko teh treh faz omogoča trajno pričujočnost vseh treh oblik.

Kot celota obstaja zato kapital sočasno v vseh svojih različnih fazah, razporejenih druga poleg druge. Vsak del pa nenehno prehaja po vrsti iz ene faze, ene funkcijske oblike \KPEstran v drugo, deluje tako po vrsti v vseh. Tako so oblike tekoče oblike, katerih zaporednost posreduje njihovo istočasnost. Vsaka oblika sledi drugi in gre pred njo, tako da je odvisna vrnitev vsakega dela kapitala v kako obliko od vrnitve drugega dela v drugo obliko. Vsak del opisuje nenehoma svoj lastni tok; vendar je v tej obliki vedno drug del kapitala; ti posebni obtoki pa so le istočasni in zaporedni členi celotnega poteka.

Nepretrganost celotnega procesa uresničuje ob zgoraj opisanem prekinjanju samo enotnost vseh treh krožnih tokov. Celotni družbeni kapital je vedno nepretrgan, njegov proces pa je vedno enotnost vseh treh krožnih tokov.

Pri individualnih kapitalih se reprodukcija tu pa tam bolj ali manj prekine. Prvič so pogosto količine vrednosti v različnih obdobjih razdeljene v neenakih delih na različne stadije in funkcijske oblike. Drugič se lahko porazdelijo ti deli različno skladno z naravo blaga, ki naj se producira, torej v skladu s posebnim produkcijskim področjem, v katerem je naložen kapital. Tretjič se neprekinjenost lahko bolj ali manj pretrga v produkcijskih panogah, ki so odvisne od letnega časa, bodisi zaradi naravnih pogojev (poljedelstvo, lov na sledi ipd.) ali pa zaradi konvencionalnih okoliščin, kakor pri tako imenovanih sezonskih delih. Najbolj redno in najbolj enakomerno poteka proces v tovarni in v rudarstvu. Vendar pa ne povzroča ta različnost produkcijskih panog nobene razlike v splošnih oblikah procesa krožnega toka.

Kot vrednost, ki se povečuje, kapital ne vključuje le razrednih odnosov, določenega družbenega značaja, ki temelji na obstoju dela kot mezdnega dela. Je gibanje, proces krožnega toka preko različnih stadijev, ki vključuje sam spet tri različne oblike procesa krožnega toka. Zato ga je mogoče razumeti le kot gibanje in ne kot mirujočo stvar. Tisti, ki vidijo v osamosvojitvi vrednosti zgolj abstrakcijo, pozabljajo, da je gibanje industrijskega kapitala ta abstrakcija in actu. Vrednost gre tu skozi različne oblike, različna gibanja, v katerih se ohranja in obenem povečuje. Ker imamo tu opravka predvsem z obliko gibanja, nam ne gre \KPEstran za prevrate, ki jih lahko pretrpi vrednost kapitala v procesu svojega krožnega toka; jasno pa je, da lahko kljub vsem prevratom v vrednosti obstaja in se nadaljuje kapitalistična produkcija le tako dolgo, dokler se vrednost kapitala povečuje, se pravi, dokler opisuje proces svojega kroženja kot osamosvojena vrednost, dokler je torej mogoče prevrate v vrednosti na tak ali drugačen način premagovati in izravnavati. Gibanja kapitala se kažejo kot dejanja posameznega industrijskega kapitalista, ker deluje kot kupec blaga in dela, prodajalec blaga in produktivni kapitalist, ker torej s svojo dejavnostjo poganja krožni tok. Če pride do prevrata v vrednosti družbenega kapitala, se lahko zgodi, da mu njegov individualni kapital podleže in da propade, ker ne more izpolniti pogojev tega gibanja vrednosti. Čim ostrejši in pogostnejši postajajo prevrati v vrednosti, tem bolj se uveljavlja nasproti predvidevanju in preračunanosti posameznih kapitalistov avtomatično gibanje osamosvojene vrednosti, ki deluje s silo elementarnega naravnega procesa, tem bolj podlega tok normalne produkcije nenormalni špekulaciji, tem večja postane nevarnost za obstoj posameznih kapitalov. Ti občasni prevrati potrjujejo torej prav tisto, kar naj bi baje ovrgli; da se vrednost kot kapital osamosvoji in da gibanje kapitala to osamosvojitev ohranja in zaostruje.

V tem zaporedju metamorfoz delujočega kapitala se nenehno primerja velikost vrednosti kapitala, ki se v krožnem toku spreminja, s prvotno vrednostjo. Če se začne vrednost osamosvajati nasproti sili, ki ustvarja vrednost, delovni sili, z aktom \( \KPED \KPEcrta \KPEDs \) (nakupom delovne sile), med produkcijskim procesom pa se z izkoriščanjem delovne sile izvede, se v krožnem toku, v katerem so denar, blago, produkcijski elementi le izmenične oblike delujoče vrednosti kapitala in se primerja pretekla velikost vrednosti kapitala z njegovo sedanjo spremenjeno vrednostjo, vrednost nič več ne osamosvaja.

»Vrednost,« ugovarja Bailey proti pojmu osamosvojene vrednosti kot značilnosti kapitalističnega produkcijskega načina, ki ga šteje za iluzijo nekih ekonomistov, »vrednost je \KPEstran razmerje med istočasno obstoječimi dobrinami, ker je mogoče samo takšne med seboj menjavati.« To pravi Bailey proti primerjanju blagovnih vrednosti v različnih časovnih obdobjih, ki pomeni, brž ko poznamo vrednost denarja za vsako obdobje, samo primerjanje porabe dela, ki je potrebna za produkcijo iste vrste blaga v različnih obdobjih. To izvira iz njegovega splošnega zmotnega mišljenja, po katerem je menjalna vrednost isto kot vrednost, oblika vrednosti vrednost sama, zaradi česar blagovnih vrednosti ni mogoče primerjati, kakor hitro ne delujejo več aktivno kot menjalne vrednosti, kakor hitro se torej ne morejo resnično menjavati med seboj. Niti najmanj ne sluti, da deluje vrednost kot vrednost kapitala ali kot kapital le, če se v različnih fazah svojega krožnega toka, ki nikakor niso istočasne, ampak slede druga drugi, prav nič ne spreminja in se primerja s seboj samo.

Če hočemo proučevati obrazec krožnega toka v vsej jasnosti, ni dovolj predpostavljati, da se prodaja blago po svoji vrednosti, ampak je treba dodati, da se to dogaja pri sicer neizpremenjenih okoliščinah. Vzemimo za primer obliko \( \KPEP \KPEpike \KPEP \), ne glede na tehnične prevrate za časa produkcijskega procesa, ki lahko razvrednotijo produktivni kapital določenega procesa, in prav tako ne glede na vsak vzvratni učinek, ki ga ima lahko sprememba vrednostnih elementov produktivnega kapitala na vrednost obstoječega blagovnega kapitala, katera se lahko poveča ali zmanjša, če je blago v zalogi. Recimo, da se proda \( \KPEB' \), 10.000 funtov preje, po svoji vrednosti 500\ f.\ št.; 8440 funtov preje = 422\ f.\ št. nadomesti v \( \KPEB' \) vsebovano vrednost kapitala. Če pa vrednost bombaža, premoga itd. naraste (ker tu ne upoštevamo nihanja cen), teh 422\ f.\ št. lahko ne zadošča, da bi v celoti nadomestili elemente produktivnega kapitala; potreben je dodaten denarni kapital; denarni kapital se veže. In nasprotno, če se te cene znižajo; denarni kapital se sprošča. Popolnoma normalno poteka proces samo, če ostanejo razmerja vrednosti konstantna; dejansko poteka normalno, dokler se motnje v ponavljanju krožnega toka izravnavajo; čim večje so motnje, tem večji denarni kapital mora imeti \KPEstran industrijski kapitalist, da jih lahko izravna; in ker se z napredkom kapitalistične produkcije povečuje obseg vsakega posameznega produkcijskega procesa in z njim vred minimalna velikost kapitala, ki ga je treba založiti, je to še ena okoliščina, ki spreminja funkcijo industrijskega kapitalista čedalje bolj v monopol velikih denarnih kapitalistov, posameznih ali združenih.

Mimogrede je treba pri tem pripomniti naslednje: če se spremeni vrednost produkcijskih elementov, se pokaže razlika med obliko \( \KPED \KPEpike \KPED' \) na eni strani in oblikama \( \KPEP \KPEpike \KPEP \) ter \( \KPEB \KPEpike \KPEB' \) na drugi.

V \( \KPED \KPEpike \KPED' \) kot obrazcu novo naloženega kapitala, ki prvič nastopa kot denarni kapital, bo zaradi znižanja vrednosti produkcijskih sredstev, na primer surovin, pomožnih tvarin itd., potrebno izdati za začetek podjetja nekega obsega manj denarnega kapitala kot pred znižanjem, ker je odvisen obseg produkcijskega procesa (pri nespremenjeni ravni produktivnih sil) od množine in obsega produkcijskih sredstev, ki jih lahko obvlada neka količina delovne sile; ne pa niti od vrednosti teh produkcijskih sredstev niti od vrednosti delovne sile (ta vpliva le na velikost povečanja vrednosti). In obratno. Če se poveča vrednost produkcijskih elementov tistega blaga, ki tvorijo elemente produktivnega kapitala, je potrebno več denarnega kapitala za osnovanje podjetja določene velikosti. V obeh primerih je prizadeta samo količina denarnega kapitala, ki ga je treba naložiti; v prvem je denarnega kapitala preveč, v drugem pa je potreben dodaten denarni kapital, če postavimo, da priraščajo v ustrezni produkcijski panogi novi posamezni industrijski kapitali kakor ponavadi.

Krožna toka \( \KPEP \KPEpike \KPEP \) in \( \KPEB' \KPEpike \KPEB' \) nastopata kot \( \KPED \KPEpike \KPED' \) le toliko, kolikor je gibanje  \( \KPEP \) in \( \KPEB' \) obenem akumulacija, kolikor se torej dodatni \( \KPEd \), denar, spreminja v denarni kapital. Če odmislimo to, pa ju sprememba vrednosti prvin produktivnega kapitala prizadene drugače, kot prizadene \( \KPED \KPEpike \KPED' \); pri tem zopet odmišljamo vzvratne učinke takšne spremembe vrednosti na sestavne dele kapitala, ki jih \KPEstran obsega produkcijski proces. Pri tem ni neposredno prizadeta prvotna naložba kapitala, ampak industrijski kapital v svojem reprodukcijskem procesu in ne v svojem prvem krožnem toku; torej \( \KPEB' \KPEpike \KPEBrazcepDsPs \), ponovna pretvorba blagovnega kapitala v njegove produkcijske elemente, kolikor so blago. Pri zmanjšanju vrednosti (oziroma cene) so možni trije primeri: reprodukcijski proces se nadaljuje v istem obsegu; potem se del dotedanjega denarnega kapitala sprosti in denarni kapital se kopiči brez dejanske akumulacije (produkcije v razširjenem obsegu) oziroma spremembe \( \KPEd \) (presežne vrednosti) v akumulacijski sklad, ki dejansko akumulacijo uvaja in spremlja; ali pa se, če dopuščajo to tehnična razmerja, razširi reprodukcijski obseg v večjem obsegu, kakor bi se sicer; ali pa nastanejo večje zaloge surovin itd.

Nasprotno pri povečanju vrednosti nadomestnih elementov blagovnega kapitala. Reprodukcija se potem ne razvije več v svojem normalnem obsegu (dela se, na primer, krajši čas); ali pa mora vstopiti dodatni denarni kapital, da bi se lahko nadaljevala v svojem dotedanjem obsegu (vezanje denarnega kapitala); ali pa rabi denarni akumulacijski sklad, če ga je kaj, v celoti ali delno namesto za razširitev reprodukcijskega procesa za njegovo obratovanje v prejšnjem obsegu. Tudi to je vezanje denarnega kapitala, le da tu dodatni denarni kapital ne prihaja od zunaj, z denarnega trga, ampak iz sredstev industrijskega kapitalista samega.

Pri \( \KPEP \KPEpike \KPEP \) in pri \( \KPEB' \KPEpike \KPEB' \) pa lahko nastopijo spreminjevalne okoliščine. Če ima na primer naš predilec bombaža veliko zalogo bombaža (torej velik del svojega produktivnega kapitala v obliki zaloge bombaža), razvrednoti znižanje cen bombaža del njegovega produktivnega kapitala; če pa se cene zvišajo, se poveča vrednost tega dela njegovega produktivnega kapitala. Če je, po drugi strani, nakopičil velike količine v obliki blagovnega kapitala, na primer v bombažni preji, se z znižanjem bombaža razvrednoti del njegovega blagovnega kapitala, torej nasploh njegovega kapitala, ki \KPEstran se zadržuje v krožnem toku; nasprotno pri zvišanju cen bombaža. Končno v procesu \( \KPEB' \KPEcrta \KPED \KPEcrta \KPEBrazcepDsPs \): če je prišlo do \( \KPEB' \KPEcrta \KPED \), realizacije blagovnega kapitala, pred spremembo vrednosti elementov \( \KPEB \), je prizadet kapital samo na način, ki smo ga opazovali v prvem primeru, namreč v drugem cirkulacijskem aktu \( \KPED \KPEcrta \KPEBrazcepDsPs \); če pa se je to zgodilo pred izvedbo \( \KPEB' \KPEcrta \KPED \), povzroči znižanje cene bombaža ob sicer enakih okoliščinah ustrezno znižanje cene preje, zvišanje cene bombaža pa nasprotno zvišanje cene preje. Učinek na različne posamezne kapitale, ki so naloženi v isti produkcijski panogi, je lahko zelo različen glede na različne okoliščine, v katerih se lahko nahajajo. -- Sproščanje in vezanje denarnega kapitala lahko izvirata tudi iz razlik v trajanju cirkulacijskega procesa, torej tudi v hitrosti cirkulacije. Vendar spada to v proučevanje obrata. Tu nas zanima samo stvarna razlika, ki se pokaže glede sprememb vrednosti elementov produktivnega kapitala med \( \KPED \KPEpike \KPED' \) in obema drugima oblikama procesa krožnega toka.

V odseku cirkulacije \( \KPED \KPEcrta \KPEBrazcepDsPs \) bo v razdobju že razvitega, torej prevladujočega kapitalističnega načina produkcije velik del blaga, iz katerega sestoji  \( \KPEPs \), produkcijska sredstva, tuji delujoči blagovni kapital. Z vidika prodajalca pride torej do \( \KPEB' \KPEcrta \KPED' \), spremembe blagovnega kapitala v denarnega. To pa ne velja brez izjeme. Nasprotno! V procesu svoje cirkulacije, kjer deluje industrijski kapital bodisi kot denar, bodisi kot blago, se križa krožni tok industrijskega kapitala, bodisi kot denarni kapital, bodisi kot blagovni, z blagovno cirkulacijo najrazličnejših družbenih načinov produkcije, kolikor so ti hkrati blagovna produkcija. Najsi je blago produkt na suženjstvu temelječe produkcije, kmetov (Kitajcev, indijskih kmetov), skupnosti (Nizozemska Vzhodna Indija), državne produkcije (kakršna se na temelju podložništva pojavlja v zgodnjih obdobjih ruske zgodovine) ali pol divjih lovskih plemen itd.: kot blago in denar se srečuje z denarjem in blagom, v katerem nastopa industrijski \KPEstran kapital, in vstopa prav tako v njegov krožni tok kakor v krožni tok presežne vrednosti, ki jo vsebuje blagovni kapital, kolikor se troši kot dohodek; torej v obe veji cirkulacije blagovnega kapitala. Značaj produkcijskega procesa, iz katerega izhaja, je čisto nepomemben; kot blago nastopa na trgu, kot blago prihaja v krožni tok industrijskega kapitala kakor tudi v cirkulacijo presežne vrednosti, ki jo vsebuje. Za cirkulacijski proces industrijskega kapitala je torej značilno, da izvira blago lahko od kjerkoli, da obstoji trg kot svetovni trg. Kar velja glede tujega blaga, velja glede tujega denarja; kakor nastopa nasproti njemu blagovni kapital le kot blago, nastopa ta denar nasproti njemu le kot denar; denar deluje tu kot svetovni denar.

Pri tem pa je treba pripomniti dvoje.

Prvič. Kakor hitro se opravi akt  \( \KPED \KPEcrta \KPEPs \), preneha biti blago (\( \KPEPs \)) blago in postane eden od načinov eksistence industrijskega kapitala v njegovi funkcijski obliki \( \KPEP \), produktivni kapital. S tem pa je zbrisan njegov izvor; obstaja samo še kot oblika obstoja industrijskega kapitala, je v njem utelešeno. Kljub temu pa velja, da ga je treba reproducirati, če naj se nadomesti, in v tej meri je kapitalistični način produkcije odvisen od produkcijskih načinov, ki niso na njegovi razvojni stopnji. Teži pa za tem, da bi spremenil vso produkcijo čimbolj v blagovno produkcijo; glavno njegovo sredstvo, s katerim si pri tem pomaga, je ravno pritegovanje teh produkcijskih mačinov v njegov cirkulacijski proces; razvita blagovna produkcija pa pomeni sama po sebi kapitalistično blagovno produkcijo. Nastop industrijskega kapitala pospešuje to spremembo vsepovsod, z njo vred pa tudi spremembo vseh neposrednih producentov v mezdne delavce.

Drugič. Najsi je njegov izvor, družbena oblika produkcijskega procesa, iz katerega izvira, takšen ali drugačen, vedno se srečuje blago, ki prihaja v cirkulacijski proces industrijskega kapitala (s\`em spadajo tudi nujne življenjske potrebščine, v katere se spremeni variabilni kapital, ko se za reprodukcijo delovne sile izplača delavcem), z industrijskim \KPEstran kapitalom samim že v obliki blagovnega kapitala, v obliki trgovinskega ali trgovskega kapitala; to pa že po svoji naravi obsega blago vseh produkcijskih načinov.

Tako kot predpostavlja kapitalistični produkcijski način velik obseg produkcije, predpostavlja nujno tudi velik obseg prodaje; torej prodajo trgovcu namesto posameznemu konsumentu. Kolikor je ta konsument produktivni konsument, torej industrijski kapitalist, kolikor torej dobavlja industrijski kapital ene produkcijske panoge drugi panogi produkcijska sredstva, imamo tudi (v obliki naročil itd.) neposredno prodajo enega industrijskega kapitalista številnim drugim. V tej meri je vsak industrijski kapitalist neposredni prodajalec, sam svoj trgovec; sicer pa je to tudi pri prodaji trgovcu.

Razvoj kapitalistične produkcije predpostavlja in vedno bolj razvija blagovno trgovino kot funkcijo trgovskega kapitala. Tu pa tam jo zato vzamemo za ponazoritev posameznih strani kapitalističnega procesa cirkulacije; pri njegovi splošni analizi pa si zamišljamo, da poteka prodaja brez posredovanja trgovca, ker le-to prikriva različne pojave gibanja.

Poglejmo Sismondija, ki prikazuje zadevo nekoliko naivno:

»Trgovina uporablja precejšen kapital, za katerega se na prvi pogled zdi, da ni del kapitala, katerega gibanje smo podrobno orisali. Na prvi pogled se zdi, da nima vrednost sukna, ki je nakopičeno v skladiščih trgovca s suknom, nobene zveze s tistim delom letnega produkta, ki ga d\'a bogataš revežu kot mezdo zato, da dela. Vendar pa je ta kapital le nadomestil tistega, o katerem smo govorili. Da bi čim jasneje doumeli nastajanje bogastva, smo ga zasledovali od nastanka do konsumpcije. Pri tem se nam je zdel kapital, ki se uporablja na primer v izdelavi sukna, vedno isti; pri zamenjavi za dohodek konsumenta se je razdelil samo na dva dela: eden je bil kot profit dohodek tovarnarja, drugi pa kot mezda dohodek delavcev, ko so izdelovali sukno.

Kmalu \KPEstran pa so odkrili, da je v korist vseh bolje, če se različni deli tega kapitala medsebojno nadomeščajo in če se 100.000 tolarjev, ki zadoščajo za vso cirkulacijo med tovarnarjem in potrošnikom, v enakih delih porazdele med tovarnarja, trgovca na debelo in trgovca na drobno. Prvi je samo s tretjino opravil isti posel, kakor bi ga bil opravil s celoto, ker je našel v trenutku, ko je dokončal svojo produkcijo, kot kupca dosti prej trgovca, kakor bi našel potrošnika. Prav tako se je kapital trgovca na debelo veliko hitreje nadomestil s kapitalom trgovca na drobno ... Razlika med vsoto založenih mezd in kupno ceno, ki jo plača zadnji konsument, je dobiček kapitalov. Ker so si razdelili med seboj njihove funkcije in ker je bilo opravljeno delo isto, čeprav so bili zaposleni namesto enega trije ljudje in trije delni kapitali, se razdeli med tovarnarja, trgovca in prodajalca na drobno.« (»Nouveaux Principes«, I. del., str. 139, 140.) -- »Vsi (trgovci) so posredno sodelovali v produkciji; ker je smisel produkcije poraba, je namreč ni mogoče šteti za dokončano, dokler ne da produkta na razpolago konsumentu.« (Prav tam, str. 137.)

Pri proučevanju splošnih oblik krožnega toka in v vsej tej drugi knjigi nasploh štejemo za denar kovinski denar in izključujemo simbolični denar, vrednostne znake, ki so posebnost samo nekaterih držav, kakor tudi kreditni denar, ki ga še nismo obravnavali. Predvsem je zgodovinski razvoj tak: kreditni denar ne igra nobene ali pa le nepomembno vlogo v prvem razdobju kapitalistične produkcije. Razen tega pa dokazuje nujnost takšnega razvoja teoretično tudi to, da vse, kar so doslej kritičnega povedali o cirkulaciji kreditnega denarja, sili Tooka in druge vedno znovič nazaj k proučevanju, kakšna bi bila stvar, če bi bil obtok izključno kovinski. Ne smemo pa pozabiti, da lahko služi kovani denar tako za kupno sredstvo kakor tudi za plačilno sredstvo. Zaradi poenostavitve ga jemljemo v tej drugi knjigi na splošno le v njegovi prvi funkcionalni obliki.

Kolikor je cirkulacijski proces industrijskega kapitala, ki tvori le del njegovega individualnega procesa krožnega toka, samo zaporedje dogodkov v splošni blagovni cirkulaciji \KPEstran sami, odločajo o njem poprej (I.\ knjiga, 3.\ poglavje) razloženi splošni zakoni. Ista količina denarja, na primer 500\ f.\ št., spravlja toliko več industrijskih kapitalov (ali tudi posamičnih kapitalov v njihovi obliki blagovnih kapitalov) drugega za drugim v cirkulacijo, kolikor večja je obtočna hitrost denarja, kolikor hitreje gre torej vsak posamezni kapital skozi vrsto svojih blagovnih ali denarnih metamorfoz. Po vrednosti enaka množina kapitala zahteva zategadelj toliko manj denarja za svojo cirkulacijo, kolikor bolj deluje denar kot plačilno sredstvo, kolikor bolj je treba, na primer pri nadomestitvi blagovnega kapitala z njegovimi produkcijskimi sredstvi, plačati samo salda in kolikor krajši so plačilni roki, na primer pri plačilu mezd. Če si mislimo, da so hitrost cirkulacije in vse druge okoliščine nespremenjene, določa po drugi strani količino denarja, ki mora cirkulirati kot denarni kapital, vsota blagovnih cen (cena pomnožena s količino blaga) ali, pri dani množini in vrednosti blaga, vrednost denarja sama.

Toda zakoni splošne cirkulacije blaga veljajo le, kolikor je proces cirkulacije kapitala vrsta enostavnih cirkulacijskih aktov, ne pa, kolikor tvorijo ti funkcionalno določene odseke krožnega toka posameznih industrijskih kapitalov.

Da bi to pojasnili, je najprimerneje pogledati cirkulacijski proces v njegovi nepretrgani povezanosti, kakor nastopa v obeh oblikah:

\[
    \begin{array}{l r l}
        \textrm{II)} & \KPEP \KPEpike \KPEB' & \left\lbrace
        \begin{array}{c c}
            \KPEB & \KPEcrta \\
            \KPEcrta & \KPED' \\
            \KPEb & \KPEcrta \\
        \end{array}
        \right.
        \left\lbrace
        \begin{array}{c c l}
            \KPED & \KPEcrta & \KPEBrazcepDsPs \KPEpike \KPEP (\KPEP') \\
             & & \\
            \KPEd & \KPEcrta & \KPEb \\
        \end{array}
        \right. \\
         & & \\
        \textrm{III)} & \KPEB' & \left\lbrace
        \begin{array}{c c}
            \KPEB & \KPEcrta \\
            \KPEcrta & \KPED' \\
            \KPEb & \KPEcrta \\
        \end{array}
        \right.
        \left\lbrace
        \begin{array}{c c l}
            \KPED & \KPEcrta & \KPEBrazcepDsPs \KPEpike \KPEP \KPEpike \KPEB' \\
             & & \\
            \KPEd & \KPEcrta & \KPEb \\
        \end{array}
        \right. \\
    \end{array}
\]

Kot vrsta cirkulacijskih aktov kot takih pomeni cirkulacijski proces (bodisi kot \( \KPEB \KPEcrta \KPED \KPEcrta \KPEB \) ali kot \( \KPED \KPEcrta \KPEB \KPEcrta \KPED \) samo obe nasprotni vrsti blagovnih metamorfoz, vsaka posamezna \KPEstran metamorfoza obeh vrst pa vključuje spet nasprotno metamorfozo tujega blaga ali tujega denarja, ki ji stoji nasproti.

Kar je \( \KPEB \KPEcrta \KPED \) za lastnika blaga, je \( \KPED \KPEcrta \KPEB \) za kupca; prva metamorfoza blaga v \( \KPEB \KPEcrta \KPED \) je druga metamorfoza blaga, ki nastopa kot \( \KPED \); nasprotno v \( \KPED \KPEcrta \KPEB \). To, kar smo torej povedali o prepletanju metamorfoze blaga v enem stadiju z metamorfozo drugega blaga v drugem stadiju, velja za cirkulacijo kapitala, kolikor nastopa kapitalist kot kupec in prodajalec blaga, torej njegov kapital nasproti tujemu blagu kot denar ali nasproti tujemu denarju kot blago. Ni pa to prepletanje obenem tudi izraz prepletanja metamorfoz kapitalov.

Prvič. \( \KPED \KPEcrta \KPEB \) (\( \KPEPs \)) lahko pomeni, kakor smo videli, preplet metamorfoz različnih posameznih kapitalov. Vzemimo, da se blagovni kapital predilca bombaža preja delno nadomešča s premogom. Del njegovega kapitala ima denarno obliko in se spreminja iz nje v blagovno obliko, medtem ko ima kapital kapitalističnega producenta premoga blagovno obliko in se torej pretvarja v denarno obliko; isti cirkulacijski akt pomeni tu nasprotni metamorfozi dveh industrijskih kapitalov (ki pripadata različnima produkcijskima panogama), torej preplet zaporedja metamorfoz teh kapitalov. Kakor pa smo videli, ni treba, da bi bil  \( \KPEPs \), v katerega se spreminja \( \KPED \), blagovni kapital v kategoričnem smislu, se pravi funkcionalna oblika industrijskega kapitala, produkt kakšnega kapitalista. Vedno imamo \( \KPED \KPEcrta \KPEB \) na eni strani, \( \KPEB \KPEcrta \KPED \) na drugi, nimamo pa vedno prepleta metamorfoz kapitala. Nadalje,  \( \KPED \KPEcrta \KPEDs \), nakup delovne sile, nikdar ni preplet metamorfoz kapitala, ker je delovna sila sicer delavčevo blago, kapital pa postane šele, ko jo proda kapitalistu. Na drugi strani ni  \( \KPED' \) v procesu \( \KPEB' \KPEcrta \KPED' \) nujno spremenjeni blagovni kapital; lahko izvira iz vnovčitve blaga delovna sila (mezda) ali produkta, ki ga je izdelal samostojen delavec, suženj, tlačan ali skupnost.

Drugič. Nikakor pa ne velja, da bi morala imeti funkcionalno določena vloga, ki jo igra vsaka v cirkulacijskem procesu posameznega kapitala nastopajoča metamorfoza, ustrezno \KPEstran nasprotno metamorfozo v krožnem toku drugega kapitala, ob predpostavki seveda, da je vsa produkcija svetovnega trga kapitalistična. V krožnem toku \( \KPEP \KPEpike \KPEP \) je na primer \( \KPED' \), ki je vnovčeni \( \KPEB' \), za kupca lahko le vnovčenje njegove presežne vrednosti (če je blago konsumni predmet); ali pa lahko vstopi v \( \KPED' \KPEcrta \KPEBrazcepDsPsII \) (kamor vstopi torej kapital akumuliran) za prodajalca \( \KPEPs \) le kot nadomestilo za kapital, ki ga je založil, ali pa celo sploh ne vstopi več v cirkulacijo njegovega kapitala, če se namreč odcepi v potrošnjo dohodka.

Iz enostavnih prepletanj metamorfoz blagovne cirkulacije, katerih akti cirkulacije kapitala so enaki kakor akti vsake druge cirkulacije blaga, torej ni mogoče dognati, kako se v procesu cirkulacije medsebojno nadomeščajo -- glede na kapital kakor tudi glede na presežno vrednost -- različni sestavni deli celotnega družbenega kapitala, pri čemer so posamezni kapitali le njegovi sestavni deli, ki samostojno delujejo. Potreben je drugačen način raziskave. Doslej smo se pri tem zadovoljevali s frazami, ki niso pri podrobnejši analizi nič drugega ko nedoločne predstave, kakršne omogočajo le prepletanja metamorfoz, ki so skupna vsem blagovnim cirkulacijam.
\medskip
\hrule
\medskip

Ena izmed najočitnejših posebnosti procesa krožnega toka industrijskega kapitala, torej tudi kapitalistične produkcije, je okoliščina, da prihajajo po eni strani sestavine produktivnega kapitala z blagovnega trga in da jih je treba na njem trajno obnavljati, kupovati kot blago; da izhaja po drugi strani produkt delovnega procesa iz njega kot blago in da se mora vedno znova prodajati kot blago. Primerjajmo na primer modernega spodnješkotskega zakupnika s staromodnim kontinentalnim malim kmetom. Prvi proda ves svoj produkt in mora zato tudi vse njegove elemente, celo seme, obnoviti z nakupom na trgu, drugi pa porabi večji del svojega produkta neposredno, kupuje in prodaja kolikor mogoče malo, izdeluje orodja, oblačila itd., če le mogoče, sam.

V \KPEstran skladu s tem so postavili naturalno gospodarstvo, denarno gospodarstvo in kreditno gospodarstvo drugo nasproti drugemu kot tri značilne ekonomske oblike gibanja družbene produkcije.

Prvič: te tri oblike niso enakovredne razvojne faze. Tako imenovano kreditno gospodarstvo samo je le oblika denarnega gospodarstva, če upoštevamo, da označujeta oba izraza način, kako je organizirana menjava med producenti. V razviti kapitalistični produkciji je denarno gospodarstvo samo še podlaga kreditnega gospodarstva. Denarno in kreditno gospodarstvo sta tako samo izraza različnih razvojnih stopenj kapitalistične produkcije, nikakor pa nista različni samostojni obliki menjave v primerjavi z naturalnim gospodarstvom. S prav isto upravičenostjo bi lahko kot enakovredne postavili nasproti tema dvema zelo različne oblike naturalnega gospodarstva.

Drugič: ker se pri kategorijah denarno gospodarstvo in kreditno gospodarstvo ne poudarja in podčrtuje kot odločilna razlika gospodarstvo, to je produkcijski proces, ampak gospodarstvu ustrezni način menjave med različnimi produkcijskimi agenti ali producenti, bi morali napraviti isto tudi pri prvi kategoriji. Namesto naturalnega gospodarstva torej menjalno gospodarstvo. Popolnoma zaprto naturalno gospodarstvo, na primer perujska država Inkov, ne bi spadalo v nobeno od teh kategorij.

Tretjič: denarno gospodarstvo je skupno vsaki blagovni produkciji, produkt pa nastopa kot blago v najrazličnejših družbenih produkcijskih organizmih. Kapitalistično produkcijo bi označeval torej samo obseg, v katerem se proizvaja produkt kot trgovinski predmet, kot blago, v katerem morajo torej tudi njegovi lastni tvorni elementi priti v gospodarstvo, iz katerega izhaja, kot trgovinski predmeti, kot blago.

\setcounter{footnote}{6}
Kapitalistična produkcija je v resnici blagovna produkcija kot splošna oblika produkcije. To pa je le, in postaja čedalje bolj v svojem razvoju, ker se prodaja tu s\^amo delo kot blago, ker prodaja delavec delo, se pravi delovanje delovne sile, in sicer, kakor predpostavljamo, po vrednosti, ki \KPEstran jo določajo njeni reprodukcijski stroški. V obsegu, v katerem postaja delo mezdno delo, postaja producent industrijski kapitalist; zato se pojavi kapitalistična produkcija (torej tudi blagovna produkcija) v vsem svojem obsegu šele, ko je tudi neposredni kmetijski producent mezdni delavec. V odnosu med kapitalistom in mezdnim delavcem postane denarno razmerje, to je razmerje med kupcem in prodajalcem, razmerje, ki je imanentno sami produkciji. Po svoji osnovi pa temelji to razmerje na družbenem značaju produkcije, ne na načinu menjave; ta izhaja iz onega. Sicer pa ustreza meščanskemu obzorju, kjer so glave polne mešetarjenja, da ne vidi temeljev načina menjave v ustreznem produkcijskem načinu, ampak obratno. \footnote{Do tod rokopis V. -- Kar sledi do konca poglavja, je zapisek, najden v zvezku od 1877 do 1878 med izvlečki iz knjig.}
\medskip
\hrule
\medskip

Kapitalist meče v obliki denarja manjšo vrednost v cirkulacijo, kakor jo iz nje potegne, ker meče v obliki blaga vanjo več vrednosti, kot ji je je odvzel v obliki blaga. Kolikor nastopa le kot poosebljenje kapitala, kot industrijski kapilalist, je njegova ponudba blagovne vrednosti vedno večja ko njegovo povpraševanje po blagovni vrednosti. Če bi se njegova ponudba in njegovo povpraševanje skladala, bi v tej zvezi pomenilo, da njegov kapital ni povečal svoje vrednosti; ne bi deloval kot produktivni kapital; produktivni kapital bi se spremenil v blagovnega, ki ne bi bil oplojen s presežno vrednostjo; med produkcijskim procesom ne bi izvlekel iz delovne sile v blagovni obliki nobene presežne vrednosti, torej sploh ne bi deloval kot kapital. V resnici mora »prodati draže, kot je kupil«. To pa se mu posreči le, ker je s pomočjo kapitalističnega produkcijskega procesa spremenil cenejše, manj vredno blago, ki ga je kupil, v več vredno, torej dražje. Ne prodaja draže, ker bi prodajal preko vrednosti, ampak ker prodaja blago, katerega vrednost je večja kakor vsota vrednosti njegovih produkcijskih sestavin.

Mera, v kateri poveča kapitalist svoj kapital, je tem večja, čim večja je razlika med njegovo ponudbo in njegovim \KPEstran povpraševanjem, se pravi, čim večji je presežek blagovne vrednosti, ki jo ponuja, nad vrednostjo blaga, po katerem povprašuje. Namesto v njuni skladnosti je njegov cilj čim večja neskladnost, presežek njegove ponudbe nad njegovim povpraševanjem.

Kar velja za posameznega kapitalista, velja tudi za kapitalistični razred.

Kolikor kapitalist samo pooseblja industrijski kapital, obstaja njegovo lastno povpraševanje le iz povpraševanja po produkcijskih sredstvih in delovni sili. Njegovo povpraševanje po \( \KPEPs \), merjeno po njihovi vrednosti, je manjše kakor njegov založeni kapital. Produkcijska sredstva kupuje v manjši vrednosti, kakor znaša vrednost njegovega kapitala, in torej v še veliko manjši, kakor jo ima blagovni kapital, katerega ponuja.

Kar zadeva njegovo povpraševanje po delovni sili, določa njegovo vrednost razmerje njegovega variabilnega kapitala do njegovega celotnega kapitala. Torej je enako \( \KPEv \) : \( \KPEC \), zaradi česar je v kapitalistični produkciji sorazmerno vedno manjše kot njegovo povpraševanje po produkcijskih sredstvih. V vedno naraščajoči meri kupuje več  \( \KPEPs \) kot  \( \KPEDs \).

Kolikor zamenjava delavec svojo mezdo po večini za življenjske potrebščine in največ za nujne življenjske potrebščine, je kapitalistovo povpraševanje po delovni sili posredno hkrati povpraševanje po konsumnih sredstvih, ki jih porablja delavski razred. To povpraševanje pa je = \( \KPEv \) in niti za atom večje (če delavec varčuje s svojo mezdo -- tu nujno puščamo vnemar vse kreditne odnose -- pomeni to, da del svoje mezde tezavrira in da pro tanto ne nastopa kot povpraševalec, kot kupec). Zgornja meja kapitalistovega povpraševanja je enaka \( \KPEC \) = \( \KPEc \) + \( \KPEv \), njegova ponudba pa je enaka \( \KPEc \) + \( \KPEv \) + \( \KPEpv \). Če je torej sestava njegovega blagovnega kapitala 80~\( \KPEc \) + 20~\( \KPEv \) + 20~\( \KPEpv \), je njegovo povpraševanje enako 80~\( \KPEc \) + 20~\( \KPEv \), po vrednosti torej za $\frac{1}{5}$ manjše kakor njegova ponudba. Čim večji je odstotek množine \( \KPEpv \) (profitna mera), ki jo producira, tem manjše je njegovo povpraševanje v primeri z njegovo ponudbo. Čeprav postaja kapitalistovo povpraševanje po delovni sili in s tem posredno \KPEstran po nujnih življenjskih potrebščinah z napredkom produkcije čedalje manjše od njegovega povpraševanja po produkcijskih sredstvih, vendar po drugi strani ne smemo pozabiti, da je njegovo povpraševanje po \( \KPEPs \) v povprečju vedno manjše ko njegov kapital. Vrednost njegovega povpraševanja po produkcijskih sredstvih mora biti torej vedno manjša kakor blagovni produkt kapitalista, ki mu dobavlja ta produkcijska sredstva in ki dela z enakim kapitalom in ob enakih okoliščinah. Da so to številni kapitalisti in ne en sam, stvari nič ne spremeni. Če si mislimo, da znaša njegov kapital 1000\ f.\ št., njegov konstantni del pa 800\ f.\ št., tedaj znaša njegovo povpraševanje nasproti njim vsem 800\ f.\ št. Skupno dobavijo na vsakih 1000\ f.\ št. (ne glede na to, koliko od tega pripade vsakemu posameznemu in kakšen del njegovega celotnega kapitala tvori količina, ki pride na vsakega) pri enaki profitni meri za 1200\ f.\ št. vrednosti produkcijskih sredstev. Njegovo povpraševanje krije torej sam\'o $\frac{2}{3}$ njihove ponudbe, medtem ko znaša po vrednosti njegovo celotno povpraševanje le $\frac{4}{5}$ njegove lastne ponudbe.

Zdaj moramo mimogrede vnaprej poseči še v obravnavo obrata. Vzemimo, da znaša njegov celotni kapital 5000\ f.\ št., od tega 4000\ f.\ št. fiksni in 1000\ f.\ št. cirkulirajoči; teh 1000 = 800~\( \KPEc \) + 200~\( \KPEv \) po gornji predpostavki. Njegov cirkulirajoči kapital se mora obrniti petkrat v letu, da se njegov celotni kapital obrne enkrat v letu. Njegov blagovni produkt je potem enak 6000\ f.\ št., je torej za 1000\ f.\ št. večji od njegovega založenega kapitala, kar daje spet isto razmerje presežne vrednosti kakor zgoraj:

5000~\( \KPEC \): 1000~\( \KPEpv \) = 100~(\( \KPEc \) + \( \KPEv \)): 20~\( \KPEpv \). Ta obrat torej prav nič ne spremeni razmerja njegovega celotnega povpraševanja do njegove celotne ponudbe; še vedno je za $\frac{1}{5}$ manjše kot ponudba.

Vzemimo, da se mora njegov fiksni kapital obnoviti v 10 letih. Letno torej odpiše $\frac{1}{10}$ = 400\ f.\ št. Tako ima samo še vrednost 3600\ f.\ št. fiksnega kapitala plus 400\ f.\ št. v denarju. Če so potrebna popravila in ne presegajo povprečne velikosti, ne pomenijo prav nič drugega kot kapital, ki ga nalaga \KPEstran šele naknadno. Zadevo lahko gledamo tako, kakor da bi vračunal stroške popravil takoj pri oceni vrednosti svojega naloženega kapitala, kolikor prehaja v letni blagovni produkt, tako da so vsebovani v $\frac{1}{10}$ odpisov. (Če so njegove potrebe po popravilih v resnici pod povprečjem, profitira, prav kakor ima izgubo, če so nad povprečjem. To pa se izravna za celotni razred kapitalistov, ki delajo v isti industrijski panogi.) Čeprav ostane torej njegovo lastno povpraševanje pri enkratnem letnem obratu celotnega njegovega kapitala 5000\ f.\ št., enako njegovi prvotno založeni kapitalski vrednosti, se v razmerju do cirkulirajočega dela kapitala vsekakor povečuje, medtem ko se v odnosu do njegovega fiksnega dela stalno zmanjšuje.

Preidimo zdaj še k reprodukciji. Vzemimo, da porabi kapitalist vso presežno vrednost \( \KPEm \) in da vnovič prenese v produktivni kapital samo prvotno velikost kapitala \( \KPEC \). Sedaj je kapitalistovo povpraševanje enakovredno njegovi ponudbi. Vendar ne glede na gibanje njegovega kapitala; kot kapitalist povprašuje le po $\frac{4}{5}$ svoje ponudbe (vrednostno); $\frac{1}{5}$ porabi kot nekapitalist, ne v svoji funkciji kapitalista, ampak za svoje zasebne potrebe ali zadovoljstvo.

Računano v odstotkih, je njegov račun potem takle:

\[
    \begin{array}{l l l}
        \textrm{kot kapitalist:} & \textrm{povpraševanje = 100,} & \textrm{ponudba = 120} \\
        \textrm{kot uživalec:} & \textrm{povpraševanje =} \phantom{.} \thinspace \thinspace \thinspace \textrm{20,} & \textrm{ponudba =} \phantom{-.} \thinspace \textrm{0} \\
        \hline
        \textrm{Vsota:} & \textrm{povpraševanje = 120,} & \textrm{ponudba = 120.} \\
    \end{array}
\]

Ta predpostavka je istovetna s predpostavko, da kapitalistična produkcija ne obstaja in da zato tudi industrijski kapitalist sam ne obstaja. Kajti kapitalizem je odpravljen že v osnovi, če predpostavljamo, da je gonilni motiv kapitalizma užitek, ne pa bogatenje samo.

Nemogoča pa je ta predpostavka tudi tehnično. Kapitalist mora ustvarjati ne le rezervni kapital za obrambo pred nihanji cen in zato, da bi lahko pričakal najugodnejše konjunkture za nakup in prodajo; akumulirati mora kapital, da bi lahko razširil produkcijo in pripojil v svoj produktivni organizem nove tehnične iznajdbe in odkritja.

Če \KPEstran hoče akumulirati kapital, mora najprej odtegniti v denarni obliki cirkulaciji del presežne vrednosti, ki mu je pritekla iz cirkulacije, jo zbirati kot zaklad, dokler ne postane tolikšen, kakor je potreben za razširitev že obstoječega podjetja ali za ustanovitev podružničnega. Dokler kapitalist zbira zaklad, ne povečuje svojega povpraševanja; denar je imobiliziran; za denarno protivrednost, ki jo je pobral s trga za prodano blago, ne pobere z njega prav nobene protivrednosti v blagu.

Kredit puščamo pri tem ob strani; za kredit pa se šteje, če npr. kapitalist nalaga denar, kakor ga postopoma nabira, pri kakšni banki na obrestovani tekoči račun.
 
\end{document}
