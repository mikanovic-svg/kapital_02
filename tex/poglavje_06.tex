\documentclass[kapital_02.tex]{subfiles}
\begin{document}
\KPEstran %TODOna naslov poglavja?
\section{Čisti cirkulacijski stroški}
\subsection{Nakupni in prodajni čas}
Oblikovne spremembe kapitala iz blaga v denar in iz denarja v blago so hkrati kapitalistovo trgovanje, akti nakupa in prodaje.
Čas, v katerem se te oblikovne spremembe kapitala izvajajo, je subjektivno, z vidika kapitalista, prodajni čas in nakupni čas, čas, v katerem nastopa na trgu kot prodajalec in kupec.
Kakor je obtočni čas kapitala potreben odsek njegovega reprodukcijskega časa, je čas, v katerem kapitalist kupuje in prodaja, ko se mudi na trgu, potreben odsek njegovega časa, v katerem deluje kot kapitalist, kot poosebljeni kapital. 
Je del njegovega poslovnega časa.

[Ker smo postavili, da se kupuje in prodaja blago po svoji vrednosti, gre pri teh procesih samo za prenos iste vrednosti iz ene oblike v drugo, iz blagovne oblike v denarno in iz denarne v blagovno — za spremembo stanja.
Če se prodaja blago po svoji vrednosti, ostane velikost vrednosti tako v rokah kupca kakor tudi prodajalca nespremenjena; spremeni se samo oblika njenega bivanja.
Če se blago ne prodaja po vrednosti, ostane vsota zamenjanih vrednosti nespremenjena; kar je na eni strani plus, je na drugi minus.

Metamorfozi \(\KPEB\KPEcrta\KPED\) in \(\KPED\KPEcrta\KPEB\) pomenita trgovanje, ki poteka med kupcem in prodajalcem; ta dva potrebujeta čas, da se pogodita, to toliko bolj, ker pri tem bijeta boj, v\KPEstran\ katerem skuša vsaka stranka prelisičiti drugo; drug drugemu nasproti stojita poslovna človeka, a »when Greek meets Greek then comes the tug of war«.\footnote{Kadar Grk sreča Grka, se vname hud boj.}\ %TODO mora biti simbol, ne št. opombe.backslash na koncu opombe, ker ni bilo predledka
Sprememba stanja stane čas in delovno silo, vendar ne, da bi se vrednost ustvarjala, ampak da bi se vrednost prenesla iz ene oblike v drugo.
Obojestranski poskus prilastiti si ob tej priložnosti dodatno količino vrednosti, tega prav nič ne spremeni.
To delo, povečano z obojestranskimi zlobnimi nameni, ne ustvarja vrednosti, kakor delo, ki ga povzroča sodni postopek, ne povečuje vrednosti spornega predmeta.
S tem delom — ki je nujni člen kapitalističnega produkcijskega procesa v njegovi celovitosti, v kateri vključuje tudi cirkulacijo oziroma se vključuje vanjo — je nekako tako kakor z delom pri sežiganju tvarine, ki se uporabi za proizvodnjo toplote.
To delo pri sežiganju ne ustvarja nobene toplote, čeprav je nujen člen procesa sežiganja. 
Če hočemo porabiti na primer premog kot kurivo, ga moramo spojiti s kisikom in spraviti iz čvrstega v plinasto stanje (kajti v ogljikovem dvokisu, ki je rezultat izgorevanja, je premog v plinastem stanju), povzročiti torej fizikalno spremembo oblike obstoja ali stanja.
Razdružitev ogljikovih molekul, ki so združene v čvrsto celoto, in razcepitev ogljikove molekule v njene posamezne atome se morata opraviti pred ponovno spojitvijo.
Za to pa je potrebna poraba določene sile, ki se torej ne spremeni v toploto, ampak jo zmanjša.
Če torej posestniki blaga niso kapitalisti, ampak samostojni neposredni producenti, je za nakup in prodajo porabljeni čas odbitek od njihovega delovnega časa.
Zato so vedno skušali (tako v starem kot v srednjem veku) odložiti takšne opravke na praznike.

Dimenzije, ki jih zavzameta nakup in prodaja blaga v rokah kapitalistov, seveda tega dela, ki ne ustvarja nobene vrednosti, ampak omogoča le spremembo oblike vrednosti, ne morejo spremeniti v delo, ki bi ustvarjalo vrednost.
Prav tako se čudež te transsubstanciacije ne more dogoditi s
transpozicijo, se pravi tako, da napravijo industrijski kapitalisti\KPEstran\ tisto »delo sežiganja« za izključno opravilo tretjih oseb, katere plačujejo, namesto da bi ga opravljali sami. Te tretje osebe jim seveda ne dajejo svoje delovne sile na voljo iz ljubezni do njihovih lepih oči.
Pobiralcu zakupnine zemljiškega lastnika ali bančnemu slugi je prav vseeno, če njuno delo niti za groš ne povečuje ne velikosti vrednosti zakupnine ne zlatnikov, ki jih v vrečah prenaša v drugo banko.]\footnote
{Besedilo v oklepajih je iz neke pripombe na koncu rokopisa VIII.}

Kapitalistu, ki dá drugim, da delajo zanj, postaneta nakup in prodaja poglavitna funkcija.
Ker si prisvaja produkt mnogih v večjem družbenem obsegu, ga mora v takem obsegu tudi prodajati in kasneje znovič spreminjati iz denarja nazaj v produkcijske elemente.
Prej ko slej nakupni in prodajni čas ne ustvarjata vrednosti.
Iluzijo o tem ustvarjanju vrednosti povzroči delovanje trgovskega kapitala.
Ne da bi se v to spuščali pobliže, pa je jasno tole: če funkcijo, ki je sama po sebi neproduktivna, vendar nujen člen reprodukcije, delitev dela spremeni iz postranskega opravila mnogih v izključno opravilo maloštevilnih, v njihov posebni posel, se značaj funkcije same ne spremeni.
\emph{En} trgovec (tu ga gledamo zgolj kot agenta spreminjanja oblik blaga, zgolj kot kupca in prodajalca) lahko skrajša s svojimi operacijami nakupni in prodajni čas \emph{številnih} producentov.
Treba ga je torej obravnavati kot stroj, ki zmanjšuje nekoristni napor ali pomaga sproščati delovni čas.\footnote{»Čeprav so stroški trgovanja potrebni, jih je vendar treba šteti za bremenilne stroške.« (Quesnay, »Analyse du Tableau Economique«, »Physiocrates«, izd. Daire, I.\ del, Paris 1846, str.\ 71.) — Po Quesnayu je »profit«, ki ga prinaša konkurenca med trgovci, to namreč, da jih prisili »znižati njihov zaslužek ali dobiček, ...\ natančno vzeto, za prodajalca-producenta in za kupca-potrošnika samo \emph{zmanjšanje izgube}.
Zmanjšanje izgube zaradi trgovinskih stroškov pa seveda ni nikak \emph{dejanski produkt} ali povečanje bogastva, ki bi ga dosegla trgovina, najsi jo gledamo samo zase kratko malo kot menjavo, neodvisno od transportnih stroškov, ali pa skupno s temi stroški.«
(Str.\ 145, 146.) »Trgovinske stroške vedno plačajo prodajalci produktov, ki bi dobili celo ceno, katero plačajo zanje kupci, če ne bi bilo nobenih posredniških stroškov.« (Str.\ 165.)
»Lastniki in prodajalci plačujejo 'mezdo', trgovci pa jo prejemajo.«
(Quesnay, »Problèmes Économigues«, »Physiocrates«, izd.\ Daire, I.\ del, Paris 1846, str.\ 164.)} %opomba je čez 2 strani.
Zaradi poenostavitve (ker bomo trgovca kot kapitalista in trgovski kapital proučevali šele kasneje) vzemimo, da je\KPEstran\ ta agent za kupovanje in prodajanje človek, ki prodaja svoje delo.
Svojo delovno silo in svoj delovni čas troši v operacijah \(\KPEB\KPEcrta\KPED\) in \(\KPED\KPEcrta\KPEB\).
Od tega živi, tako kot kdo drug npr. od predenja ali od vrtanja makaronov.
Opravlja potrebno funkcijo, ker obsega reprodukcijski proces sam neproduktivne funkcije.
Dela prav tako kakor kdorkoli drug, vsebina njegovega dela pa ne ustvarja niti vrednosti niti produkta.
On sam spada v faux frais\footnote{Neproduktivni, vendar potrebni stroški.} %TODOsimboljna ne štv. opomba 
produkcije.
Njegova koristnost ni v tem, da bi spreminjal neproduktivno funkcijo v produktivno ali neproduktivno delo v produktivno.
Bil bi čudež, če bi bilo mogoče doseči takšno spremembo s takšnim prenosom funkcije.
Nasprotno, koristen je zato, ker povzroči, da se troši za to neproduktivno funkcijo manjši del družbene delovne sile in družbenega delovnega časa.
Še več!
Recimo, da je navaden mezdni delavec, lahko tudi bolje plačan.
Njegova plača je lahko takšna ali drugačna, kot mezdni delavec dela del svojega časa zastonj.
Dnevno dobi, denimo, vrednostni produkt osmih delovnih ur, dela pa deset.
Ti dve uri presežnega dela, ki ga opravi, prav tako ne producirata vrednosti, kot je ne producira njegovih osem ur potrebnega dela, čeprav prenaša s tem svojim delom nase del družbenega produkta.
Prvič: gledano z družbenega vidika, se vendar ena delovna sila uporablja deset ur za zgolj cirkulacijsko funkcijo.
Za kaj drugega, za produktivno delo, je ni mogoče uporabiti. Drugič: družba teh dveh ur presežnega dela ne plača, čeprav jih individuum, ki ga opravlja, potroši.
S tem si družba ne prisvoji nobenega presežnega produkta ali vrednosti.
Cirkulacijski stroški, ki jih predstavlja, pa se zmanjšajo za petino, od desetih ur na osem.
Družba ne plača za petino tega\KPEstran\ aktivnega cirkulacijskega časa, katerega agent je, nobene protivrednosti. Če pa uporablja tega agenta kapitalist, se, ker dveh ur ne plača, zmanjšajo cirkulacijski strošk \emph{njegovega} kapitala, ki pomenijo odbitek od njegovih prejemkov.
Zanj je to pozitivni dobiček, ker se zoži negativna meja večanja vrednosti njegovega kapitala.
Dokler porabljajo samostojni mali blagovni producenti del svojega lastnega časa za nakup in prodajo, je to samo čas, ki ga tro šijo v premorih med svojim produktivnim delovanjem, ali pa skrčenje njihovega produkcijskega časa.

Čas, ki se uporablja za ta namen, pomeni v vsakem primeru cirkulacijske stroške, ki ničesar ne dodajo prodanim ali kupljenim vrednostim.
To so stroški, ki so potrebni, da jih spremene iz blagovne v denarno obliko. 
Kolikor nastopa kapitalistični blagovni producent kot agent cirkulacije, se razlikuje od neposrednega blagovnega producenta samo po tem, da kupuje in prodaja v večjem obsegu in da zato v večjem obsegu deluje kot agent cirkulacije.
Ko ga pa obseg njegovega posla prisili ali usposobi, da kupi (najame) kot mezdne delavce posebne agente cirkulacije, se pojav v svojem bistvu ne spremeni.
V neki meri se delovna sila in delovni čas morata trošiti v procesu cirkulacije (kolikor je zgolj sprememba oblike). 
Zdaj pa postane to dodatni izdatek kapitala; del variabilnega kapitala se mora založiti v nakup teh delovnih sil, ki delujejo le v cirkulaciji.
Ta založitev kapitala ne ustvarja niti produkta niti vrednosti.
Pro tanto zmanjšuje obseg, v katerem deluje založeni kapital produktivno.
To je isto, kakor če bi se del produkta spremenil v stroj, ki bi kupoval in prodajal preostali del produkta.
Ta stroj povzroči odbitek od produkta.
Čeprav lahko zmanjša delovno silo itd., ki se troši v cirkulaciji, ne sodeluje v produkcijskem procesu.
Tvori samo del cirkulacijskih stroškov.

\subsection{Knjigovodstvo}
Razen pri kupovanju in prodajanju v pravem pomenu se porablja delovni čas v knjigovodstvu; v njem se porablja razen\KPEstran\ tega še popredmeteno delo, kakor peresa, črnilo, papir, pisalna miza, stroški pisarne.
V tej dejavnosti se porabljajo torej na eni strani delovna sila, na drugi pa delovna sredstva.
To je prav tako kot pri nakupnem in prodajnem času.

Kot enotnost v svojih krožnih tokih, kot krožeča vrednost, bodisi na produkcijskem področju ali pa v obeh fazah cirkulacijskega področja, obstaja kapital samo idealno v podobi obračunskega denarja, predvsem v glavi blagovnega producenta oziroma kapitalističnega blagovnega producenta.
To gibanje zajema in nadzira knjigovodstvo, ki obsega tudi določanje cen ali izračun blagovnih cen (kalkulacijo cen).
Potek produkcije in zlasti večanja vrednosti — pri čemer nastopajo razne vrste blaga le kot nosilci vrednosti, kot imena stvari, katerih idealni obstoj vrednosti je izražen v obračunskem denarju — pridobi tako simbolično podobo v predstavi.
Dokler vodi posamezni blagovni producent knjigovodstvo bodisi samo v svoji glavi (kot na primer kmet; šele kapitalistična produkcija ustvari zakupnika, ki vodi knjige) ali pa dela zapiske o svojih izdatkih, prejemkih, plačilnih rokih itd.\ kot postransko opravilo, ne v svojem produkcijskem času, toliko časa je očitno, da pomenijo ta njegova funkcija in delovna sredstva, ki jih pri tem uporablja, kakor papir itd., dodatno porabo delovnega časa in delovnih sredstev.
Ta čas in sredstva so potrebna, odvzame pa jih tako času, katerega lahko porabi produktivno, kakor tudi delovnim sredstvom, ki delujejo v dejanskem produkcijskem procesu, sodelujejo pri tvorjenju produkta in vrednosti.\footnote
{V srednjem veku so imeli v poljedelstvu knjigovodstvo samo v samostanih.
Videli pa smo (I.\ knjiga, str.\ 407), da nastopa že v prastari indijski skupnosti v poljedelstvu knjigovodja.
Tu se je osamosvojilo knjigovodstvo v izključno opravilo občinskega uradnika.
S to delitvijo dela so prihranili čas, trud in izdatke; produkcija in knjigovodstvo o produkciji pa sta prav tako različni stvari kakor ladijski tovor in tovorni list.
S knjigovodjo je odvzet produkciji del občinske delovne sile.
Stroškov njegovega delovanja ne nadomešča njegovo lastno delo, ampak odbitek od občinskega produkta.
Kar velja za knjigovodjo indijske občine, velja mutatis mutandis za kapitalistovega knjigovodjo. (Iz rokopisa II.)}
Narava funkcije same\KPEstran\ se ne spremeni niti z obsegom, ki ga pridobi s tem, da se koncentrira v rokah kapitalističnega blagovnega producenta in nastopa namesto kot funkcija mnogih majhnih blagovnih producentov kot funkcija \emph{enega} kapitalista, kot funkcija produkcijskega procesa v velikem obsegu, niti s tem, da jo odtrga od produktivnih funkcij, ki jim je bila postransko opravilo, in da se osamosvoji kot funkcija posebnih agentov, ki jim je izključno opravilo.

Če kaka funkcija ni sposobna ustvarjati produktov in vrednosti sama po sebi, torej že pred svojo osamosvojitvijo, je tudi delitev dela, njena osamosvojitev, ne napravi za tako.
Če naloži kapitalist svoj kapital na novo, mora naložiti del kapitala v nakup knjigovodje itd.\ in v knjigovodske potrebščine.
Če njegov kapital že deluje, če je že v svojem stalnem reprodukcijskem procesu, mora stalno spreminjati s pretvorbo v denar del blagovnega produkta nazaj v knjigovodjo, pomočnika in podobno.
Ta del kapitala je odtegnjen produkcijskemu procesu in pripada cirkulacijskim stroškom, odtegljajem od skupnega donosa (vštevši tisto delovno silo, ki se uporablja izključno za to funkcijo).

Vendar pa je neka razlika med stroški, ki izvirajo iz knjigovodstva, oziroma med neproduktivnim trošenjem delovnega časa na eni strani in stroški samega nakupnega in prodajnega časa na drugi.
Slednji izvirajo izključno iz določene družbene oblike produkcijskega procesa, iz tega, da je to proces produkcije blaga.
Knjigovodstvo kot kontrola in miselno zajetje tega procesa postaja tem bolj potrebno, čim bolj poteka proces v družbenem obsegu in izgublja čisto individualni značaj; torej je bolj nujno v kapitalistični kot v razdrobljeni produkciji rokodelskega in kmečkega gospodarstva, bolj nujno v kolektivni produkciji kakor v kapitalistični.
Stroški knjigovodstva pa se krčijo, čim bolj se produkcija koncentrira in čim bolj se knjigovodstvo spreminja v družbeno knjigovodstvo.

Tu nam gre le za splošni značaj cirkulacijskih stroškov, ki jih povzroča čisto formalna metamorfoza.
Tu ni treba obravnavati vseh njihovih posebnih oblik.
Kako pa nas lahko te oblike, ki jih povzroča zgolj oblikovno spreminjanje\KPEstran\ vrednosti in izhajajo torej iz določene družbene oblike produkcijskega procesa, katere so pri individualnem blagovnem producentu samo kratkotrajni in komaj opazni momenti, ki spremljajo njegove produktivne funkcije ali se z njimi prepletajo — kako nas lahko te oblike zaradi množičnosti cirkulacijskih stroškov zavarajo, vidimo že pri prejemanju in izdajanju denarja, kakor hitro se kot izključna funkcija bank ipd.\ ali blagajnika v posameznih podjetjih osamosvoje in koncentrirajo v velikem obsegu.
Ne smemo pa pozabiti, da ti cirkulacijski stroški ne spremenijo svojega značaja, če se spremeni njihova podoba. 
\subsection{Denar}
Ne glede na to, ali se kak produkt producira kot blago ali ne, je vedno tvarna podoba bogastva, uporabna vrednost, ki je določena, da preide v osebno ali produktivno konsumpcijo.
Kot blago ima svojo vrednost idealno v ceni, ki ničesar ne spremeni na njegovi dejanski uporabni podobi.
To, da nekatere vrste blaga, kakor zlato in srebro, delujejo kot denar in se kot take zadržujejo izključno v cirkulacijskem procesu (tudi kot zaklad, rezerva itd.\ ostanejo, čeprav latentno, na področju cirkulacije), je zgolj nasledek določene družbene oblike produkcijskega procesa, ki je produkcijski proces blaga.
Ker postane na podlagi kapitalistične produkcije blago splošna oblika produkta in se producira največja množina produkta kot blago in mora zato privzeti denarno obliko, ker torej množina blaga, kot blago nastopajoči del družbenega bogastva, nenehno raste — se povečuje tudi količina zlata in srebra, ki služi za cirkulacijsko sredstvo, plačilno sredstvo, zaklad itd.
To kot denar delujoče blago ne gre niti v osebno niti v produktivno konsumpcijo.
Je družbeno delo, strnjeno v obliki, v kateri služi zgolj kot cirkulacijski stroj.
Razen tega, da je prikovan del družbenega bogastva v to neproduktivno obliko, zahteva obraba denarja, da ga stalno nadomeščamo ali da spreminjamo več družbenega dela — v obliki produkta — v več zlata in srebra.
Ti stroški za nadomeščanje so\KPEstran\ v kapitalistično razvitih deželah pomembni, ker je sploh obsežen del bogastva prikovan v obliko denarja.
Denar in srebro pomenita kot denarno blago za družbo cirkulacijske stroške, ki izvirajo zgolj iz družbene oblike produkcije.
To so faux frais blagovne produkcije nasploh, ki naraščajo z razvojem blagovne produkcije in zlasti kapitalistične.
To je del družbenega bogastva, ki se mora žrtvovati cirkulacijskemu procesu.\footnote
{Denar, ki cirkulira v kaki deželi, je določen del kapitala te dežele, ki je docela odtegnjen produktivnim smotrom z namenom, da se olajša ali poveča produktivnost ostalega kapitala.
Zato je določen obseg bogastva prav tako potreben za uporabljanje zlata kot obtočnega sredstva, kakor je potreben za izdelavo stroja, ki naj olajša katero koli drugo produkcijo.« (»Economist«, 8.\ maja 1847, V.\ zv., str.\ 520.)}

\section{Stroški za hrambo}
Cirkulacijski stroški, ki jih povzroča golo spreminjanje oblike vrednosti, cirkulacija, abstraktno gledana, ne gredo v vrednost blaga.
Kolikor imamo v mislih kapitalista, so zanje porabljeni deli kapitala čisti odtegljaji od produktivno porabljenega kapitala.
Drugačne narave pa so cirkulacijski stroški, ki jih obravnavamo sedaj.
Ti lahko izhajajo iz produkcijskih procesov, ki se v cirkulaciji samo nadaljujejo, katerim cirkulacijska oblika samo zakriva njihov produktivni značaj.
Po drugi strani so z družbenega vidika lahko goli stroški, neproduktivni izdatki bodisi živega, bodisi opredmetenega dela, pa vendar ravno s tem delujejo za posameznega kapitalista kot ustvarjalci vrednosti, so dodatek k prodajni ceni njegovega blaga.
To izhaja že iz tega, ker so ti stroški na različnih produkcijskih področjih in tu pa tam tudi za različne posamezne kapitale znotraj istega produkcijskega področja različni.
S tem da se dodajo ceni blaga, se porazdelijo tako, kakor obremenijo posamezne kapitaliste.
Vsako delo, ki dodaja vrednost, pa lahko dodaja tudi presežno vrednost in bo na kapitalistični podlagi vedno dodajalo presežno vrednost, ker je vrednost,\KPEstran\ ki jo ustvarja, odvisna od količine dela, presežna vrednost, ki jo ustvarja, pa od obsega, v katerem ga kapilalist plača.
Stroški, ki podražujejo blago, ne da bi mu dodajali uporabno vrednost, ki spadajo torej za družbo med faux frais produkcije, so zato lahko za posameznega kapitalista vir obogatitve.
Po drugi strani pa se ne odpravi njihov neproduktivni značaj, če dodatek, ki ga pribijajo k ceni blaga, te cirkulacijske stroške enakomerno porazdeli.
Zavarovalne družbe, na primer, porazdelijo izgube posameznih kapitalistov na kapitalistični razred.
To pa vendar ne prepreči, da ne bi tako izravnane izgube bile z vidika celotnega družbenega kapitala vendarle izgube.

\subsection{Ustvarjanje zalog nasploh}
V času, ko je blagovni kapital, ko se zadržuje na trgu, torej v presledku med produkcijskim procesom, iz katerega prihaja, in konsumnim procesom, v katerega stopa, je produkt v blagovni zalogi.
Kot blago na trgu, in zato v podobi zaloge, nastopa blagovni kapital v vsakem krožnem toku dvakrat.
Enkrat kot blagovni produkt tistega delujočega kapitala, katerega krožni tok opazujemo, drugič kot blagovni produkt kakega drugega kapitala, ki mora biti na trgu, da ga je moč kupiti in spremeniti v produktivni kapital.
Možno je seveda, da se producira ta blagovni kapital le po naročilu.
Tedaj nastane prekinitev, dokler blago ni producirano.
Tok produkcijskega in reprodukcijskega procesa pa zahteva, da se stalno zadržuje na trgu določena množina blaga (produkcijskih sredstev), da torej tvori zalogo.
Prav tako obsega produktivni kapital nakup delovne sile, in njegova denarna oblika je tu samo oblika vrednosti življenjskih potrebščin, ki jih mora delavec večidel že dobiti na trgu.
V nadaljevanju tega paragrafa si bomo to pobliže ogledali.
Za zdaj velja, da je to že doseženo.
Če se postavimo na stališče krožeče vrednosti kapitala, ki se je spremenila v blagovni produkt in se mora sedaj prodati oziroma spremeniti nazaj v denar, ki torej deluje zdaj kot blagovni kapital na trgu, je njegov položaj, ko tvori zalogo,\KPEstran\ nesmotrno, neprostovoljno zadrževanje na trgu.
Čim hitreje se proda, tem hitreje poteka reprodukcijski proces.
Zadrževanje v spreminjanju oblike \(\KPEB'\KPEcrta\KPED'\) zavira stvarno presnavljanje tvarine, ki se mora izvesti v krožnem toku kapitala, kakor tudi njegovo nadaljnje delovanje kot produktivnega kapitala.
Po drugi strani je nenehna prisotnost blaga na trgu, blagovna zaloga, z vidika \(\KPED\KPEcrta\KPEB\) pogoj toka reprodukcijskega procesa kakor tudi naložbe novega ali dodatnega kapitala.

Zadrževanje blagovnega kapitala kot blagovne zaloge na trgu zahteva stavbe, skladišča, posode za blago, shrambe, torej izdatek konstantnega kapitala; prav tako plačilo delovne sile za spravljanje blaga v njegove shrambe. 
Razen tega se blago kvari in je izpostavljeno škodljivim elementarnim vplivom.
Da se pred njimi zavaruje, je treba založiti dodatni kapital, deloma v delovnih sredstvih, v tvarni obliki, deloma v delovni sili.\footnote
{Corbet ceni leta 1841 stroške za uskladiščenje pšenice za sezono devetih mesecev na $\frac{1}{2}\%$ izgube pri količini, 3\% za obresti na ceno pšenice, 2\% za najemnino skladišča, 1\% za natovarjanje in prevoz, $\frac{1}{2}\%$) za delo pri odpremi, skupno 7\% ali 3 šilinge 5 penijev za quarter, če znaša cena pšenice 50 šilingov. 
(Th. Corbet, »An Inguiry into the Causes and Modes of the Wealth of Individuals etc.«, London 1841.) Po izpovedih liverpoolskih trgovcev pred železniško komisijo so znašali (čisti) stroški skladiščenja pšenice leta 1865 mesečno 2 penija za quarter ali 9 do 10 penijev za tono.
(»Royal Commission on Railways. Minutes of Evidence«,
London 1867, str.\ 19, št.\ 331.)}

Bivanje kapitala v njegovi obliki blagovnega kapitala in zato blagovne zaloge povzroča torej stroške, ki štejejo, ker ne spadajo na produkcijsko področje, k stroškom cirkulacije.
Ti cirkulacijski stroški se razlikujejo od onih, ki smo jih navedli pod 1, po tem, da v določenem obsegu preidejo v vrednost blaga, da torej podražujejo blago. %"pod I" -> "pod 1; podobno spodaj"
V vsakem primeru sta kapital in delovna sila, ki sta potrebna za vzdrževanje in hrambo blagovnih zalog, odtegnjena neposrednemu produkcijskemu procesu.
Na drugi strani je potrebno kapitale, ki se pri tem uporabljajo, vštevši dlovno silo kot sestavni del kapitala, nadomestiti iz družbenega\KPEstran\ produkta.
Njihov izdatek učinkuje zato kot zmanjšanje produktivne sile dela, v tem smislu, da je potrebna večja količina dela in kapitala za dosego določenega koristnega učinka.
To so \emph{neproduktivni stroški}.

Kolikor izvirajo cirkulacijski stroški, ki jih zahteva tvorjenje blagovnih zalog, le iz časa, ki je potreben za spremembo obstoječih vrednosti iz blagovne oblike v denarno, torej le iz določene družbene oblike produkcijskega procesa (samo iz tega, da se produkt producira kot blago in da se mora zaradi tega tudi spremeniti v blago) — so povsem enakega značaja kakor cirkulacijski stroški, ki smo jih našteli pod 1. 
Po drugi strani se vrednost blaga ohranja oziroma povečuje samo zato, ker se postavlja uporabna vrednost, produkt sam, v takšne tvarne pogoje, ki povzročajo trošenje kapitala, in ker se podvrže operacijam, ki povzročajo učinkovanje dodatnega dela na uporabne vrednosti.
Preračunavanje blagovnih vrednosti, knjigovodstvo o tem procesu, prodajno in nakupno trgovanje pa v nasprotju s tem ne vplivajo na uporabno vrednost, v kateri obstaja blagovna vrednost.
Vplivajo le na njeno obliko.
Čeprav povzroča v obravnavanem primeru te neproduktivne stroške tvorjenja zalog (ki je tu neprostovoljno) samo zavlačevanje oblikovne spremembe in nujnost te spremembe, se vendar razlikujejo od neproduktivnih stroškov pod l po tem, da njihov smoter ni sprememba oblike vrednosti, ampak ohranitev vrednosti, ki obstaja v blagu kot produkt, kot uporabna vrednost, in ki se torej lahko ohrani le z ohranitvijo produkta, njegove uporabne vrednosti.
Uporabna vrednost se pri tem niti ne poveča niti ne pomnoži;
nasprotno, zmanjšuje se.
Toda njeno zmanjšanje se omejuje in uporabna vrednost se ohranja. Tudi založena vrednost, ki obstaja v blagu, se ne poveča.
Dodaja pa se ji novo delo, opredmeteno in živo.

Sedaj je treba raziskati še, v kakšni meri izvirajo ti neproduktivni stroški iz posebnega značaja blagovne produkcije nasploh in blagovne produkcije v njeni splošni, absolutni obliki, se pravi iz kapitalistične blagovne produkcije; po drugi strani, koliko so skupni vsaki družbeni produkciji\KPEstran\ in privzemajo tu, v okviru kapitalistične produkcije, samo posebno podobo, posebno pojavno obliko.

A.\ Smith je avtor čudovite ideje, da je ustvarjanje zalog poseben pojav kapitalistične produkcije.\footnote
{»Wealth of Nations«, II.\ knjiga, uvod.}
Novejši ekonomisti, na primer Lalor, pa trdijo nasprotno, da se z razvojem kapitalistične produkcije zaloge zmanjšujejo. Sismondi vidi v tem celo eno temnih strani te produkcije.

Zaloga obstaja v resnici v treh oblikah: v obliki produktivnega kapitala, v obliki osebnega konsumpcijskega sklada in v obliki blagovne zaloge ali blagovnega kapitala.
Zaloga v eni obliki se relativno zmanjšuje, če v drugi obliki narašča, čeprav lahko po svoji absolutni velikosti narašča v vseh treh oblikah istočasno.

Že od začetka je jasno, da tam, kjer je usmerjena produkcija neposredno na zadovoljitev potreb producentov in se producira le v manjšem delu za menjavo ali prodajo, kjer torej družbeni produkt sploh ne privzame oblike blaga ali pa jo privzame le v manjšem delu, tvori zaloga v obliki blaga ali blagovna zaloga samo neznaten in nepomemben del bogastva.
Konsumpcijski sklad, posebno pravih življenjskih potrebščin, pa je tam razmeroma velik. 
Pogledati je treba samo staroveško kmečko gospodarstvo. 
Pretežni del produkta se spreminja neposredno, ne da bi ustvaril blagovno zalogo — ravno zato, ker ostane v rokah svojega posestnika — v zalogo produkcijskih sredstev ali življenjskih potrebščin.
Ne navzame pa oblike blagovne zaloge.
Zaradi tega po A.\ Smithu v gospodarstvih, ki temeljijo na takšnem produkcijskem načinu, ni nobenih zalog. A.\ Smith zamenjuje obliko zaloge z zalogo samo in misli, da je živela družba doslej iz rok v usta in da se je zanašala na to, kaj bo prinesel prihodnji dan.\footnote
{Namesto da bi nastale zaloge šele s spremembo produkta v blago in zaloge za potrošnjo v blagovno zalogo, kakor domneva A.\ Smith, povzroča nasprotno ta sprememba oblike med prehodom iz produkcije za lastne potrebe v blagovno produkcijo najhujše krize v gospodarstvu producentov.
V Indiji se je na primer do najnovejšega časa ohranila »navada, da se žito, za katero se dobi v letih izobilja le malo, v velikih množinah uskladišči«.
(»Return. Bengal and Orissa Famine. H. of C. 1867«, I, str.\ 230, št. 75.)
Zaradi naglo povečanega povpraševanja po bombažu, juti itd., ki ga je povzročila ameriška državljanska vojna, so v mnogih delih Indije producenti zelo omejili pridelovanje riža, zvišali njegovo ceno in prodali stare zaloge riža.
Vrh tega se je v letih 1864—66 kakor doslej še nikoli povečal izvoz riža v Avstralijo, na Madagaskar itd.
Od tod akutni značaj lakote leta 1866, ki je samo v okrožju Orissa pobrala milijon ljudi.
(Prav tam, str.\ 174, 175, 215, 214, in III: »Papers relating to the Famine in Behar«, str.\ 52, 53, kjer se poudarja med vzroki lakote the drain of old stock [izčrpanje stare zaloge].)
(Iz rokopisa II.)}
To je otročje nerazumevanje.

Zaloga\KPEstran\ v obliki produktivnega kapitala ima obliko produkcijskih sredstev, ki so že v produkcijskem procesu ali pa vsaj v producentovih rokah, torej latentno že v produkcijskem procesu.
Prej smo videli, da z razvojem produktivnosti dela, torej tudi z razvojem kapitalističnega produkcijskega načina — ki razvija družbeno produktivno silo dela bolj ko vsi drugi produkcijski načini — nenehno raste množina produkcijskih sredstev, ki so v obliki delovnih sredstev enkrat za vselej pripojena procesu in v njem vedno znova delujejo v daljših ali krajših obdobjih (stavbe, stroji itd.), in da je njeno naraščanje tako predpostavka kakor tudi posledica razvoja družbene produktivne sile dela.
Ne samo absolutna, ampak tudi relativna rast bogastva v tej obliki je značilna posebno za kapitalistični produkcijski način (prim.\ 1.\ knjigo, 25.\ poglavje, 2).
Tvarne oblike, v katerih obstaja konstantni kapital, produkcijska sredstva, pa ne sestoje le iz takšnih delovnih sredstev, ampak tudi iz delovnih tvarin na najrazličnejših stopnjah predelave in iz pomožnih tvarin.
Z obsegom produkcije in z večanjem produktivne sile dela s kooperacijo, z delitvijo dela, stroji itd.\ raste množina surovin, pomožnih tvarin itd., ki prihajajo vsak dan v reprodukcijski proces.
Ti elementi morajo biti že pripravljeni na mestu produkcije. Obseg te zaloge, ki ima obliko produktivnega kapitala, narašča torej absolutno.
Da bi proces lahko potekal — pri tem je čisto brez pomena, ali je možno obnavljati to zalogo dnevno ali le v določenih obdobjih — mora biti na mestu produkcije vedno\KPEstran\ nakopičenih več že pripravljenih surovin itd., kakor jih porabi npr.\ na dan ali na teden.
Nepretrganost procesa zahteva, da obstoj njegovih pogojev ni odvisen niti od prekinitev, ki so možne pri dnevnih nakupih, niti od tega, da se pri dnevni ali tedenski prodaji blagovni produkt le neenakomerno spreminja nazaj v svoje produkcijske elemente.
Vendar je očitno produktivni kapital lahko latenten ali tvori zalogo v zelo različnih obsegih.
Velika razlika je, na primer, ali mora imeti predilec zalogo bombaža ali premoga za tri mesece ali za en mesec.
Vidimo torej, da se ta zaloga relativno lahko krči, čeprav absolutno narašča.

To je odvisno od različnih pogojev, katerih bistvo so večja hitrost, rednost in zanesljivost, s katerimi se potrebna množina surovin lahko vedno dobavi tako, da nikoli ne pride do prekinitve.
Čim slabše so izpolnjeni ti pogoji, tem manjša je torej zanesljivost, rednost in hitrost dobave, tem večji mora biti latentni del produktivnega kapitala, tj.\ zaloga surovin itd., ki čaka v rokah producenta na svojo predelavo.
Ti pogoji so v obratnem sorazmerju z razvojno stopnjo kapitalistične produkcije in s tem produktivne sile človeškega dela.
Torej tudi zaloga v tej obliki.

Vendar je to, kar se kaže kot zmanjšanje zaloge (na primer pri Lalorju), deloma le zmanjšanje zaloge v obliki blagovnega kapitala ali prave blagovne zaloge; torej le sprememba oblike iste zaloge.
Če je na primer množina premoga, ki se dnevno producira v lastni deželi, torej količina in energija produkcije premoga, velika, predilec ne potrebuje velikega skladišča, da zagotovi nepretrganost svoje produkcije.
Zaradi stalnega zanesljivega obnavljanja dobave premoga postane odveč.
Drugič: hitrost, s katero lahko preide produkt kakega procesa kot produkcijsko sredstvo v drug proces, je odvisna od razvoja prevoznih in komunikacijskih sredstev.
Cenenost transporta igra pri tem veliko vlogo.
Nepretrgan transport, na primer premoga od rudnika do predilnice, bi bil dražji kakor oskrbovanje z večjo množino premoga za daljši čas ob razmeroma cenejšem transportu.
Obe ti dve doslej obravnavani okoliščini izhajata iz samega produkcijskega procesa.
Tretjič, pomemben\KPEstran\ je razvoj kreditnega sistema.
Čim manj je odvisen predilec glede obnovitve svojih zalog bombaža, premoga itd. od neposredne prodaje svoje preje — in čim bolj je razvit kreditni sistem, tem manjša je ta neposredna odvisnost — tem manjša je lahko relativna velikost teh zalog, da zagotovi nepretrgano produkcijo preje v določenem obsegu, neodvisno od naključij pri prodaji preje.
Četrtič, mnoge surovine, polizdelki ipd.\ pa potrebujejo daljša obdobja za svojo produkcijo; zlasti velja to za vse surovine, ki jih dobavlja poljedelstvo.
Da se produkcijski proces ne pretrga, mora biti torej v zalogi določena količina surovin za vse obdobje, v katerem novi produkt ne more nadomestiti starega.
Če se ta zaloga v rokah industrijskega kapitalista zmanjša, pomeni to le, da narašča v obliki blagovne zaloge v rokah trgovca.
Razvoj prevoznih sredstev omogoča, na primer, da se prepelje bombaž, ki leži v uvoznem pristanišču, hitro iz Liverpoola v Manchester, tako da lahko tovarnar v razmeroma majhnih delih po potrebi obnavlja svojo zalogo bombaža.
Vendar pa leži potem isti bombaž v toliko večjih količinah kot blagovna zaloga v rokah trgovcev v Liverpoolu.
Gre torej samo za spremembo oblike zaloge; Lalor in drugi so to prezrli.
Z vidika družbenega kapitala leži tu v obliki zaloge slej ko prej ista množina produktov.
Z razvojem prevoznih sredstev se za posamezno deželo zmanjšuje obseg potrebne množine, ki mora biti pripravljena na primer za eno leto.
Če pluje veliko parnikov in jadrnic med Ameriko in Anglijo, je več možnosti za obnovitev bombažnih zalog v Angliji in se zato zmanjšuje količina bombaža, ki mora biti v Amgliji povprečno v zalogi.
Enako učinkuje razvoj svetovnega trga, ker se tako pomnože dobavni viri istega blaga.
Blago se dovaža postopoma iz različnih dežel in ob različnih rokih.

\subsection{Prava blagovna zaloga}
Videli smo že, da postane na podlagi kapitalistične produkcije blago splošna oblika produkta, in to tem bolj, čim bolj se ta produkcija razvije po obsegu in globini.
Celo pri\KPEstran\ enakem obsegu produkcije ima torej, bodisi v primerjavi s prejšnjimi produkcijskimi načini, bodisi v primerjavi s kapitalističnim produkcijskim načinom na manj razviti stopnji, neprimerno večji del produkta obliko blaga.
Vsako blago — torej tudi vsak blagovni kapital, ki je samo blago, vendar blago kot eksistenčna oblika vrednosti kapitala — pa je, razen če iz svoje produkcijske sfere ne preide neposredno v produktivno ali osebno potrošnjo, dokler se torej zadržuje na trgu, sestavina blagovne zaloge.
Sama po sebi — ob nespremenjenem obsegu produkcije — zato blagovna zaloga (se pravi ta osamosvojitev in utrditev blagovne oblike produkta) narašča hkrati s kapitalistično produkcijo.
Videli smo že, da je to le sprememba v obliki zaloge; na eni strani narašča zaloga v blagovni obliki zato, ker se na drugi strani v obliki neposredne produkcijske in konsumpcijske zaloge zmanjšuje.
To je samo spremenjena družbena oblika zaloge.
Če se povečuje hkrati ne le relativna velikost blagovne zaloge v primeri z družbenim celotnim produktom, ampak tudi njegova absolutna velikost, je to posledica dejstva, da se s kapitalistično produkcijo povečuje množina celotnega produkta.

Z razvojem kapitalistične produkcije se določa obseg produkcije v čedalje manjši meri z neposrednim povpraševanjem, v čedalje večji meri pa z obsegom kapitala, ki je na voljo posameznemu kapitalistu, z nagonom po večanju vrednosti njegovega kapitala in nujnostjo, da ostane njegov produkcijski proces nepretrgan in da se razširi.
S tem nujno raste v vsaki posamezni produkcijski panogi množina produktov, ki je kot blago na trgu ali išče kupca.
Narašča množina kapitala, ki je krajši ali daljši čas prikovana na obliko blagovnega kapitala.
Zaradi tega narašča blagovna zaloga.

Končno, največji del družbe se spreminja v mezdne delavce, ljudi, ki žive iz rok v usta, prejemajo svojo mezdo tedensko in jo dnevno trošijo, za katere morajo biti torej njihove življenjske potrebščine pripravljene kot zaloga.
Najsi se posamezne sestavine te zaloge še tako hitro zamenjujejo, nekaj\KPEstran\ jih mora biti vendar vedno pri roki, če naj jih nenehno jemljejo iz zaloge.

Vse te značilnosti izhajajo iz oblike produkcije in iz oblikovne spremembe, ki jo vključuje in skozi katero mora produkt v procesu cirkulacije.

Kakršna koli že je družbena oblika zaloge produktov, vedno zahteva njeno hranjenje stroške: stavbe, posodo itd., v katerih se hranijo, prav tako pa produkcijska sredstva in delo, ki jih je treba v večji ali manjši meri trošiti v obrambo pred kvarnimi vplivi, kakor pač zahteva narava produktov.
Čim bolj so zaloge družbeno koncentrirane, tem manjši so sorazmerno ti stroški.
Ti izdatki pomenijo vedno del družbenega dela, bodisi v opredmeteni, bodisi v živi obliki — v kapitalističnem gospodarstvu torej izdatke kapitala — ki ne preide v ustvarjanje produkta, torej odbitke od produkta.
To so nujni neproduktivni stroški družbenega bogastva.
So vzdrževalni stroški družbenega produkta, ne glede na to, ali izhaja kot sestavina blagovne zaloge izključno iz družbene oblike produkcije, torej iz oblike blaga in njene nujne spremembe oblike, ali pa štejemo blagovno zalogo samo za posebno obliko zaloge produktov, ki je skupna vsem družbam, čeprav ne v obliki \emph{blagovne} zaloge, tiste oblike zaloge produktov, ki pripada cirkulacijskemu procesu.

Vprašanje je zdaj, v kakšni meri prehajajo ti stroški v vrednost blaga.

Če je kapitalist svoj v produkcijska sredstva in v delovno silo založeni kapital spremenil v produkt, v dogotovljeno količino blaga, ki je določena za prodajo, in ostane ta neprodana v skladišču, se v tem času ne zaustavi samo proces večanja vrednosti njegovega kapitala.
Izdatki, ki jih zahteva vzdrževanje te zaloge za zgradbe, dodatna dela itd., so za kapitalista nedvomna izguba.
Končni kupec bi se mu smejal, če bi rekel: »Svojega blaga šest mesecev nisem mogel prodati.
Njegovo vzdrževanje v teh šestih mesecih mi ni le omrtvičilo toliko in toliko kapitala, ampak povzročilo razen tega x neproduktivnih stroškov.«
»Toliko slabše za vas,« pravi kupec.
»Tu poleg vas je drug\KPEstran\ prodajalec, čigar blago je bilo dogotovljeno šele predvčerajšnjim.
Vaše blago ni kurantno in najbrž ga je že bolj ali manj načel zob časa. 
Prodati ga morate torej ceneje kakor vaš tekmec.« — 
Vprašanje, ali je blagovni producent dejanski producent svojega blaga ali njegov kapitalistični producent, v resnici torej samo predstavnik njegovega dejanskega producenta, prav nič ne spremeni življenjskih pogojev blaga. Svojo robo mora spremeniti v denar. 
Neproduktivni stroški, ki mu jih povzroča okoliščina, da je vklenjena v blagovno obliko, spadajo med njegove osebne prigode, ki se kupca blaga nič ne tičejo.
Ta mu ne plača cirkulacijskega časa njegovega blaga.
Celo če kapitalist namerno zadržuje svoje blago, ga ne da na trg v časih dejanskega ali domnevnega prevrata vrednosti, je odvisno od tega, ali bo ta prevrat vrednosti res nastal, od pravilnosti ali nepravilnosti njegove špekulacije, ali bo lahko realiziral dodatne neproduktivne stroške.
Prevrat vrednosti pa ni posledica njegovih neproduktivnih stroškov.
Kolikor pomeni torej tvorjenje zaloge zastoj cirkulacije, ne dodajo stroški, ki jih povzroča, blagu nobene vrednosti.
Po drugi strani pa ne more biti nobene zaloge brez postanka na
cirkulacijskem področju, brez daljšega ali krajšega zadrževanja kapitala v njegovi blagovni obliki; torej ni nobene zaloge brez zastoja v cirkulaciji, kakor tudi noben denar ne more krožiti, če se ne ustvari denarna zaloga.
Brez blagovne zaloge torej ni blagovne cirkulacije.
Če se ne pokaže ta nujnost kapitalistu v \(\KPEB'\KPEcrta\KPED'\), se mu pokaže v \(\KPED\KPEcrta\KPEB\); ne za njegov blagovni kapital, ampak za blagovni kapital drugih kapitalistov, ki producirajo produkcijska sredstva zanj in življenjske potrebščine za njegove delavce.

Zdi se, da ne more stvari prav nič spremeniti, če je tvorjenje zaloge prostovoljno ali neprostovoljno, se pravi, če dela blagovni producent zalogo namerno ali pa ostaja njegovo blago v zalogi, ker se upirajo njegovi prodaji okoliščine cirkulacijskega procesa.
Vendar je za rešitev tega vprašanja koristno vedeti, kaj razlikuje prostovoljno tvorjenje zaloge od neprostovoljnega. 
Neprostovoljno tvorjenje zaloge izhaja ali je istovetno z zastojem cirkulacije, ki\KPEstran\ nastane brez vednosti blagovnega producenta in je navzkriž z njegovo voljo.
Kaj označuje prostovoljno tvorjenje zaloge?
Vsekakor se skuša prodajalec znebiti svojega blaga kakor le mogoče hitro. 
Svoj produkt ponuja vedno na prodaj kot blago.
Če bi ga odtegnil prodaji, bi bil samo možen (GRŠČINA), ne pa dejanski (GRŠČINA) element blagovne zaloge. %TODOδυνάμει in έναγεία
Blago kot tako je prej ko slej le nosilec svoje menjalne vrednosti, kot tako pa lahko deluje samo, če in ko zapusti svojo blagovno obliko in privzame denarno.

Zaloga blaga mora imeti določen obseg, da lahko v določenem razdobju zadošča za obseg povpraševanja.
Pri tem se računa, da se krog kupcev neprestano širi.
Zato, da lahko zadošča na primer en dan, se mora del blaga, ki je na trgu, trajno zadrževati v blagovni obliki, medtem ko drugi del teče, se spreminja v denar.
Tisti del, ki zastaja, medtem ko drugi teče, se sicer stalno zmanjšuje, kakor se zmanjšuje obseg zaloge nasploh, dokler ni končno vsa prodana.
Zastajanje blaga šteje torej tu kot nujni pogoj njegove prodaje.
Razen tega mora biti obseg večji kot povprečna prodaja ali obseg povprečnega povpraševanja.
Sicer ne bi bilo mogoče zadovoljiti nadpovprečnega povpraševanja.
Po drugi strani je treba zalogo stalno obnavljati, ker se stalno izčrpava.
Ta obnovitev lahko pride navsezadnje le iz produkcije, iz ponudbe blaga.
Ali pride iz uvoza ali ne, je čisto nepomembno.
Obnavljanje je odvisno od obdobij, ki jih potrebuje blago za svojo reprodukcijo.
Med tem časom mora zadoščati blagovna zaloga.
Dejstvo, da ne ostane v rokah prvotnega producenta, ampak teče skozi različne rezervoarje, od trgovca na debelo do prodajalca na drobno, spremeni samo pojav, ne pa stvari same. Dokler ne preide blago v produktivno ali osebno konsumpcijo, je z družbe nega vidika prej ko slej del kapitala v obliki blagovne zaloge.
Da ne bi bil odvisen neposredno od produkcije in da bi si zagotovil stalen krog strank, skuša imeti producent tolikšno zalogo blaga, ki ustreza povprečnemu povpraševanju po blagu. 
Skladno s produkcijskimi periodami se določajo nakupni termini, blago pa je daljši ali krajši čas v zalogi, dokler se ne more nadomestiti z novimi primerki %?? na strani 161 spodaj je "11 — Kapital II."
iste vrste.\KPEstran\ Samo takšno tvorjenje zalog zagotavlja trajnost in nepretrganost cirkulacijskega procesa, in torej tudi reprodukcijskega, ki vključuje cirkulacijskega.

Pomnimo: za producenta \KPEB\ je lahko \(\KPEB'\KPEcrta\KPED'\) opravljen, čeprav je \KPEB\ še vedno na trgu.
Če bi hotel zadržati producent svoje lastno blago sam v skladišču, dokler ga ne bi prodal končnemu kupcu, bi moral naložiti dvojni kapital, enega kot producent blaga, drugega kot trgovec. 
Za blago samo — ali ga gledamo kot posamezno blago ali pa kot sestavni del kapitala — je popolnoma nepomembno, ali obremenijo stroški tvorjenja zaloge njegovega producenta ali pa vrsto trgovcev od A do Ž.

Kolikor ni blagovna zaloga nič drugega kot blagovna oblika zaloge, ki bi obstajala na določeni stopnji družbene produkcije bodisi kot produktivna zaloga (latentni produkcijski sklad), bodisi kot konsumpcijski sklad (rezerva konsumpcijskih sredstev), če ne bi obstajala kot blagovna zaloga, so tudi stroški, ki jih zahteva ohranitev zaloge, torej stroški tvorjenja zaloge — se pravi za to uporabljeno opredmeteno ali živo delo — samo preneseni stroški za vzdrževanje bodisi družbenega produkcijskega sklada, bodisi družbenega konsumpcijskega sklada.
Povečanje vrednosti blaga, ki ga povzroče, porazdeli te stroške na različne vrste blaga pro rata, ker so za različne vrste blaga različni.
Prej ko slej ostanejo stroški tvorjenja zaloge odtegljaji od družbenega bogastva, čeprav so eden izmed pogojev za njegov obstoj.

Samo kolikor je blagovna zaloga pogoj blagovne cirkulacije in celo oblika, ki je nujno nastala v blagovni cirkulaciji, kolikor je torej ta navidezni zastoj oblika toka samega, podobno kakor je nastanek denarne zaloge pogoj denarnega kroženja — samo toliko je normalna.
Kakor hitro pa se blago, ki se mudi v svojem cirkulacijskem rezervoarju, ne umakne dohitevajočemu ga valu produkcije, brž ko so torej rezervoarji prenapolnjeni, se razširi blagovna zaloga zaradi zastoja cirkulacije, prav tako kakor naraščajo zakladi, če zastaja denarna cirkulacija.
Pri tem je vseeno, ali nastane ta zastoj v kaščah industrijskega kapitalista ali v skladiščih trgovca.\KPEstran\ 
V takem primeru blagovna zaloga ni pogoj nepretrgane prodaje, ampak posledica dejstva, da blaga ni mogoče prodati.
Stroški ostanejo enaki.
Ker pa izhajajo sedaj izključno iz oblike, namreč iz nujnosti, da se spremeni blago v denar, in iz težavnosti te metamorfoze, pa ne preidejo v vrednost blaga, ampak pomenijo odtegljaje, izgubo dela vrednosti pri njeni realizaciji.
Ker se normalna in nenormalna oblika zaloge po obliki ne razlikujeta in sta obe zastoj v cirkulaciji, se lahko pojava zamenjujeta drug z drugim, in toliko laže lahko prevarita celo produkcijskega agenta samega, ker za producenta cirkulacijski proces njegovega kapitala lahko poteka v redu, čeprav je cirkulacijski proces njegovega blaga, ki je prešlo v roke trgovcev, zastal.
Če se povečata obseg produkcije in konsumpcije, se poveča ob sicer nespremenjenih okoliščinah tudi obseg zaloge blaga.
Obnavlja in črpa se prav tako hitro, njen obseg pa je večji.
Naraščajoči obseg blagovne zaloge, ki ga povzroča zastoj v cirkulaciji, se zato lahko napak šteje za znamenje razširjanja reprodukcijskega procesa, posebno ko razvoj kreditnega sistema omogoči, da se prikrijejo dejanska gibanja.

Stroški tvorjenja zaloge sestoje iz: 1.\ količinskega skrčenja množine produktov (na primer pri zalogi moke); 2.\ kvara kakovosti: 3.\ popredmetenega in živega dela, ki ga zahteva vzdrževanje zaloge.

\section{Transportni stroški}
Tu ni treba obravnavati vseh podrobnosti cirkulacijskih stroškov, kakor pakiranja, sortiranja ipd.
Splošni zakon je, da \emph{ne dodajajo nobeni cirkulacijski stroški, ki izhajajo zgolj iz oblikovne spremembe, blagu nobene vrednosti}.
To so le stroški, potrebni za realizacijo vrednosti ali za njeno spremembo iz ene oblike v drugo.
V te stroške založeni kapital (vštevši delo, ki se uporabi zanj) spada k faux frais kapitalistične produkcije.
Nadomestiti jih je treba iz presežne vrednosti in so z vidika vsega razreda kapitalistov odtegljaj od presežne vrednosti ali presežnega produkta, tako\KPEstran\ kakor je za delavca čas, ki ga potrebuje za nakup svojih življenjskih potrebščin, izgubljen čas.
Transportni stroški pa igrajo prevažno vlogo, da jih ne bi tu vsaj na kratko obravnavali.

V krožnem toku kapitala in v blagovni metamorfozi, ki tvori enega mjegovih odsekov, se menjuje tvarina družbenega dela. 
Ta menjava lahko zahteva prostorsko premestitev produktov, njihovo dejansko premaknitev z enega kraja v drug kraj. 
Cirkulacija blaga pa se lahko izvede brez njihovega fizičnega premikanja in transport produktov brez blagovne cirkulacije in celo brez neposredne menjave produktov.
Hiša, ki jo A proda B, cirkulira kot blago, vendar pa ne gre na drug kraj.
Premične blagovne vrednosti, kakor bombaž ali surovo železo, tičijo v istem skladišču ob istem času, ko opravijo ducate cirkulacijskih procesov, ko jih špekulanti\footnote
{Storch imenuje to circulation factice [namišljena cirkulacija].}
kupujejo in znova prodajajo.
Kar se pri tem resnično premika, je lastninski naslov na stvari, ne stvar sama.
Na drugi strani pa je igrala transportna industrija, na primer v državi Inkov, veliko vlogo, čeprav družbeni produkt ni cirkuliral kot blago niti ga niso razdeljevali z neposredno zamenjavo.

Če se torej pojavlja transportna industrija na podlagi kapitalistične produkcije kot vzrok cirkulacijskih stroškov, ta posebna pojavna oblika v ničemer ne spremeni bistva stvari.

Transport ne povečuje množine produktov. Tudi sprememba njihovih naravnih lastnosti, ki jo morebiti povzroči, ni z nekaterimi izjemami noben nameravani koristni učinek, ampak le nujno zlo.
Toda uporabna vrednost stvari se uveljavi edinole v njihovi konsumpciji, in da bi se konsumirale, utegne biti potrebno, da se premestijo, to je, potreben je dodaten produkcijski proces transportne industrije.
V njo naloženi produktivni kapital dodaja torej premeščenim produktom vrednost, deloma s prenosom vrednosti transportnih sredstev, deloma z dodatkom vrednosti zaradi transportnega\KPEstran\ dela.
Le-ta dodatek vrednosti se deli tako kakor pri vsaki kapitalistični produkciji na nadomestilo za mezdo in na presežno vrednost.

V vsakem produkcijskem procesu je zelo pomembno premeščanje delovnega predmeta in za to potrebna delovna sredstva in delovne sile — na primer bombaž, ki gre iz mikalnice v predilnico, premog, ki ga iz jame spravljajo na površino.
Prehod gotovega produkta kot gotovega blaga z enega samostojnega produkcijskega mesta, na drugo, ki je prostorsko oddaljeno, kaže prav isti pojav, samo v večjem obsegu.
Transportu produktov z enega produkcijskega mesta na drugo sledi še transport gotovih produktov s produkcijskega področja v konsumpcijsko. 
Produkt je pripravljen za porabo šele, kadar dokonča to gibanje.

Kakor smo prej pokazali, se glasi splošni zakon blagovne produkcije: produktivnost dela in ustvarjanje vrednosti z delom sta si v obratnem sorazmerju.
Kakor za vsako drugo, velja to tudi za transportno industrijo.
Čim manjša je količina dela, mrtvega in živega, ki ga zahteva transport blaga na določeno razdaljo, tem večja je produktivna sila dela, in narobe.\footnote
{Ricardo navaja Saya, ki šteje za blagoslov trgovine, da s transportnimi stroški podražuje produkte ali povečuje njihovo vrednost.
»Trgovina,« pravi Say, »omogoča, da dobimo blago na kraju njegovega nastanka in da ga transportiramo na drug kraj, kjer ga konsumiramo; omogoča nam, da povečamo vrednost blaga za vso razliko med njegovo ceno na prvem in na drugem kraju.«
Ricardo pripominja k temu: »Prav, toda kako se mu dodaja dodatna vrednost?
Tako, da se dodajo produkcijskim stroškom, prvič, izdatki za transport in, drugič, profit od kapitala, ki ga je založil trgovec.
Blago je več vredno samo iz istega razloga, iz katerega lahko postane več vredno vsako drugo blago, ker se porabi za njegovo produkcijo in transport več dela, preden ga kupi konsument.
Tega ne smemo označiti kot eno od prednosti trgovine.«
(Ricardo, »Principles of Political Economy«, 5.\ izd., London 1821, str.\ 509, 510.)}

Absolutna velikost vrednosti, ki jo blagu dodaja transportna industrija, je ob sicer nespremenjenih okoliščinah v obratnem sorazmerju s produktivno silo transportne industrije\KPEstran\ in v premem sorazmerju z razdaljami, ki jih je treba premagati.

Relativni del vrednosti, ki ga ob sicer nespremenjenih okoliščinah dodajajo transportni stroški ceni blaga, je v premem sorazmerju z njegovo obsežnostjo in njegovo težo.
Je pa veliko okoliščin, ki spreminjajo to pravilo.
Transport zahteva na primer večje ali manjše varnostne ukrepe, torej večjo ali manjšo porabo dela in delovnih sredstev, kolikor je pač potrebno zaradi relativne krhkosti, minljivosti, eksplozivnosti predmeta.
Pri tem kažejo železniški mogotci s fantastičnim razčlenjevanjem vrst večjo genialnost kakor botaniki in zoologi.
Klasifikacija dobrin, na primer na angleških železnicah, obsega debele knjige in je po svojem splošnem načelu zasnovana na težnji spremeniti najrazličnejše naravne lastnosti dobrin v prav toliko transportnih težav in neizogibnih pretvez za goljufije.
»Steklo, ki je bilo prej vredno 11 f.\ št.\ per crate (zaboj določene prostornine), je zaradi izpopolnitev v industriji in odprave davka na steklo vredno sedaj samo 2 f.\ št., transportni stroški pa so prav tako visoki kakor prej, za prevoz po kanalih pa še višji.
Prej so razvažali steklo in steklenino za svinčeno obdelavo do 50 milj daleč od Birminghama po 10 šil.\ za tono.
Zdaj so povišali transportno ceno, češ da zaradi loma veliko tvegajo, na trikratno višino. 
Toda tisti, ki ne plača, kar se v resnici razbije, je uprava železnice.«\footnote
{Royal Commission on Railways«, str.\ 31, št.\ 650.}
Dalje je dejstvo, da je relativna velikost vrednosti, ki jo dodajajo predmetu prevozni stroški, v obratnem sorazmerju z njegovo vrednostjo, za železniške mogotce poseben razlog, da obremenijo tak predmet premo sorazmerno z njegovo vrednostjo. 
Pritožbe industrijcev in trgovcev nad tem se ponavljajo na vsaki strani pričevanj navedenega poročila.

Kapitalistični produkcijski način zmanjšuje transportne stroške za posamezno blago z razvojem transportnih in komunikacijskih sredstev kakor tudi s koncentracijo — velikostjo obsega — transporta.
Pomnožuje tisti del družbenega dela, živega in opredmetenega, ki se troši pri transportu blaga,\KPEstran\ najprej s spremembo velike večine vseh produktov v blago, potem pa še, ker stopajo na mesto krajevnih trgov oddaljeni trgi.

Cirkuliranje, se pravi dejansko kroženje blaga v prostoru, je v bistvu transport blaga.
Po eni strani je transportna industrija samostojna produkcijska panoga in zato posebno naložbeno področje produktivnega kapitala.
Po drugi strani pa se razlikuje po tem, da nastopa kot nadaljevanje produkcijskega procesa \emph{znotraj} cirkulacijskega procesa in za cirkulacijski proces.

\end{document}