\documentclass[kapital_02.tex]{subfiles}

\begin{document}

Proces \KPEstran krožnega toka\footnote{Iz rokopisa II.} kapitala poteka v treh stadijih, ki tvorijo, kot smo razložili v prvem zvezku, naslednje zaporedje:

\emph{Prvi stadij:} Kapitalist se pojavi na trgu blaga in dela kot kupec; njegov denar se spremeni v blago oziroma opravi cirkulacijski akt \(\KPED\KPEcrta\KPEB\).

\emph{Drugi stadij:} Kapitalist produktivno porabi kupljeno blago. Deluje kot kapitalistični producent blaga; njegov kapital opravi produkcijski proces. Rezultat: blago z večjo vrednostjo, kakor je vrednost njegovih produkcijskih elementov.

\emph{Tretji stadij:} Kapitalist se vrne na trg kot prodajalec; njegovo blago se spremeni v denar oziroma opravi cirkulacijski akt \(\KPEB\KPEcrta\KPED\).

Obrazec za krožni tok denarnega kapitala je torej:

\(\KPED\KPEcrta\KPEB\KPEpike\KPEP\KPEpike\KPEB'\KPEcrta\KPED'\), kjer nakazujejo pike prekinitev cirkulacijskega procesa, \(\KPEB'\) in \(\KPED'\) pa pomenita \(\KPEB\) in \(\KPED\), ki sta se povečala za presežno vrednost.

Prvi \KPEstran in tretji stadij smo obravnavali v prvi knjigi samo, kolikor je bilo nujno za razumevanje drugega stadija, produkcijskega procesa kapitala. Zato se nismo ozirali na različne oblike, v katere se oblači kapital v svojih različnih stadijih in ki jih pri ponavljanem krožnem toku zdaj privzema, zdaj odlaga. Te so zdaj neposredni predmet raziskave.

Da bi oblike lahko jasno doumeli, moramo predvsem odmisliti vse momente, ki nimajo nobene zveze s spreminjanjem in tvorjenjem oblik kot takima. Zato tu ne predpostavljamo samo, da se prodaja blago po svoji vrednosti, ampak tudi, da se to dogaja ob nespremenjenih okoliščinah. Odmislili bomo torej tudi spremembe vrednosti, ki lahko nastanejo med procesom krožnega toka.

\section{Prvi stadij. D--B\footnote{Od tod naprej rokopis VII, začet 2.\ julija 1878.}}

\(\KPED\KPEcrta\KPEB\) predstavlja spremembo denarne vsote v vsoto blaga; za kupca spremembo njegovega denarja v blago, za prodajalca spremembo njegovega blaga v denar. Kar napravi iz tega akta splošne blagovne cirkulacije hkrati tudi funkcionalno določen odsek v samostojnem krožnem toku individualnega kapitala, ni toliko oblika akta, kolikor njegova tvarna vsebina, posebni uporabni značaj blaga, ki zamenja svoje mesto z denarjem. Na eni strani so to produkcijska sredstva, na drugi delovna sila, materialni in osebni faktorji blagovne produkcije, ki morajo s svojim posebnim značajem seveda ustrezati vrsti predmeta, ki naj se izdela. Če označimo delovno silo z \(\KPEDs\), produkcijska sredstva s \(\KPEPs\), potem je vsota blaga, ki ga je treba kupiti, \(\KPEB\textrm{ = }\KPEDs\textrm{ + }\KPEPs\), ali krajše \(\KPEBrazcepDsPs\). Glede na svojo vsebino se torej izkaže \(\KPED\KPEcrta\KPEB\) kot \(\KPED\KPEcrta\KPEBrazcepDsPs\); z drugo besedo, \(\KPED\KPEcrta\KPEB\) razpade na \(\KPED\KPEcrta\KPEDs\) in \(\KPED\KPEcrta\KPEPs\); denarni znesek \(\KPED\) se razcepi na dva dela, od katerih kupi eden delovno silo, drugi produkcijska \KPEstran sredstva. Ti dve vrsi nakupov pripadata popolnoma različnima trgoma: prva trgu blaga v pravem pomenu, druga trgu dela.

\(\KPED\KPEcrta\KPEBrazcepDsPs\) pa kaže razen te kakovostne razcepitve blagovne vsote, v katero se pretvarja denar, še izredno značilno količinsko razmerje.

Znano nam je, da se plača vrednost oziroma cena delovne sile njenemu lastniku, ki jo prodaja kot blago, v obliki mezde, se pravi kot cena količine dela, ki vsebuje presežno delo; tako, da nastopa ta količina v pogodbi med kupcem in prodajalcem, če je dnevna vrednost delovne sile enaka na primer trem markam, tj. produktu peturnega dela, kot cena ali mezda, denimo, deseturnega dela. Če se sklene takšna pogodba s 50 delavci, morajo opraviti kupcu v enem dnevu skupno 500 delovnih ur, od katerih sestoji polovica, 250 delovnih ur ali 25 deseturnih delovnih dni, izključno iz presežnega dela. Množina kakor tudi obseg produkcijskih sredstev, ki jih je treba kupiti, morata zadostovati za uporabo te količine dela.

\(\KPED\KPEcrta\KPEBrazcepDsPs\) ne izraža torej le kakovostnega razmerja, po katerem se določeni denarni znesek, denimo 422\ f.\ št., pretvori v produkcijska sredstva in delovno silo, ki drug drugemu ustrezata, ampak tudi količinsko razmerje med obema deloma denarja, ki sta založena v delovno silo \(\KPEDs\) in v produkcijska sredstva \(\KPEPs\) in ki ga že vnaprej opredeljuje vsota prekomernega presežnega dela, katero mora opraviti določeno število delavcev.

Če znaša torej na primer v kaki predilnici tedenska mezda 50 delavcev 50\ f.\ št., tedaj je treba izdati za produkcijska sredstva 372\ f.\ št., če je to vrednost produkcijskih sredstev, ki pretvorijo 3000-urno tedensko delo, od katerega je 1500 ur presežnega dela, v prejo.

Tu je čisto vseeno, koliko dodatne vrednosti v obliki produkcijskih sredstev zahteva uporaba dodatnega dela v različnih industrijskih panogah. Važno je samo, da mora v \KPEstran vsakem primeru za nakup produkcijskih sredstev porabljeni del denarja -- oziroma v \(\KPED\KPEcrta\KPEPs\) nakupljenih produkcijskih sredstev -- zadoščati, biti torej vnaprej za ta namen preračunan in preskrbljen v ustreznem sorazmerju. Z drugo besedo, množina produkcijskih sredstev mora biti tolikšna, da lahko vsrka tisto količino dela, katera jo pretvori v produkt. Če ne bi bilo dovolj produkcijskih sredstev, ne bi bilo mogoče uporabiti dodatnega dela, ki je na voljo kupcu; njegova pravica, da razpolaga z njim, bi bila prazna. Če pa bi bilo več produkcijskih sredstev, kakor je razpoložljivega dela, jih delo ne bi nasitilo in se ne bi spremenila v produkt.

Kakor hitro je akt \(\KPED\KPEcrta\KPEBrazcepDsPs\) opravljen, nima kupec v rokah samo za produkcijo koristnega predmeta potrebnih produkcijskih sredstev in delovne sile. Pridobil je možnost, da uporabi v produkciji več delovne sile, torej večjo količino dela, kot je potrebna za nadomestitev vrednosti delovne sile, obenem s produkcijskimi sredstvi, ki jih potrebuje, da to količino dela ostvari ali popredmeti; v rokah ima torej faktorje, ki ustvarjajo predmete z večjo vrednostjo, kakor je vrednost njihovih produkcijskih elementov, oziroma množino blaga, ki vsebuje presežno vrednost. Vrednost, ki jo je bil založil v obliki denarja, ima sedaj takšno naturalno obliko, v kateri se lahko uveljavi kot vrednost, ki producira presežno vrednost. Z drugimi besedami: zdaj je v stanju ali obliki \emph{produktivnega kapitala}, ki je sposoben delovati kot ustvarjalec vrednosti in presežne vrednosti. Kapital v tej obliki imenujmo \(\KPEP\).

Vrednost \(\KPEP\) pa je enaka vrednosti \(\KPEDs\) plus \(\KPEPs\), enaka v \(\KPEDs\) in \(\KPEPs\) pretvorjenemu \(\KPED\). \(\KPED\) je ista kapitalska vrednost kot \(\KPEP\), samo da obstaja v drugačni obliki, namreč kapitalska vrednost v denarnem stanju oziroma denarni obliki -- \emph{denarni kapital}.

To dejanje splošne blagovne cirkulacije, \(\KPED\KPEcrta\KPEBrazcepDsPs\) ali v njegovi splošni obliki \(\KPED\KPEcrta\KPEB\), vsota blagovnih nakupov, je kot stopnja v samostojnem procesu krožnega toka kapitala \KPEstran zato hkrati sprememba kapitalske vrednosti iz njene denarne oblike v njeno produktivno obliko, ali krajše, sprememba \emph{denarnega kapitala} v \emph{produktivni kapital}. V podobi krožnega toka, ki ga tu predvsem proučujemo, je zato denar prvi nosilec kapitalske vrednosti, torej denarni kapital kot oblika, v kateri se kapital založi.

Kot denarni kapital je kapital v stanju, v katerem lahko opravlja funkcije denarja, kakor -- v obravnavanem primeru -- funkciji splošnega kupnega in splošnega plačilnega sredstva. (To, kolikor se delovna sila sicer najprej kupi, plača pa šele potem, ko je že delovala. Tudi pri \(\KPED\KPEcrta\KPEPs\) deluje denar prav tako kot plačilno sredstvo, če produkcijska sredstva niso že pripravljena na trgu, ampak jih je treba šele naročiti.) Ta sposobnost ne izvira iz dejstva, da je denarni kapital kapital, ampak iz dejstva, da je denar.

Po drugi plati lahko opravlja kapitalska vrednost v denarnem stanju samo denarne funkcije in nobenih drugih. To, kar naredi le-te za kapitalske funkcije, je njihova določena vloga v gibanju kapitala, od tod tudi povezanost stadija, v katerem se pojavljajo, z drugimi stadiji njegovega krožnega toka. Na primer: v primeru, ki ga tu predvsem obravnavamo, se spreminja denar v razne vrste blaga, katerih spojitev tvori naturalno obliko produktivnega kapitala, ki torej latentno, se pravi glede na možnost, že skriva v sebi rezultat kapitalističnega produkcijskega procesa.

Del denarja, ki opravlja v \(\KPED\KPEcrta\KPEBrazcepDsPs\) funkcijo denarnega kapitala, preide z izvedbo te cirkulacije v funkcijo, v kateri njegov kapitalski značaj izgine, ostane pa njegov denarni značaj. Cirkulacija denarnega kapitala \(\KPED\) se razdeli na \(\KPED\KPEcrta\KPEPs\) in \(\KPED\KPEcrta\KPEDs\), v nakup produkcijskih sredstev in nakup delovne sile. Oglejmo si zadnji akt sam zase! \(\KPED\KPEcrta\KPEDs\) je kapitalistov nakup delovne sile; in je prodaja delovne sile -- tu lahko rečemo dela, ker predpostavlja obliko mezde -- po delavcu, lastniku delovne sile. Kar je za kupca \(\KPED\KPEcrta\KPEB\) (enako \(\KPED\KPEcrta\KPEDs\)), je tu, tako kakor pri vsakem nakupu, za prodajalca (delavca) \(\KPEDs\KPEcrta\KPED\) (enako \(\KPEB\KPEcrta\KPED\)), prodaja njegove \KPEstran delovne sile. To je prvi stadij cirkulacije ali prva metamorfoza blaga (I.\ knjiga, 3.\ poglavje, 2.a); za prodajalca dela je sprememba njegovega blaga v njegovo denarno obliko. Tako prejeti denar delavec postopoma izdaja za takšno blago, ki zadovoljuje njegove potrebe, za potrošne predmete. Celotno menjavo njegovega blaga predstavlja torej \(\KPEDs\KPEcrta\KPED\KPEcrta\KPEB\), se pravi, prvič, \(\KPEDs\KPEcrta\KPED\) (enako \(\KPEB\KPEcrta\KPED\)) in, drugič, \(\KPED\KPEcrta\KPEB\), torej splošna oblika enostavne cirkulacije blaga \(\KPEB\KPEcrta\KPED\KPEcrta\KPEB\), v kateri nastopa denar zgolj kot prehodno menjalno sredstvo, zgolj kot posredovalec zamenjave blaga za blago.

\(\KPED\KPEcrta\KPEDs\) je značilni moment spremembe denarnega kapitala v produktivni kapital, ker je bistveni pogoj, da se v denarni obliki založena vrednost dejansko spremeni v kapital, v presežno vrednost ustvarjajočo vrednost. \(\KPED\KPEcrta\KPEPs\) je potreben samo, da popredmeti z \(\KPED\KPEcrta\KPEDs\) nakupljeno množino dela. S tega vidika smo torej prikazali \(\KPED\KPEcrta\KPEDs\) v drugem oddelku prve knjige ">Spreminjanje denarja v kapital"<. Tu moramo pogledati zadevo še z drugega vidika, s posebnim ozirom na denarni kapital kot pojavno obliko kapitala.

Na splošno velja akt \(\KPED\KPEcrta\KPEDs\) za značilen za kapitalistični produkcijski način. Vendar nikakor iz navedenega razloga, po katerem je nakup delovne sile kupna pogodba, s katero se dogovori dobava večje količine dela, kot je potrebna za nadomestitev cene delovne sile, mezde; torej dobava presežnega dela, ki je glavni pogoj za kapitalizacijo založene vrednosti ali, kar je isto, za produkcijo presežne vrednosti. Pač pa zaradi svoje oblike, ker se v obliki mezde kupuje delo z \emph{denarjem}, kar je značilno za denarno gospodarstvo.

Iracionalnost oblike tudi tu ne šteje za značilno. To iracionalnost puščamo celo ob strani. Iracionalnost je v tem, da delo samo kot vrednost ustvarjajoči element, torej tudi ne določena količina dela, ne more imeti nobene vrednosti, ki bi se izrazila v njegovi ceni, v njegovi enakovrednosti z določeno količino denarja. Znano pa nam je, da je mezda le \KPEstran prikrita oblika, oblika, v kateri nastopa na primer dnevna cena delovne sile kot cena dela, ki ga opravi ta delovna sila v enem dnevu, tako da se na primer vrednost, ki jo je ta delovna sila producirala v šestih urah dela, izraža kot vrednost njenega dvanajsturnega delovanja ali dela.

\(\KPED\KPEcrta\KPEDs\) velja za značilnost, za spoznavno znamenje tako imenovanega denarnega gospodarstva, ker nastopa tu delo kot blago svojega posestnika in zato denar kot kupec -- torej zaradi denarnega odnosa (se pravi nakupa in prodaje človekove dejavnosti). Denar pa nastopa že zelo zgodaj kot kupec tako imenovanih storitev, ne da bi se bil \(\KPED\) spremenil v denarni kapital in ne da bi se bil preobrnil splošni značaj gospodarstva.

Denarju je popolnoma vseeno, v kakšno vrsto blaga ga spremenimo. Denar je splošna oblika, v kateri izpričajo ekvivalentnost vse vrste blaga, ki kažejo že s svojimi cenami, da so idealno neki denarni znesek, da čakajo na svojo spremembo v denar in da pridobe le z zamenjavo svojega mesta z denarjem obliko, v kateri se lahko pretvorijo v uporabne vrednosti za svoje posestnike. Če se torej pojavi na trgu delovna sila kot blago svojega posestnika in se prodaja v obliki plačila za delo, v obliki mezde, tedaj njen nakup in prodaja nista prav nič bolj nenavadna kot nakup in prodaja katerega koli drugega blaga. Značilnost ni v tem, da je blago -- delovno silo mogoče kupiti, ampak v tem, da se delovna sila pojavi kot blago.

Z \(\KPED\KPEcrta\KPEBrazcepDsPs\), spremembo denarnega kapitala v produktivni kapital, spoji kapitalist materialne in osebne produkcijske faktorje, kolikor obstajajo ti faktorji kot blago. Če se pretvori denar prvič v produktivni kapital ali če prvič deluje kot denarni kapital za svojega posestnika, mora ta kupiti prej produkcijska sredstva, delavnice, stroje itd., preden kupi delovno silo; kakor hitro preide delovna sila v njegovo oblast, morajo biti produkcijska sredstva že tu, da jo lahko uporabi kot delovno silo.

Tako je to s kapitalistovega vidika.

Z \KPEstran vidika delavca: produktivno delovanje njegove delovne sile je mogoče šele od trenutka, v katerem se, ker je bila prodana, združi s produkcijskimi sredstvi. Pred prodajo je torej ločena od produkcijskih sredstev, materialnih pogojev svojega delovanja. Dokler je ločena od produkcijskih sredstev, je ni mogoče uporabiti niti neposredno za produkcijo uporabnih vrednosti za njenega posestnika niti za produkcijo blaga, od katerega prodaje bi ta lahko živel. Kakor hitro pa se s svojo prodajo združi s produkcijskimi sredstvi, je sestavni del produktivnega kapitala svojega kupca, prav tako kakor so to produkcijska sredstva.

Čeprav se torej obnašata v aktu \(\KPED\KPEcrta\KPEDs\) posestnik denarja in posestnik delovne sile drug nasproti drugemu le kot kupec in prodajalec, četudi stopata drug nasproti drugemu le kot posestnik denarja in posestnik blaga in sta torej po tej plati sam\'o v denarnem odnosu, nastopa vendar kupec že prav od početka hkrati kot posestnik produkcijskih sredstev, ki so materialni pogoji, da lahko posestnik produktivno porablja svojo delovno silo. Z drugimi besedami: ta produkcijska sredstva nastopajo nasproti posestniku delovne sile kot tuja lastnina. Na drugi strani nastopa prodajalec dela nasproti kupcu dela kot tuja delovna sila, ki mora preiti v njegovo oblast, se vključiti v njegov kapital, da ta resnično lahko deluje kot produktivni kapital. Razredno razmerje med kapitalistom in mezdnim delavcem torej že obstaja, je že predpostavljeno v trenutku, ko stopita v aktu \(\KPED\KPEcrta\KPEDs\) (\(\KPEDs\KPEcrta\KPED\) s strani delavca) drug nasproti drugemu. To je nakup in prodaja, denarni odnos, vendar takšen nakup in prodaja, v katerem je kupec že vnaprej kapitalist, prodajalec pa že vnaprej mezdni delavec, in je to razmerje določeno z okoliščino, da so pogoji za ostvaritev delovne sile -- življenjske potrebščine in produkcijska sredstva -- kot tuja lastnina ločeni od posestnika delovne sile.

Kako ta ločitev nastane, nas tu ne zanima. Obstaja, kadarkoli pride do akta \(\KPED\KPEcrta\KPEDs\). Kar nas tu zanima, je naslednje: če se pojavlja \(\KPED\KPEcrta\KPEDs\) kot funkcija denarnega kapitala ali denar kot eksistenčna oblika kapitala, tedaj nikakor \KPEstran ne le zaradi tega, ker nastopa tu denar kot plačilno sredstvo za človeško dejavnost, ki ima koristen učinek, za kakšno storitev; nikakor torej ne zaradi funkcije denarja kot plačilnega sredstva. V tej obliki se lahko izda denar edinole, ker je delovna sila ločena od svojih produkcijskih sredstev (vštevši življenjske potrebščine kot produkcijska sredstva same delovne sile) in ker se ta ločenost odpravi samo tako, da se delovna sila proda posestniku produkcijskih sredstev; da pripada kupcu torej tudi delovanje delovne sile, katerega meje nikakor ne sovpadajo z mejami količine dela, ki je potrebna za reprodukcijo njene lastne cene. Kapitalski odnos se pojavi v produkcijskem procesu samo zato, ker obstaja sam po sebi v cirkulacijskem aktu, v različnih temeljnih ekonomskih pogojih, v katerih nastopata drug nasproti drugemu kupec in prodajalec, v njunem razrednem odnosu. Ni denar tisti, ki s svojo naravo ustvarja ta odnos; pač pa je obstoj tega odnosa tisto, kar lahko spremeni zgolj denarno funkcijo v funkcijo kapitala.

V pojmovanju denarnega kapitala (z njim imamo za zdaj opravka le znotraj določene funkcije, v kateri se nam tu odkriva) hodita po navadi vštric ali se prepletata dve zmoti. Prva: funkcije, ki jih opravlja kapitalska vrednost kot denarni kapital, in ki jih lahko opravlja prav zato, ker se nahaja v denarni obliki, se napačno izvajajo iz njenega kapitalskega značaja; v resnici so le posledica denarnega stanja kapitalske vrednosti, njene denarne pojavne oblike. Druga pa prav nasprotno: specifična vsebina denarne funkcije, ki jo dela hkrati za kapitalsko funkcijo, se izvaja iz narave denarja (denar se torej zamenjuje s kapitalom), medtem ko predpostavlja, kakor tu v izvedbi \(\KPED\KPEcrta\KPEDs\), družbene razmere, ki v goli blagovni in ustrezni denarni cirkulaciji nikakor niso podane.

Tudi nakup in prodaja sužnjev je po svoji obliki nakup in prodaja blaga. Brez obstoja suženjstva pa denar ne more opravljati te funkcije. Če obstaja suženjstvo, se denar troši za nakup sužnjev. Nasprotno pa denar v kupčevi roki nikakor ne zadošča, da bi omogočil suženjstvo.

Dejstvo, da \KPEstran prodaja lastne delovne sile (v obliki prodaje lastnega dela za mezdo) ni izoliran pojav, ampak družbeno odločilna predpostavka blagovne produkcije, da opravlja torej denarni kapital funkcijo \(\KPED\KPEcrta\KPEBrazcepDsPs\), ki jo tu proučujemo, v družbenem merilu, predpostavlja zgodovinske procese, ki so raztrgali prvotno spojenost produkcijskih sredstev z delovno silo; procese, zaradi katerih si stoje nasproti ljudske množice, delavci kot nelastniki in nedelavci kot lastniki teh produkcijskih sredstev. Pri tem je vseeno, ali je imela ta spojenost pred svojim razkrojem tako obliko, da je pripadal delavec kot produkcijsko sredstvo drugim produkcijskim sredstvom ali pa je bil njihov lastnik.

Dejstvo, na katerem temelji akt \(\KPED\KPEcrta\KPEBrazcepDsPs\), je razdelitev; ne razdelitev v navadnem smislu kot razdelitev potrošnih sredstev, ampak razdelitev samih produkcijskih elementov, od katerih so materialni faktorji koncentrirani na eni strani, delovna sila pa izolirana od njih na drugi.

Produkcijska sredstva, materialni del produktivnega kapitala, morajo stati nasproti delavcu torej že kot taka, kot kapital, preden lahko postane akt \(\KPED\KPEcrta\KPEDs\) splošno družbeni akt.

Poprej smo videli, da kapitalistična produkcija, ko je uvedena, v svojem razvoju to ločitev ne samo reproducira, ampak da jo razširja v vedno večjem obsegu, dokler ne postane splošno vladajoče družbeno stanje. Stvar pa ima še drugo plat. Da lahko kapital nastane in se polasti produkcije, je potrebna določena razvojna stopnja trgovine, torej tudi blagovne cirkulacije in s tem blagovne produkcije; predmeti namreč ne morejo vstopati v cirkulacijo kot blago, če niso producirani za prodajo, torej kot blago. Blagovna produkcija kot normalna, prevladujoča značilnost produkcije pa se pojavi šele na osnovi kapitalistične produkcije.

Ruski zemljiški lastniki, ki obdelujejo zaradi tako imenovane kmečke odveze svoja posestva z mezdnimi delavci namesto \KPEstran s podložnimi prisilnimi delavci, tarnajo nad dvojim. Najprej nad pomanjkanjem denarnega kapitala. Tako na primer pravijo: Preden prodamo pridelke, moramo v večjih zneskih plačati mezdne delavce, za to pa nam manjka prvi pogoj, gotovina. Kapital v obliki denarja mora biti vedno na razpolago za kapitalistično izvrševanje produkcije prav zaradi plačila mezd. Toda zemljiški lastniki naj se glede tega potolažijo. Čas zaceli vse rane in industrijski kapitalist bo imel ne samo svoj denar, ampak tudi l'argent des autres [denar drugih].

Bolj značilna pa je druga pritožba, namreč da ne dobe za nakup v zadostnem številu in v vsakem času proste delovne sile, četudi imajo denar, kajti ruski poljski delavec zaradi tega, ker je zemlja skupna last vaške občine, še ni popolnoma ločen od svojih produkcijskih sredstev in torej še ni ">svobodni mezdni delavec"< v pravem pomenu besede. Obstoj le-tega v družbenem merilu pa je neizogibni pogoj, da se \(\KPED\KPEcrta\KPEB\), sprememba denarja v blago, lahko pokaže kot sprememba denarnega kapitala v produktivni kapital.

Samo po sebi je torej umljivo, da je obrazec krožnega toka denarnega kapitala \(\KPED\KPEcrta\KPEB\KPEpike\KPEP\KPEpike\KPEB'\KPEcrta\KPED'\) samoumevna oblika krožnega toka kapitala le na podlagi že razvite kapitalistične produkcije, ker predpostavlja obstoj razreda mezdnih delavcev v družbenem obsegu. Kakor smo videli, kapitalistična produkcija ne producira samo blaga in presežne vrednosti; reproducira, in v vedno večjem obsegu, razred mezdnih delavcev in spreminja ogromno večino neposrednih producentov v mezdne delavce. Ker je prva predpostavka, da se lahko izvrši akt \(\KPED\KPEcrta\KPEB\KPEpike\KPEP\KPEpike\KPEB'\KPEcrta\KPED'\), nenehno eksistiranje razreda mezdnih delavcev, predpostavlja torej že kapital v obliki produktivnega kapitala, zato pa obliko krožnega toka produktivnega kapitala.

\section{Drugi stadij. Delovanje produktivnega kapitala}

Krožni tok kapitala, ki ga proučujemo tu, se začenja s cirkulacijskim aktom \(\KPED\KPEcrta\KPEB\), spremembo denarja v blago, z nakupom. Cirkulacijo je treba torej dopolniti z nasprotno metamorfozo \KPEstran \(\KPEB\KPEcrta\KPED\), spremembo blaga v denar, s prodajo. Neposredni rezultat \(\KPED\KPEcrta\KPEBrazcepDsPs \) pa je prekinitev cirkulacije v denarni obliki založene kapitalske vrednosti. S spremembo denarnega kapitala v produktivni kapital je dosegla kapitalska vrednost naturalno obliko, v kateri ne more krožiti dalje, ampak mora priti v potrošnjo, namreč v produktivno potrošnjo. Uporaba delovne sile, tj.\ delo, je možna le v delovnem procesu. Kapitalist ne more delavca prodati kot blago naprej, ker ni njegov suženj in ker ni kupil nič drugega kot uporabo njegove delovne sile za določen čas. Po drugi strani lahko porabi delovno silo le tako, da z njo porabi produkcijska sredstva kot tvorce blaga. Rezultat prvega stadija je torej vstop v drugi stadij, produktivni stadij kapitala.

To gibanje se izraža kot \(\KPED\KPEcrta\KPEBrazcepDsPs\KPEpike\KPEP \), pri čemer pomenijo pike, da je cirkulacija kapitala prekinjena, proces njegovega krožnega toka pa se nadaljuje s prehodom s področja blagovne cirkulacije na produkcijsko področje. Prvi stadij, sprememba denarnega kapitala v produktivni kapital, je torej le predhodnik in uvodna faza drugega stadija, delovanja produktivnega kapitala.

\(\KPED\KPEcrta\KPEBrazcepDsPs\) ne predpostavlja samo, da razpolaga individuum, ki opravlja ta akt, z vrednostmi poljubne uporabne oblike, ampak da ima te vrednosti v denarni obliki, da je posestnik denarja. Bistvo akta pa je prav oddaja denarja, zaradi česar lahko ostane posestnik denarja le, če mu denar priteka implicite s samim aktom oddaje. Denar pa mu lahko priteka nazaj le s prodajo blaga. Akt ga zato predpostavlja kot producenta blaga.

\(\KPED\KPEcrta\KPEDs\). Mezdni delavec živi samo od prodaje svoje delovne sile. Njeno vzdrževanje -- njegova samoohranitev -- zahteva vsakdanjo potrošnjo. Treba ga je torej plačevati vedno znova v krajših rokih, da lahko ponavlja za svojo samoohranitev \KPEstran nujne nakupe -- akt \(\KPEDs\KPEcrta\KPED\KPEcrta\KPEB\) ali \(\KPEB\KPEcrta\KPED\KPEcrta\KPEB\). Kapitalist se mora torej trajno pojavljati pred njim kot denarni kapitalist, njegov kapital pa kot denarni kapital. Po drugi strani pa se morajo nasproti množici neposrednih producentov, mezdnih delavcev, trajno nahajati nujne življenjske potrebščine v nakupni, se pravi v blagovni obliki, da lahko opravijo akt \(\KPEDs\KPEcrta\KPED\KPEcrta\KPEB\). Ta položaj zahteva torej že visoko stopnjo cirkulacije produktov kot blaga, torej tudi obsega blagovne cirkulacije. Kakor hitro postane produkcija na osnovi mezdnega dela splošna, mora biti blagovna produkcija splošna oblika produkcije. Ko se ta uveljavi kot splošna, zahteva vedno večjo delitev družbenega dela, se pravi vedno večje diferenciranje produktov, ki jih producira določeni kapitalist kot blago, vedno večjo cepitev dopolnjujočih se produkcijskih procesov v osamosvojene. V enaki meri kot \(\KPED\KPEcrta\KPEDs\) se razvija zato tudi \(\KPED\KPEcrta\KPEPs\); se pravi, v istem obsegu se ločuje produkcija produkcijskih sredstev od produkcije blaga, v kateri delujejo produkcijska sredstva, ter nastopajo ta nasproti vsakemu blagovnemu producentu kot blago, ki ga le-ta ne producira, ampak kupuje za potrebe svojega določenega produkcijskega procesa. Prihajajo iz samostojno obratujočih produkcijskih panog, ki so od njegove popolnoma ločene, in vstopajo v njegovo produkcijsko panogo kot blago in jih je treba zato kupiti. Materialni pogoji blagovne produkcije se pojavljajo pred njim v vedno večjem obsegu kot produkti drugih blagovnih producentov, kot blago. V prav istem obsegu mora nastopati kapitalist kot denarni kapitalist oziroma se razširja obseg, v katerem mora njegov kapital delovati kot denarni kapital.

Po drugi strani: prav iste okoliščine, ki ustvarjajo temeljne pogoje za kapitalistično produkcijo -- obstoj razreda mezdnih delavcev -- pospešujejo prehod vse blagovne produkcije v kapitalistično blagovno produkcijo. V obsegu, v katerem se ta razvija, razkraja in razdira vsako starejšo obliko produkcije, ki zaradi naravnanosti na neposredne lastne potrebe spreminja v blago le presežek produkta. Prodajo \KPEstran produkta napravi za glavni namen, spočetka ne da bi vidno napadla sam način produkcije; takšen je bil na primer prvi učinek kapitalistične svetovne trgovine na narode, kakor so npr.\ Kitajci, Indijci, Arabci itd. Ko pa se kasneje zakorenini, uniči vse oblike blagovne produkcije, ki temelje bodisi na lastnem delu producenta, bodisi na prodaji zgolj presežnega produkta kot blaga. Najprej blagovno produkcijo posploši, potem pa postopoma spremeni vso blagovno produkcijo v kapitalistično.\footnote{Do tu rokopis VII. Od tod naprej rokopis VI.}

Kakršne koli že so družbene oblike produkcije, delavci in produkcijska sredstva so vedno njeni faktorji. Dokler so ločeni drug od drugega, pa so tako eni kot drugi samo možni faktorji. Če naj se sploh producira, se morajo združiti. Po posebnem načinu, kako se ta zveza vzpostavi, razločujemo različna ekonomska obdobja družbene strukture. V obravnavanem primeru je dano izhodišče ločitev svobodnega delavca od njegovih produkcijskih sredstev; videli pa smo že, kako in v kakšnih pogojih se združita oba v kapitalistovih rokah -- namreč kot produktivna oblika njegovega kapitala. Dejanski proces, v katerem se tako združijo tvarni in osebni faktorji produkcije blaga, produkcijski proces, postane zato samo funkcija kapitala -- kapitalistični produkcijski proces, katerega naravo smo izčrpno razložili v prvi knjigi tega spisa. Vsako produciranje blaga je hkrati izkoriščanje delovne sile; toda šele kapitalistična produkcija blaga postane tak način izkoriščanja, ki uvede novo dobo, ki v svojem nadaljnjem zgodovinskem razvoju prevrže z organizacijo delovnega procesa in z orjaškim napredkom tehnike vso ekonomsko strukturo družbe in neizmerno preseže vsa prejšnja obdobja.

Po različnih vlogah, ki jih igrajo v produkcijskem procesu pri nastajanju vrednosti in torej tudi pri nastajanju presežne vrednosti, se produkcijska sredstva in delovna sila, kolikor so eksistenčne oblike založene kapitalske vrednosti, razlikujejo kot konstantni kapital in variabilni kapital. Kot \KPEstran različni sestavni deli produktivnega kapitala se razlikujejo poleg tega še po tem, da ostanejo produkcijska sredstva v posesti kapitalista, njegov kapital tudi zunaj produkcijskega procesa, medtem ko je delovna sila eksistenčna oblika individualnega kapitala samo znotraj njegovih meja. Medtem ko je delovna sila blago samo v rokah svojega prodajalca, mezdnega delavca, pa postane kapital le v rokah svojega kupca, kapitalista, ki mu pripada njena začasna uporaba. Produkcijska sredstva sama postanejo stvarni deli produktivnega kapitala oziroma produktivni kapital šele od trenutka, v katerem se mu lahko pripoji delovna sila kot osebna oblika njegovega obstoja. Prav tako kot človeška delovna sila tudi produkcijska sredstva po naravi niso kapital. Ta posebni družbeni značaj pridobe le v določenih zgodovinsko nastalih razmerah, podobno kot pridobe le v takšnih razmerah žlahtne kovine značaj denarja ali celo denar značaj denarnega kapitala.

Medtem ko produktivni kapital deluje, troši svoje lastne sestavine, da bi jih spremenil v več vredno množino produktov. Ker deluje delovna sila le kot eden njegovih organov, je tudi z njenim presežnim delom producirani presežek vrednosti produkta nad vrednostjo njegovih tvornih elementov plod kapitala. Presežno delo delovne sile je brezplačno delo za kapital in tvori zato za kapitalista presežno vrednost, za katero ne plača nobenega ekvivalenta. Produkt torej ni le blago, ampak s presežno vrednostjo oplojeno blago. Njegova vrednost je enaka \(\KPEP \textrm{ + } \KPEPv\), vrednosti za njegovo produkcijo porabljenega produktivnega kapitala \(\KPEP\) in presežni vrednosti \(\KPEPv\), ki jo proizvede. Denimo, da sestoji to blago iz 10.000 funtov preje, za katere produkcijo je bilo porabljenih produkcijskih sredstev v vrednosti 372\ f.\ št., delovne sile pa v vrednosti 50\ f.\ št. Med procesom predenja prenesejo predilci vrednost s svojim delom porabljenih produkcijskih sredstev v znesku 372\ f.\ št.\ na prejo, hkrati pa ustvarijo ustrezno svoji porabi dela novo vrednost v višini, denimo, 128\ f.\ št. Teh 10.000 funtov preje vsebuje torej vrednost 500\ f.\ št.

\section{Tretji stadij. B\ensuremath{^\prime}--D\ensuremath{^\prime}}

Blago \KPEstran postane \emph{blagovni kapital} kot funkcionalna eksistenčna oblika že povečane vrednosti kapitala, ki izvira neposredno iz produkcijskega procesa samega. Če bi bila blagovna produkcija v vsem svojem družbenem obsegu kapitalistična, bi bilo vsako blago prav od početka element blagovnega kapitala, naj bo surovo železo ali bruseljske čipke, žveplena kislina ali cigare. Vprašanje, katerim vrstam blagovne vojske pripada zaradi njihove kakovosti čin kapitala, katere pa so določene za navadno blagovno službo, je ena od milih nadlog, ki si jih sholastična ekonomija sama dela.

V svoji blagovni obliki mora opravljati kapital blagovno funkcijo. Predmeti, iz katerih obstaja in ki se že od početka producirajo za trg, se morajo prodati, spremeniti v denar, opraviti torej gibanje \(\KPEB\KPEcrta\KPED\).

Denimo, da sestoji kapitalistovo blago iz 10.000 funtov bombažne preje. Če se je porabilo v predilnem procesu 372\ f.\ št.\ vrednosti produkcijskih sredstev in je bila ustvarjena nova vrednost 128\ f.\ št., ima preja vrednost 500\ f.\ št., ki jo izraža v svoji ceni enakega imena. To ceno vnovči prodaja \(\KPEB\KPEcrta\KPED\). Kaj napravi iz tega preprostega akta vsake blagovne cirkulacije hkrati funkcijo kapitala? Nobena sprememba, ki bi nastala med njegovim potekom, niti v pogledu njegove uporabne vrednosti, kajti kot uporabni predmet preide blago h kupcu, niti v pogledu njegove vrednosti, kajti ta svoje velikosti ne spremeni, ampak spremeni le obliko. Prej je obstajala v preji, sedaj obstaja v denarju. Tako se pokaže bistvena razlika med prvim stadijem \(\KPED\KPEcrta\KPEB\) in zadnjim stadijem \(\KPEB\KPEcrta\KPED\). Tam deluje založeni kapital kot denarni kapital, ker se s posredovanjem menjave preobrazi v blago specifičnih uporabnih vrednosti. Tu pa lahko deluje blago kot kapital samo, če prinese ta značaj že izoblikovan s seboj iz produkcijskega procesa, preden začne njegova cirkulacija. Pri predenju so ustvarili predilci vrednost preje v znesku 128\ f.\ št. Od tega je za kapitalista, recimo, 50\ f.\ št.\ samo ekvivalent za to, kar je založil za delovno silo, 78\ f.\ št.\ pa \KPEstran tvori -- s stopnjo izkoriščanja delovne sile 156~\% -- presežno vrednost. Vrednost 10.000 funtov preje vsebuje torej, prvič, vrednost porabljenega produktivnega kapitala \(\KPEP\), od katerega znaša konstantni kapital 372\ f.\ št., variabilni 50\ f.\ št., oba skupaj 422\ f.\ št., enako 8440 funtom preje. Vrednost produktivnega kapitala \(\KPEP\) pa je enaka \(\KPEB\), vrednosti njegovih sestavnih elementov, ki so stale v stadiju \(\KPED\KPEcrta\KPEB\) nasproti kapitalistu kot blago v rokah svojih prodajalcev. -- Drugič, vrednost preje vsebuje presežno vrednost 78\ f.\ št., kar je enako 1560 funtom preje. \(\KPEB\) kot vrednostni izraz 10.000 funtov preje je torej \(\textrm{= } \KPEB \textrm{ + } \Delta\KPEB\), \(\KPEB\) plus prirastek (ki znaša 78\ f.\ št.), ki ga bomo imenovali \(\KPEb\), ker ima enako blagovno obliko, kot jo ima sedaj prvotna vrednost \(\KPEB\). Vrednost 10.000 funtov preje ali 500\ f.\ št.\ je torej enaka \(\KPEB \textrm{ + } \KPEb \textrm{ = } \KPEB'\). To, kar naredi \(\KPEB\) kot vrednostni izraz 10.000 funtov preje za \(\KPEB'\), ni absolutna velikost njegove vrednosti (500\ f.\ št.), kajti tako kakor pri vseh drugih \(\KPEB\) jo podobno kot vrednostni izraz katere koli druge vsote blaga določa velikost v njem opredmetenega dela. To je njegova relativna velikost vrednosti, velikost njegove vrednosti v primerjavi z vrednostjo v njegovi produkciji porabljenega kapitala \(\KPEP\). V sebi nosi to vrednost plus presežno vrednost, ki jo je dal produktivni kapital. Njegova vrednost je večja, presega to vrednost kapitala za presežno vrednost \(\KPEb\). Teh 10.000 funtov preje je nosilec povečane, s presežno vrednostjo obogatene vrednosti kapitala, in to kot produkt kapitalističnega produkcijskega procesa. \(\KPEB'\) izraža vrednostno razmerje, razmerje vrednosti blagovnega produkta do vrednosti v njegovi produkciji porabljenega kapitala, torej sestavljenost njegove vrednosti iz kapitalske vrednosti in presežne vrednosti. Teh 10.000 funtov preje je blagovni kapital, \(\KPEB'\), samo kot spremenjena oblika produktivnega kapitala \(\KPEP\), torej v povezanosti, ki obstaja najprej le v krožnem toku tega individualnega kapitala, oziroma za kapitalista, ki je produciral prejo s svojim kapitalom. Tisto, kar napravi 10.000 funtov preje kot nosilca vrednosti za blagovni kapital, je tako rekoč samo notranje, ne kako zunanje razmerje; svojega kapitalističnega materinskega znamenja ne nosijo v absolutni \KPEstran velikosti svoje vrednosti, ampak v svoji relativni velikosti, v velikosti svoje vrednosti, primerjane z vrednostjo, ki jo je imel v njih vsebovani produktivni kapital, preden se je spremenil v blago. Če se torej proda 10.000 funtov preje po njeni vrednosti 500\ f.\ št., tedaj je cirkulacijski akt sam po sebi \(\KPEB\KPEcrta\KPED\), samo sprememba nespremenjene vrednosti iz blagovne oblike v denarno obliko. Kot poseben stadij v krožnem toku posameznega kapitala pa pomeni isti akt vnovčenje v blagu vsebovane vrednosti kapitala 422\ f.\ št.\ plus v njem vsebovane presežne vrednosti 78\ f.\ št., torej \(\KPEB'\KPEcrta\KPED'\), spremembo blagovnega kapitala iz njegove blagovne oblike v denarno obliko.\footnote{Do tu rokopis VI. Od tod naprej rokopis V.}

Funkcija \(\KPEB'\) je torej enaka kakor funkcija vsakega blagovnega produkta: spremeniti se mora v denar, prodati, opraviti cirkulacijsko fazo \(\KPEB\KPEcrta\KPED\). Dokler vztraja kapital, katerega vrednost je sedaj povečana, v obliki blagovnega kapitala, dokler tiči na trgu, produkcijski proces miruje. Ne deluje niti kot ustvarjalec produktov niti kot ustvarjalec vrednosti. V skladu z večjo ali manjšo hitrostjo, s katero odlaga kapital svojo blagovno obliko in privzema denarno, ali v skladu z bolj ali manj hitro prodajo služi ista vrednost kapitala v zelo neenaki meri kot ustvarjalka produktov in vrednosti in se razširja ali krči obseg reprodukcije. V prvi knjigi smo pokazali, da določajo učinkovitost danega kapitala potence produkcijskega procesa, ki so v določeni meri neodvisne od velikosti njegove lastne vrednosti. Tu vidimo, da spravlja cirkulacijski proces v gibanje nove potence njegove učinkovitosti, njegovega širjenja in krčenja, ki so neodvisne od velikosti vrednosti kapitala.

Nadalje mora količina \(\KPEB'\) blaga kot nosilka kapitala. katerega vrednost je narasla, opraviti metamorfozo \(\KPEB'\KPEcrta\KPED'\) v vsem svojem obsegu. Prodana količina je tu bistvenega pomena. Posamezno blago velja le še kot sestavni del celotne količine. Vrednost 500\ f.\ št.\ je v 10.000 funtih preje. Če proda kapitalist samo 7440 funtov za njihovo vrednost 379\ f.\ št., tedaj nadomesti le vrednost svojega konstantnega kapitala, \KPEstran vrednost porabljenih produkcijskih sredstev; če 8440 funtov, potem le vrednost celotnega založenega kapitala. Da bi realiziral presežno vrednost, mora prodati več, če hoče realizirati celotno presežno vrednost 78\ f.\ št.\ (= 1560 funtom preje), mora prodati vseh 10.000 funtov preje. V 500\ f.\ št.\ denarja prejme torej le prodanemu blagu enako vrednost; njegova transakcija v cirkulaciji je preprosti akt \(\KPEB\KPEcrta\KPED\). Če bi plačal svojim delavcem 64\ f.\ št.\ namesto 50\ f.\ št.\ mezd, bi znašala njegova presežna vrednost le 64\ f.\ št.\ namesto 78\ f.\ št., mera izkoriščanja pa le 100~\% namesto 156~\%. Vrednost njegove preje pa bi ostala slej ko prej nespremenjena. Le razmerje med njenimi različnimi deli bi bilo drugačno. Akt cirkulacije \(\KPEB\KPEcrta\KPED\) bi bil slej ko prej prodaja 10.000 funtov preje za 500\ f.\ št., za njihovo vrednost.

\(\KPEB' \textrm{ = } \KPEB \textrm{ + } \KPEb\) (= 422\ f.\ št.\ + 78\ f.\ št.). -- \(\KPEB\) je enak vrednosti \(\KPEP\) ali produktivnemu kapitalu, ta pa je enak vrednosti \(\KPED\), ki je bil založen v \(\KPED\KPEcrta\KPEB\), nakup produkcijskih elementov; v našem primeru je = 422\ f.\ št. Če proda kapitalist to blago po njegovi vrednosti, potem je \(\KPEB\) = 422\ f.\ št.\ in \(\KPEb\) = 78\ f.\ št., vrednosti presežne vrednosti 1560 funtov preje. Če označimo z \(\KPEd\) v denarju izraženi \(\KPEb\), dobimo \(\KPEB'\KPEcrta\KPED' \textrm{ = } (\KPEB \textrm{ + } \KPEb)\KPEcrta(\KPED\textrm{ + }\KPEd)\), in krožni tok \(\KPED\KPEcrta\KPEB\KPEpike\KPEP\KPEpike\KPEB'\KPEcrta\KPED'\) v svoji eksplicitni obliki je torej \(\KPED\KPEcrta\KPEBrazcepDsPs\KPEpike\KPEP\KPEpike(\KPEB \textrm{ + } \KPEb)\KPEcrta(\KPED \textrm{ + } \KPEd)\).

V prvem stadiju odtegne kapitalist uporabni predmet pravemu blagovnemu trgu in trgu delovne sile; v tretjem stadiju vrže blago vnovič na trg, vendar samo na en sam trg, na pravi blagovni trg. Toda če odtegne potem s svojim blagom trgu več vrednosti, kot je je bil prvotno vrgel nanj, lahko stori to le, ker vrže nanj večjo blagovno vrednost, kakor jo je prvotno odtegnil. Nanj je vrgel vrednost \(\KPED\), odtegnil pa je enako vrednost \(\KPEB\); nanj vrže vrednost \(\KPEB \textrm{ + } \KPEb\), odtegne pa enako vrednost \(\KPED \textrm{ + } \KPEd\). -- V našem primeru je bil \(\KPED\) enak vrednosti 8440 funtom preje; na trg pa vrže 10.000 funtov preje, daje mu torej večjo vrednost, kakor mu jo je odvzel. Po drugi strani pa je vrgel to naraslo vrednost nanj le zato, ker je produciral v produkcijskem procesu \KPEstran z izkoriščanjem delovne sile presežno vrednost (kot alikvotni del produkta, izražen v presežnem produktu). Samo kot produkt tega procesa je množina blaga blagovni kapital, nosilec s presežno vrednostjo povečane vrednosti kapitala. Z izvedbo \(\KPEB'\KPEcrta\KPED'\) se vnovči tako založena vrednost kapitala kakor tudi presežna vrednost. Obedve se vnovčita skupaj v vrsti prodaj ali tudi v hkratni prodaji celotne količine blaga, ki jo izraža \(\KPEB'\KPEcrta\KPED'\). Ta isti cirkulacijski akt \(\KPEB'\KPEcrta\KPED\) pa je toliko različen za kapitalsko in za presežno vrednost, kolikor pomeni za vsako od njiju drugačen stadij njune cirkulacije, drug odsek zaporedja metamorfoz, skozi katerega morata v okviru cirkulacije. Presežna vrednost \(\KPEb\) je prišla na svet šele v procesu produkcije. Na blagovni trg stopa torej prvič, in sicer v blagovni obliki. Ta je prva oblika, v kateri kroži, zaradi česar je tudi akt \(\KPEb\KPEcrta\KPEd\) njen prvi cirkulacijski akt ali njena prva metamorfoza, ki se mora dopolniti še z nasprotnim cirkulacijskim aktom ali z obratno metamorfozo \(\KPEd\KPEcrta\KPEb\).\footnote{To velja, kakor koli že ločimo kapitalsko vrednost od presežne. V 10.000 funtih preje tiči 1560 funtov = 78\ f.\ št.\ presežne vrednosti, v 1 funtu preje = 1 šiling pa tiči prav tako 2,496 unče = 1,872 penija presežne vrednosti.}

Drugače je s cirkulacijo, ki jo opravlja kapitalska vrednost \(\KPEB\) v istem cirkulacijskem aktu \(\KPEB'\KPEcrta\KPED'\), ki je zanjo cirkulacijski akt \(\KPEB\KPEcrta\KPED\), kjer je \(\KPEB \textrm{ = } \KPEP\), enako prvotno založenemu \(\KPED\). Svoj prvi cirkulacijski akt je pričela kot \(\KPED\), kot denarni kapital, in se vrača z aktom \(\KPEB\KPEcrta\KPED\) v isto obliko; šla je torej skozi obe nasprotni fazi cirkulacije (1) \(\KPED\KPEcrta\KPEB\) in (2) \(\KPEB\KPEcrta\KPED\) in je zdaj vnovič v obliki, v kateri lahko znova prične isti krožni tok. Kar je za presežno vrednost prva sprememba blagovne oblike v denarno obliko, je za kapitalsko vrednost vrnitev ali povratna sprememba v njeno prvotno denarno obliko.

Z \(\KPED\KPEcrta\KPEBrazcepDsPs\) se je denarni kapital spremenil v enakovredno vsoto blaga, \(\KPEDs\) in \(\KPEPs\). To blago nič več ne deluje kot blago, kot predmeti za prodajo. Njihova vrednost je sedaj \KPEstran v rokah kupca, kapitalista, kot vrednost njegovega produktivnega kapitala \(\KPEP\). V funkciji \(\KPEP\), v produktivni konsumpciji, pa se spremene v vrsto blaga, ki je snovno različna od produkcijskih sredstev, v prejo, v kateri se njihova vrednost ne le ohrani, ampak tudi poveča, od 422\ f.\ št.\ na 500\ f.\ št.\ S to realno metamorfozo se nadomesti v prvem stadiju \(\KPED\KPEcrta\KPEB\) trgu odtegnjeno blago s snovno in vrednostno drugačnim blagom, ki mora zdaj delovati kot blago, se spremeniti v denar in prodati. Zaradi tega je produkcijski proces samo prekinitev procesa cirkulacije vrednosti kapitala, od katerega je opravljena dotlej le prva faza \(\KPED\KPEcrta\KPEB\). Drugo in zaključno fazo \(\KPEB\KPEcrta\KPED\) opravi potem, ko se \(\KPEB\) snovno in vrednostno spremeni. Kolikor pa gre za vrednost kapitala samo po sebi, je doživela v produkcijskem procesu le spremembo svoje uporabne oblike. Obstajala je kot 422\ f.\ št.\ vrednosti v \(\KPEDs\) in \(\KPEPs\), sedaj obstaja kot 422\ f.\ št.\ vrednosti 8440 funtov preje. Če opazujemo torej samo obe fazi cirkulacijskega procesa vrednosti kapitala, ki si jo zamislimo ločeno od njene presežne vrednosti, potem vidimo, da izvede (1) \(\KPED\KPEcrta\KPEB\) in (2) \(\KPEB\KPEcrta\KPED\), pri čemer ima drugi \(\KPEB\) spremenjeno uporabno obliko, vendar pa isto vrednost kot prvi \(\KPEB\); torej \(\KPED\KPEcrta\KPEB\KPEcrta\KPED\), cirkulacijska oblika, ki z dvojno premestitvijo blaga v nasprotni smeri, spremembo denarja v blago in spremembo blaga v denar, nujno zahteva vrnitev vrednosti, ki je bila založena kot denar, v njeno denarno obliko: njeno vzvratno spremembo v denar.

Isti cirkulacijski akt \(\KPEB'\KPEcrta\KPED'\), ki pomeni za kapitalsko vrednost, založeno v denarju, drugo, zaključno metamorfozo, vrnitev v denarno obliko, pa je za presežno vrednost, ki jo vsebuje blagovni kapital in ki se vnovčuje obenem z njegovo spremembo v denarno obliko, prva metamorfoza, sprememba iz blagovne oblike v denarno, \(\KPEB\KPEcrta\KPED\), prva faza cirkulacije.

Tu moramo pripomniti dvoje. Prvič: sklepna vzvratna sprememba vrednosti kapitala v njeno prvotno denarno obliko je funkcija blagovnega kapitala. Drugič: ta funkcija vključuje prvo oblikovno spremembo presežne vrednosti iz njene prvotne blagovne oblike v denarno. Denarna oblika ima \KPEstran torej tu dvojno vlogo; na eni strani je vzvratna oblika prvotno v denarju založene vrednosti, torej vrnitev k vrednostni obliki, ki je pričela proces; po drugi strani pa je prva spremenjena oblika vrednosti, ki prvič vstopa v blagovni obliki v cirkulacijo. Če se proda blago, iz katerega sestoji blagovni kapital, tako kot tu predpostavljamo, po svoji vrednosti, se spremeni \(\KPEB \textrm{ + } \KPEb\) v enakovredni \(\KPED \textrm{ + } \KPEd\); v tej obliki \(\KPED \textrm{ + } \KPEd\) (422\ f.\ št.\ + 78\ f.\ št.\ = 500\ f.\ št.) je vnovčeni blagovni kapital sedaj v rokah kapitalista. Vrednost kapitala in presežna vrednost obstajata sedaj kot denar, torej v splošni ekvivalentni obliki.

Na koncu procesa se torej znajde vrednost kapitala spet v isti obliki, v kateri je vanj stopila, in ga zato lahko kot denarni kapital znova prične in opravi. Prav zaradi tega, ker je izhodiščna in sklepna oblika procesa denarni kapital (\(\KPED\)), imenujemo to obliko procesa krožnega toka krožni tok denarnega kapitala. Na koncu je spremenjena le velikost založene vrednosti, ne pa oblika.

\(\KPED \textrm{ + } \KPEd\) nista nič drugega kot denarni znesek določene velikosti, v našem primeru 500\ f.\ št. Kot rezultat krožnega toka kapitala, kot vnovčeni blagovni kapital, pa vsebuje ta denarni znesek vrednost kapitala in presežno vrednost; vendar ti dve zdaj nista več vraščeni druga v drugo kakor v preji, zdaj stojita druga poleg druge. Njuna vnovčitev je dala vsaki od njiju samostojno denarno obliko. 211/250 je vrednost kapitala, 422\ f.\ št., 39/250 pa presežna vrednost v višini 78\ f.\ št. Ta ločitev, ki jo je povzročilo vnovčenje blagovnega kapitala, nima samo formalne vsebine, o kateri bomo takoj govorili. V skladu s tem, ali se \(\KPEd\) pripoji v celoti ali delno ali pa se sploh ne pripoji \(\KPED\), torej po tem, ali deluje še naprej kot sestavni del založene vrednosti kapitala ali pa ne, postane pomembna v reprodukcijskem procesu kapitala. \(\KPEd\) in \(\KPED\) lahko izvedeta tudi popolnoma različni cirkulaciji.

V \(\KPED'\) se je vrnil kapital znovič v svojo prvotno obliko \(\KPED\), v svojo denarno obliko; vendar v takšni obliki, v kateri je realiziran kot kapital.

Tu imamo, prvič, količinsko razliko. Prej smo imeli \(\KPED\), 422\ f.\ št.; sedaj imamo \(\KPED'\), 500\ f.\ št., to razliko pa izražata \(\KPED\KPEpike\KPED'\), \KPEstran količinsko različna ekstrema krožnega toka, katerega gibanje označujejo samo pike~\(\KPEpike\) \(\KPED'\) je večji ko \(\KPED\), \(\KPED'\) minus \(\KPED \textrm{ = } \KPEPv\), presežna vrednost. -- Kot rezultat tega krožnega toka \(\KPED\KPEpike\KPED'\) pa obstaja sedaj samo še \(\KPED'\); to je produkt, v katerem je končan proces njegovega nastanka. \(\KPED'\) obstaja sedaj samostojno sam zase, neodvisno od gibanja, ki ga je ustvarilo. Gibanje je minilo, namesto njega je tu \(\KPED'\).

Kot \(\KPED \textrm{ + } \KPEd\), kot 422\ f.\ št.\ založenega kapitala plus njegov prirastek v višini 78\ f.\ št., pa je \(\KPED'\), 500\ f.\ št., hkrati kakovostno razmerje, čeprav obstaja to kakovostno razmerje s\^amo le kot razmerje delov istoimenske vsote, torej kot količinsko razmerje. Založeni kapital \(\KPED\), ki je sedaj spet v svoji prvotni obliki (422\ f.\ št.), obstaja sedaj kot realizirani kapital. Ni se le ohranil, ampak se je kot kapital tudi realiziral, ker se kot tak razlikuje od \(\KPEd\) (78\ f.\ št.), s katerim je v istem razmerju kot s \emph{svojim} prirastkom, \emph{svojim} plodom, kot z dodatkom, ki ga je sam ustvaril. Kot kapital se je realiziral zato, ker se je realiziral kot vrednost, ki je rodila vrednost. \(\KPED'\) obstaja kot kapitalsko razmerje: \(\KPED\) se ne pojavlja več zgolj kot denar, ampak nastopa izrečno kot denarni kapital, izražen kot vrednost, ki se je povečala in ki ima zato tudi sposobnost povečati svojo vrednost, ustvariti več vrednosti, kakor je ima sama. \(\KPED\) je postal kapital zaradi razmerja, v katerem je z drugim delom \(\KPED'\), s katerim je v razmerju kot s svojim produktom, kot z učinkom, ki ga povzroča, kot s posledico, katere vzrok je. Tako se pojavlja \(\KPED'\) kot vsota vrednosti, ki je sama v sebi diferencirana, ki se sama v sebi funkcionalno (pojmovno) razlikuje in ki izraža kapitalsko razmerje.

Toda to razmerje je izraženo le kot rezultat, ne glede na proces, iz katerega izvira.

Deli vrednosti se kot taki ne razlikujejo med seboj kakovostno, razen kolikor nastopajo kot vrednosti različnih predmetov, konkretnih stvari, torej v različnih uporabnih oblikah, zato kot vrednosti različnih blagovnih teles -- kar pa je razlika, ki ne izvira iz njih samih kot samo delov vrednosti. V denarju so zbrisane vse razlike med različnimi vrstami blaga, ker je denar vsem skupna oblika ekvivalentnosti. Denarni \KPEstran znesek 500\ f.\ št.\ sestoji iz istoimenskih elementov po 1\ f.\ št. Ker je v obstoju tega denarnega zneska zbrisana vsaka zveza z njegovim izvorom in ker je izginila vsaka sled o posebni razliki, ki so jo imeli različni sestavni deli kapitala v produkcijskem procesu, obstaja razlika samo še v pojmovni obliki glavnice (angleško principal), ki je enaka založenemu kapitalu v višini 422\ f.\ št., in presežni vsoti vrednosti v višini 78\ f.\ št. Recimo, da je na primer \(\KPED'\) = 110\ f.\ št., od česar 100 = \(\KPED\), glavnica, in 10 = \(\KPEPv\), presežna vrednost. Oba sestavna dela vsote 110\ f.\ št.\ sta popolnoma enakovrstna, tako da ju pojmovno ni mogoče ločiti. Katerih koli 10\ f.\ št.\ je vedno 1/11 skupne vsote 110\ f.\ št., bodisi da so 1/10 založene glavnice 100\ f.\ št.\ ali pa 10\ f.\ št., ki jo presegajo. Glavnico in prirastek, kapital in presežno vsoto, lahko izrazimo zato z ulomki skupnega zneska; v našem primeru tvori 10/11 glavnico ali kapital, 1/11 pa presežno vsoto. Znesek, v katerem se pojavlja tu na koncu svojega procesa realizirani kapital v svojem denarnem izrazu, izraža torej nediferencirano kapitalsko razmerje.

Seveda velja to tudi za \(\KPEB'\) (= \(\KPEB + \KPEb\)). Vendar s to razliko, da \(\KPEB'\), v katerem sta tudi \(\KPEB\) in \(\KPEb\) samo sorazmerna vrednostna dela iste homogene blagovne množine, opozarja na svoj izvor \(\KPEP\), katerega neposredni produkt je, medtem ko je v obliki \(\KPED'\), ki izvira neposredno iz cirkulacije, izginila neposredna zveza s \(\KPEP\).

Vsebinsko neopredeljena razlika med glavnico in prirastkom, ki jo vsebuje \(\KPED'\), kolikor izraža rezultat gibanja \(\KPED\KPEpike\KPED'\), izgine takoj, kakor hitro spet aktivno deluje kot denarni kapital, kolikor torej nasprotno ni fiksiran kot denarni izraz industrijskega kapitala, ki je povečal svojo vrednost. Krožni tok kapitala se ne more nikoli pričeti z \(\KPED'\) (čeprav deluje \(\KPED'\) sedaj kot \(\KPED\)), ampak le z \(\KPED\); torej nikoli kot izraz kapitalskega razmerja, ampak le kot založitvena oblika kapitalske vrednosti. Kakor hitro je 500\ f.\ št.\ ponovno založenih kot kapital, da bi ponovno povečali vrednost, so izhodiščna točka in ne povratna. Namesto kapitala 422\ f.\ št.\ se založi sedaj kapital 500\ f.\ št., več denarja kot prej, večja kapitalska vrednost. Razmerje med obema sestavnima deloma pa je \KPEstran odpadlo, prav kot če bi bil znesek 500\ f.\ št.\ namesto zneska 422\ f.\ št.\ deloval kot kapital že na začetku.

Denarni kapital ne opravlja nobene aktivne naloge, ko se predstavlja kot \(\KPED'\); njegovo predstavljanje kot \(\KPED'\) je nasprotno funkcija \(\KPEB'\). Že v enostavni cirkulaciji blaga, (1) \(\KPEB_1\KPEcrta\KPED\), (2) \(\KPED\KPEcrta\KPEB_2\), deluje \(\KPED\) aktivno šele v drugem aktu \(\KPED\KPEcrta\KPEB_2\); kot \(\KPED\) se predstavi le kot rezultat prvega akta, ki mu šele omogoči, da nastopi kot spremenjena oblika \(\KPEB_1\). V \(\KPED'\) vsebovano kapitalsko razmerje, odnos enega njegovih delov kot kapitalske vrednosti do drugega kot prirastka njegove vrednosti, pridobi seveda funkcionalni pomen, kolikor se pri nenehnem ponavljanju krožnega toka \(\KPED\KPEpike\KPED'\) cepi \(\KPED'\) na dve cirkulaciji, cirkulacijo kapitala in cirkulacijo presežne vrednosti, kolikor torej ne opravljata oba dela samo količinsko, ampak tudi kakovostno različni funkciji, \(\KPED\) drugo kakor \(\KPEd\). Gledana sama zase pa oblika \(\KPED\KPEpike\KPED'\) ne vključuje potrošnje kapitalista, ampak izrečno edinole povečanje lastne vrednosti in akumulacijo, kolikor se izrazi le-ta v periodičnem povečanju vedno znova zalaganega denarnega kapitala.

Čeprav je vsebinsko neopredeljena oblika kapitala, je \(\KPED' \textrm{ = } \KPED \textrm{ + } \KPEd\) vendar denarni kapital samo kot realizirani kapital, kot denar, ki je ustvaril denar. Od tega pa je potrebno ločiti delovanje denarnega kapitala v prvem stadiju \(\KPED\KPEcrta\KPEBrazcepDsPs\). V tem prvem stadiju kroži \(\KPED\) kot denar. Kot denarni kapital deluje le, ker samo v svojem denarnem stanju lahko opravlja denarno funkcijo, se spremeni v elemente \(\KPEP\), to je v \(\KPEDs\) in \(\KPEPs\), ki mu stoje nasproti kot blago. V tem cirkulacijskem aktu deluje le kot denar; ker pa je ta akt prvi stadij delujoče vrednosti kapitala, je zaradi specifične uporabne oblike kupovanega blaga \(\KPEDs\) in \(\KPEPs\) obenem tudi funkcija denarnega kapitala. \(\KPED'\), sestavljen iz kapitalske vrednosti \(\KPED\) in iz presežne vrednosti \(\KPEd\), katero je ustvarila, pa izraža povečano vrednost kapitala, namen in rezultat, funkcijo celotnega procesa krožnega toka kapitala. Da izraža ta rezultat v denarni obliki, kot vnovčeni denarni kapital, ne izvira iz tega, da je denarna oblika kapitala, da je \emph{denarni} kapital, \KPEstran ampak obratno iz tega, da je denarni \emph{kapital}, kapital v denarni obliki, da je začel kapital proces v tej obliki, da je bil v denarni obliki založen. Kakor smo videli, je vzvratna sprememba v denarno obliko funkcija blagovnega kapitala \(\KPEB'\), ne pa denarnega kapitala. Kar pa zadeva razliko med \(\KPED'\) in \(\KPED\), je ta (\(\KPEd\)) le denarna oblika \(\KPEb\), prirastka \(\KPEB\); \(\KPED'\) je \(\KPED \textrm{ + } \KPEd\) samo zato, ker je bil \(\KPEB' \textrm{ = } \KPEB \textrm{ + } \KPEb\). \(\KPEB'\) vsebuje in izraža torej to razliko in razmerje kapitalske vrednosti do presežne vrednosti, ki jo je ustvarila, preden sta se obe spremenili v \(\KPED'\), v vsoto denarja, v kateri nastopata oba dela vrednosti drug proti drugemu kot samostojna in zaradi česar ju je mogoče tudi uporabiti za samostojne in med seboj različne funkcije.

\(\KPED'\) je zgolj rezultat realizacije \(\KPEB'\). Obadva, tako \(\KPEB'\) kakor tudi \(\KPED'\), sta le različni obliki, blagovna in denarna oblika, povečane kapitalske vrednosti: obema je skupno, da sta povečana kapitalska vrednost. Obedve sta realizirani kapital, ker obstaja kapitalska vrednost kot taka tu skupno s presežno vrednostjo kot od nje različnim plodom, ki ga je ustvarila, čeprav se izraža to razmerje le v vsebinsko neopredeljeni obliki sorazmerja dveh delov ene denarne vsote ali ene blagovne vrednosti. Kot izraza kapitala v razmerju do presežne vrednosti, ki jo je ustvaril, in v razliko od nje, torej kot izraza povečane vrednosti, pa sta \(\KPED'\) in \(\KPEB'\) eno in isto in izražata isto, le v drugačni obliki; ne razlikujeta se kot denarni in blagovni kapital, ampak kot denar in blago. Kolikor pomenita povečano vrednost, kapital, ki je bil aktiven kapital, izražata le rezultat delovanja produktivnega kapitala, edine funkcije, v kateri kapitalska vrednost producira vrednost. Skupno jima je, da sta oba, denarni kakor tudi blagovni kapital, načina eksistence kapitala. Eden je kapital v denarni, drugi v blagovni obliki. Posebne funkcije, po katerih se razlikujeta, ne morejo biti torej nič drugega kot razlike med funkcijo denarja in funkcijo blaga. Kot neposredni produkt kapitalističnega produkcijskega procesa opozarja blagovni kapital na ta svoj izvor in je zaradi tega po svoji obliki bolj racionalen, ni tako neracionalen kakor denarni kapital, v katerem je izginila vsaka sled \KPEstran tega procesa, tako kot v denarju sploh zgine vsaka posebna uporabna oblika blaga. Samo tam, kjer deluje \(\KPED'\) sam kot blagovni kapital, kjer je neposredni produkt produkcijskega procesa, ne pa spremenjena oblika tega procesa, izgine zato njegova nenavadna oblika -- se pravi v reprodukciji denarne tvarine same. Za produkcijo zlata bi bil na primer obrazec: \(\KPED\KPEcrta\KPEBrazcepDsPs\KPEpike\KPEP\KPEpike\KPED' (\KPED \textrm{ + } \KPEd)\), kjer pomeni \(\KPED'\) blagovni produkt, ker daje \(\KPEP\) več zlata, kot je bilo založenega za njegove produkcijske elemente v prvem \(\KPED\), denarnem kapitalu. Tu izgine torej neracionalnost izraza \(\KPED\KPEpike\KPED' (\KPED \textrm{ + } \KPEd)\), v katerem nastopa en del denarne vsote kot mati drugega dela iste denarne vsote.

\section{Celotni krožni tok}

Videli smo, da se cirkulacijski proces, ko konča svoj prvi stadij \(\KPED\KPEcrta\KPEBrazcepDsPs\), prekine s \(\KPEP\), v katerem se porabijo na trgu nakupljene vrste blaga \(\KPEDs\) in \(\KPEPs\) kot tvarne in vrednostne sestavine produktivnega kapitala; produkt te porabe je tvarno in vrednostno spremenjeno novo blago \(\KPEB'\). Prekinjeni cirkulacijski proces \(\KPED\KPEcrta\KPEB\) se mora dopolniti z \(\KPEB\KPEcrta\KPED\). Kot nosilec te druge in zaključne faze cirkulacije pa nastopa \(\KPEB'\), blago, ki je tvarno in vrednostno različno od prvega \(\KPEB\). Cirkulacijsko zaporedje je torej (1) \(\KPED\KPEcrta\KPEB_1\); (2) \(\KPEB'_2\KPEcrta\KPED'\), kjer je med prekinitvijo, ki jo je povzročilo delovanje \(\KPEP\), med produkcijo \(\KPEB'\) iz elementov \(\KPEB\), eksistenčnih oblik produktivnega kapitala \(\KPEP\), v drugi fazi zasedlo mesto prvega blaga \(\KPEB_1\) drugo blago večje vrednosti in drugačne uporabne oblike, \(\KPEB_2\). Prva pojavna oblika, v kateri smo naleteli na kapital (I.\ knjiga, 4.\ poglavje, [str.\ 167--177]), \(\KPED\KPEcrta\KPEB\KPEcrta\KPED'\) (razčlenjena: (1) \(\KPED\KPEcrta\KPEB_1\); (2) \(\KPEB_1\KPEcrta\KPED'\)), pa pokaže isto blago dvakrat. Blago, v katero se spremeni v prvi fazi denar in katero se v drugi fazi spremeni v več denarja, je obakrat isto. Kljub tej bistveni različnosti je obema cirkulacijama skupno to, da se spreminja v \KPEstran njuni prvi fazi denar v blago in v drugi blago v denar, da torej priteka v prvi fazi izdani denar v drugi zopet nazaj. Na eni strani je obema skupen ta povratni tok denarja na njegovo izhodišče, po drugi pa tudi presežek vračajočega se denarja nad založenim. V tej meri vsebuje splošni obrazec \(\KPED\KPEcrta\KPEB\KPEcrta\KPED'\) tudi \(\KPED\KPEcrta\KPEB\KPEpike\KPEB'\KPEcrta\KPED'\).

Nadalje se tu izkaže, da si v obeh metamorfozah \(\KPED\KPEcrta\KPEB\) in \(\KPEB'\KPEcrta\KPED'\), ki pripadata cirkulaciji, stojita nasproti in se medsebojno zamenjujeta vsakokrat enako veliki in istočasno obstoječi vrednosti. Do spremembe vrednosti pride izključno le v teku metamorfoze \(\KPEP\), v produkcijskem procesu, ki je tako realna metamorfoza kapitala v nasprotju s čisto formalnima metamorfozama cirkulacije.

Poglejmo sedaj celotno gibanje \(\KPED\KPEcrta\KPEB\KPEpike\KPEP\KPEpike\KPEB'\KPEcrta\KPED'\) ali njegovo razvito obliko \(\KPED\KPEcrta\KPEBrazcepDsPs\KPEpike\KPEP\KPEpike\KPEB'(\KPEB \textrm{ + } \KPEb)\KPEcrta\KPED'(\KPED \textrm{ + } \KPEd)\). Kapital nastopa tu kot vrednost, ki gre skozi vrsto povezanih, medsebojno pogojenih sprememb, skozi vrsto metamorfoz, ki tvorijo prav toliko faz ali stadijev celotnega procesa. Dve od teh faz pripadata področju cirkulacije, ena področju produkcije. V vsaki od teh faz ima vrednost kapitala drugačno obliko, kateri ustreza drugačna, posebna funkcija. Med tem spreminjanjem se založena vrednost ne le ohrani, ampak raste, poveča svojo velikost. V sklepnem stadiju se vrne nazadnje v isto obliko, v kateri se je pojavila na začetku celotnega procesa. Ta celotni proces je zaradi tega krožni proces.

Obe obliki, ki ju privzema kapitalska vrednost v teku stadijev svoje cirkulacije, sta oblika \emph{denarnega} in oblika \emph{blagovnega kapitala}; njena oblika, ki pripada produkcijskemu procesu, je oblika \emph{produktivnega kapitala}. Kapital, ki privzema v teku svojega celotnega krožnega toka te oblike in jih zopet odlaga, v vsaki pa opravi njej ustrezno funkcijo, je \emph{industrijski kapital} -- industrijski v tem pomenu, da vključuje vsako na kapitalistični način vodeno produkcijsko panogo.

Denarni kapital, blagovni kapital in produktivni kapital tu torej ne naznačujejo samostojnih vrst kapitala, katerih funkcije \KPEstran bi tvorile vsebino prav tako samostojnih in medsebojno ločenih poslovnih panog. Tu naznačujejo samo posebne oblike funkcije industrijskega kapitala, ki jih vse tri privzema drugo za drugo.

Krožni tok kapitala poteka normalno samo, dokler prehajajo različne njegove faze brez zastoja druga v drugo. Če se zaustavi kapital v prvi fazi \(\KPED\KPEcrta\KPEB\), otrpne denarni kapital v zaklad; če se zaustavi v produkcijski fazi, obleže na eni strani produkcijska sredstva brez dela, na drugi pa ostane delovna sila brezposelna; če se zaustavi v zadnji fazi \(\KPEB'\KPEcrta\KPED'\), zavre nakopičeno blago, ki ga ni mogoče prodati, tok cirkulacije.

Na drugi strani ustreza naravi stvari, da zahteva sam krožni tok, da se kapital zadržuje določen čas v posameznih njegovih odsekih. V vsaki od svojih faz je industrijski kapital vezan na določeno obliko, obliko denarnega, produktivnega, blagovnega kapitala. Šele ko opravi nalogo, ki ustreza njegovi vsakokratni obliki, pridobi obliko, v kateri lahko vstopi v novo fazo spreminjanja. Da bi to prikazali čim jasneje, smo v našem primeru predpostavili, da je vrednost kapitala v produkcijskem stadiju ustvarjene množine blaga enaka skupni vsoti prvotno kot denar založene vrednosti, z drugimi besedami, da vstopi celotna kot denar založena kapitalska vrednost naenkrat iz enega stadija v vsakokratnega naslednjega. Videli pa smo (I.\ knjiga, 6.\ poglavje), da služi del konstantnega kapitala, prava delovna sredstva (na primer stroji), v večjem ali manjšem številu ponovitev vedno znova istemu produkcijskemu procesu, da oddajajo torej tudi svojo vrednost na produkt le kos za kosom. Pozneje se bo pokazalo, v kakšni meri spreminja ta okoliščina proces krožnega toka. Tu zadošča naslednje: v našem primeru je vsebovala vrednost produktivnega kapitala v znesku 422\ f.\ št.\ samo v povprečju obračunano obrabo tovarniških zgradb, strojev itd., torej le tisti del vrednosti, ki ga pri pretvorbi 10.600 funtov bombaža v 10.000 funtov preje prenašajo na le-to, na produkt 60-urnega tedenskega procesa predenja. Zaradi tega so produkcijska sredstva, v katera se je pretvoril založeni konstantni kapital v znesku 372\ f.\ št., delovna \KPEstran sredstva, zgradbe, stroji itd., tudi nastopala tako, kot da bi bila na trgu le izposojena na tedensko obročno odplačevanje. Vendar to ne spremeni na stvari prav ničesar. V tednu dni producirano količino preje 10.000 funtov je treba samo pomnožiti s številom na določeno vrsto let preračunanih tednov, pa se prenese nanjo celotna vrednost kupljenih in v tem času porabljenih delovnih sredstev. Potem je jasno, da se mora založeni denarni kapital najprej spremeniti v ta sredstva, stopiti najprej iz prvega stadija \(\KPED\KPEcrta\KPEB\), preden lahko deluje kot produktivni kapital \(\KPEP\). Prav tako jasno je v našem primeru, da vsota kapitalske vrednosti 422\ f.\ št., ki se je utelesila v preji med produkcijskim procesom, ne more vstopiti kot sestavni del vrednosti 10.000 funtov preje v cirkulacijsko bazo \(\KPEB'\KPEcrta\KPED'\), preden ni gotova. Preje ni mogoče prodati, dokler ni spredena.

V splošnem obrazcu gledamo na produkt \(\KPEP\) kot na materialno stvar, ki je različna od elementov produktivnega kapitala, kot na predmet, ki ima od produkcijskega procesa ločen obstoj, ki ima uporabno obliko, ki je različna od uporabne oblike produkcijskih elementov. To, da nastopa rezultat produkcijskega procesa kot stvar, se zgodi v vsakem primeru, celo tam, kjer vstopi del produkta kot element vnovič v obnovljeno produkcijo. Tako služi žito kot seme za svojo lastno produkcijo; produkt pa sestoji samo iz žita, ima torej podobo, ki je različna od skupno z njim uporabljanih elementov, delovne sile, orodij, gnojil. Obstajajo pa samostojne industrijske panoge, v katerih produkt produkcijskega procesa ni predmetni produkt, ni blago. Ekonomsko pomembna od teh je le industrija komunikacij, najsi je prava prevozna industrija za blago in ljudi ali pa samo prenašanje sporočil, pisem, brzojavov itd.

A.~Čuprov\footnote{\begin{otherlanguage}{russian}\emph{А. Чупров}, ">Желђзнодорожное хозяйстЬо"<\end{otherlanguage}, Moskva 1875, str.\ 69, 70.} pravi o tem: ">Tovarnar lahko predmet najprej producira in potem išče potrošnika zanj"< (potem ko produkcijski proces izloči svoj produkt, ker je dokončan, preide \KPEstran v cirkulacijo kot blago, ki je od procesa ločeno). ">Produkcija in konsumpcija sta tako dva prostorsko in časovno ločena akta. V transportni industriji, ki ne ustvarja nobenih novih produktov, ampak samo premešča ljudi in stvari, ta dva akta sovpadata; storitve"< (premestitve) ">je treba porabiti v istem trenutku, v katerem so producirane. Zaradi tega sega področje, na katerem lahko iščejo železnice svoje odjemalce, največ 50 vrst (53 km) na vsako stran."<

Rezultat je -- bodisi da se prevažajo ljudje, bodisi da se prevaža blago -- sprememba njihovega krajevnega bivanja, na primer, da je preja sedaj v Indiji namesto v Angliji, kjer so jo izdelali.

To, kar prodaja transportna industrija, pa je premestitev sama. Ustvarjeni koristni učinek je neločljivo povezan s transportnim procesom, se pravi s produkcijskim procesom transportne industrije. Ljudje in blago potujejo s prevoznim sredstvom, njegovo potovanje, njegovo premeščanje pa je ravno produkcijski proces, katerega povzroča. Koristni učinek je mogoče konsumirati le med produkcijskim procesom; ne obstoji kot uporabna stvar, različna od tega procesa, ki nastopa šele po svoji produkciji kot trgovinski predmet, kroži kot blago. Menjalna vrednost tega koristnega učinka pa je določena tako kakor vrednost vsakega drugega blaga z vrednostjo v njem porabljenih produkcijskih elementov (delovne sile in produkcijskih sredstev) plus presežno vrednostjo, ki jo je ustvarilo presežno delo v transportni industriji zaposlenih delavcev. Tudi v razmerju do svoje konsumpcije se obnaša ta koristni učinek tako kakor drugo blago. Če se konsumira individualno, izgine njegova vrednost s konsumpcijo; če se porabi produktivno, tako da je sam produkcijski stadij blaga na transportu, se prenese njegova vrednost kot dodatna vrednost na blago. Obrazec za transportno industrijo bi bil torej \(\KPED \KPEcrta \KPEBrazcepDsPs \KPEpike \KPEP \KPEcrta \KPED' \), ker plačamo in konsumiramo sam produkcijski proces, ne pa od njega ločljivi produkt. Ima torej skoraj natančno isto obliko, kot jo ima obrazec za produkcijo žlahtnih kovin, samo da je \(\KPED'\) tu spremenjena \KPEstran oblika v produkcijskem procesu ustvarjenega koristnega učinka, ne pa naturalna oblika v tem procesu ustvarjenega in iz njega izločenega zlata ali srebra.

Industrijski kapital je edini način obstoja kapitala, v katerem ni naloga kapitala samo prilaščanje presežne vrednosti oziroma presežnega produkta, ampak hkrati tudi njegovo ustvarjanje. Je torej pogoj kapitalističnega značaja produkcije; njegov obstoj vključuje obstoj razrednega nasprotja med kapitalisti in mezdnimi delavci. V tistem obsegu, v katerem industrijski kapital osvaja družbeno produkcijo, prevrača tehniko in družbeno organizacijo delovnega procesa, s tem pa ekonomsko-zgodovinski tip družbe. Druge vrste kapitala, ki so se pojavile pred njim v preteklih ali propadajočih družbenih produkcijskih ureditvah, se mu ne le podredijo in prilagodijo z mehanizmom svojih funkcij, ampak delujejo samo še na njegovi podlagi, živijo in umrjejo, stoje in padejo s to svojo podlago. Denarni in blagovni kapital sta, kolikor nastopata s svojimi funkcijami kot nosilca lastnih poslovnih panog poleg industrijskega kapitala, le še na podlagi družbene delitve dela osamosvojena in razvita načina obstoja različnih funkcijskih oblik, ki jih industrijski kapital v sferi cirkulacije zdaj privzema, zdaj odlaga.

Krožni tok \(\KPED\KPEpike\KPED'\) se po eni strani prepleta s splošno blagovno cirkulacijo, izhaja iz nje in se vrača vanjo ter tvori njen del. Po drugi strani tvori za posameznega kapitalista lastno samostojno gibanje kapitalske vrednosti, gibanje, ki poteka deloma znotraj splošne blagovne cirkulacije, deloma zunaj nje, ki pa vedno ohranja svoj samostojni značaj. Prvič tako, da imata obe njeni v sferi cirkulacije potekajoči fazi \(\KPED\KPEcrta\KPEB\) in \(\KPEB'\KPEcrta\KPED'\), kot fazi gibanja kapitala, funkcionalno določena značaja; v \(\KPED\KPEcrta\KPEB\) je \(\KPEB\) tvarno opredeljen kot delovna sila in produkcijska sredstva; v \(\KPEB'\KPEcrta\KPED'\) se realizira kapitalska vrednost plus presežna vrednost. Drugič, produkcijski proces vključuje \(\KPEP\), produktivno konsumpcijo. Tretjič, vrnitev denarja na njegovo izhodišče povzroči, da je gibanje \(\KPED\KPEpike\KPED'\) samo v sebi zaključeno krožno gibanje.

Po \KPEstran eni strani je torej vsak individualni kapital z obema svojima cirkulacijskima polovicama \(\KPED\KPEcrta\KPEB\) in \(\KPEB'\KPEcrta\KPED'\) gibalo splošne blagovne cirkulacije, v kateri deluje ali je z njo povezan bodisi kot denar, bodisi kot blago, in tvori tako sam člen splošnega zaporedja metamorfoz blagovnega sveta. Po drugi strani pa opravi v okviru splošne cirkulacije svoj lastni samostojni krožni tok, v katerem je produkcijsko področje prehodni stadij in v katerem se povrne k svojemu izhodišču v isti obliki, v kateri ga je zapustil. V mejah lastnega krožnega toka, ki vključuje njegovo realno metamorfozo v produkcijskem procesu, spremeni hkrati velikost svoje vrednosti. Vrne se ne le kot denarna vrednost, ampak kot povečana, narastla denarna vrednost.

Če si ogledamo končno \(\KPED \KPEcrta \KPEB \KPEpike \KPEP \KPEpike \KPEB' \KPEcrta \KPED' \) kot posebno obliko procesa krožnega toka kapitala poleg drugih oblik, ki jih bomo proučili pozneje, se odlikuje po temle:

1.\ Pojavlja se kot \emph{krožni tok denarnega kapitala}, ker tvori industrijski kapital v svoji denarni obliki, kot denarni kapital, izhodiščno in povratno točko njegovega celotnega procesa. Sam obrazec kaže, da se tu denar ne izdaja kot denar, ampak samo zalaga, da je torej le denarna oblika kapitala, denarni kapital. Nadalje kaže, da je odločilni namen gibanja menjalna, ne pa uporabna vrednost. Ravno zato, ker je denarna podoba vrednosti njegova samostojna, otipljiva pojavna oblika, kaže cirkulacijska oblika \(\KPED\KPEpike\KPED'\), katere začetna in sklepna točka je dejanski denar, na najbolj oprijemljivi način, da je pridobivanje denarja gonilni nagib kapitalistične produkcije. Produkcijski proces je pri tem le neizogibni srednji člen, nujno zlo zaradi pridobivanja denarja. Vse dežele s kapitalističnim produkcijskim načinom zajame zato od časa do časa vrtoglavica, v kateri hočejo priti do denarja brez posredovanja produkcijskega procesa.

2.\ Produkcijski stadij, delovanje \(\KPEP\), prekinja v tem krožnem toku dve fazi cirkulacije, \(\KPED\KPEcrta\KPEB\KPEpike\KPEB'\KPEcrta\KPED'\), ki pa posredujeta le enostavno cirkulacijo \(\KPED\KPEcrta\KPEB\KPEcrta\KPED'\). V obliki procesa krožnega toka se pokaže produkcijski proces formalno in izrecno takšen, kakršen v kapitalističnem produkcijskem načinu \KPEstran je, zgolj sredstvo za povečanje založene vrednosti; obogatitev sama po sebi se pokaže torej kot absolutni namen produkcije.

3.\ Ker se pričenja zaporedje faz z \(\KPED\KPEcrta\KPEB\), je drugi člen cirkulacije \(\KPEB'\KPEcrta\KPED'\); izhodiščna točka je torej \(\KPED\), denarni kapital, ki se mora povečati, sklepna točka \(\KPED'\), povečani denarni kapital \(\KPED \textrm{ + } \KPEd\), kjer nastopa \(\KPED\) kot realizirani kapital poleg svojega poganjka \(\KPEd\). To razlikuje krožni tok \(\KPED\) od obeh drugih krožnih tokov \(\KPEP\) in \(\KPEB'\), in sicer na dvojen način. Na eni strani z denarno obliko obeh ekstremov; denar pa je samostojna otipljiva eksistenčna oblika vrednosti, vrednost produkta v njegovi samostojni obliki vrednosti, v kateri je zbrisana vsaka sled uporabne vrednosti blaga. Na drugi strani se oblika \(\KPEP\KPEpike\KPEP\) ne spremeni nujno v \(\KPEP\KPEpike\KPEP'\) (\(\KPEP \textrm{ + } \KPEp\)), v obliki \(\KPEB'\KPEpike\KPEB'\) pa sploh ni vidna nobena razlika vrednosti med obema skrajnostma. -- Za obrazec \(\KPED\KPEpike\KPED'\) je po eni strani torej značilno, da tvori kapitalska vrednost izhodiščno, povečana kapitalska vrednost pa povratno točko, tako da je založitev kapitalske vrednosti sredstvo, povečana kapitalska vrednost pa namen celotne operacije; na drugi strani pa je zanj značilno, da je to razmerje izraženo v denarni obliki, v samostojni obliki vrednosti, denarni kapital torej kot denar, ki rodi denar. Produciranje presežne vrednosti s pomočjo vrednosti ni prikazano le kot alfa in omega procesa, ampak izrecno v bleščeči denarni obliki.

4.\ Ker je \(\KPED'\), realizirani denarni kapital kot rezultat \(\KPEB'\KPEcrta\KPED'\), dopolnilne in sklepne faze \(\KPED\KPEcrta\KPEB\), popolnoma iste oblike, v kateri je pričel svoj prvi krožni tok, lahko, brž ko pride iz njega, znovič prične isti krožni tok kot povečani denarni kapital: \(\KPED' \textrm{ = } \KPED \textrm{ + } \KPEd\); vsaj v obliki \(\KPED\KPEpike\KPED'\) pa ni povedano, da se pri ponovitvi krožnega toka loči cirkulacija \(\KPEd\) od cirkulacije \(\KPED\). Če ga gledamo v njegovi enkratni podobi, formalno, izraža krožni tok denarnega kapitala le proces večanja vrednosti in akumulacije. Konsumpcijo izraža v njem \(\KPED\KPEcrta\KPEBrazcepDsPs\) le kot produktivno konsumpcijo, samo ta je vključena v ta krožni tok individualnega kapitala. \(\KPED\KPEcrta\KPEDs\) \KPEstran je \(\KPEDs\KPEcrta\KPED\) ali \(\KPEB\KPEcrta\KPED\) z delavčevega vidika; je torej prva faza cirkulacije, ki posreduje njegovo individualno konsumpcijo: \(\KPEDs\KPEcrta\KPED\KPEcrta\KPEB\) (življenjske potrebščine). Druga faza \(\KPED\KPEcrta\KPEB\) ne spada več v krožni tok individualnega kapitala; jo je pa uvedel, predpostavil, ker mora delavec predvsem živeti, vzdrževati se torej z individualno konsumpcijo, da bi bil lahko vedno na trgu kot tvarina, ki je kapitalistu na voljo za izkoriščanje. Toda ta konsumpcija sama je tu predpostavljena le kot pogoj kapitalove produktivne konsumpcije delovne sile, torej tudi le toliko, kolikor se delavec s svojo individualno konsumpcijo ohranja in reproducira kot delovna sila. \(\KPEPs\), pravo, resnično blago, ki gre v krožni tok, pa je le hranivo produktivne konsumpcije. Akt \(\KPEDs\KPEcrta\KPED\) posreduje delavčevo individualno konsumpcijo, spremembo življenjskih potrebščin v njegovo meso in kri. Seveda je potreben tudi kapitalist, torej mora tudi živeti in konsumirati, da lahko deluje kot kapitalist. Za to bi bilo dejansko dovolj, če bi trošil toliko, kolikor delavec, več pa ta oblika cirkulacijskega procesa zato ne predpostavlja. Formalno ne izraža niti tega, ker se končuje obrazec z \(\KPED'\), torej z rezultatom, ki takoj lahko znovič deluje kot povečani denarni kapital.

\(\KPEB'\KPEcrta\KPED'\) vsebuje prodajo \(\KPEB'\) neposredno; \(\KPEB'\KPEcrta\KPED'\), prodaja z ene strani, pa je \(\KPED\KPEcrta\KPEB\), nakup z druge; vendar se blago dokončno kupi le zaradi svoje uporabne vrednosti, da bi prišlo (vmesne prodaje odmislimo) v proces konsumpcije, naj bo že individualen ali produktiven, odvisno od narave kupljenega predmeta. Ta konsumpcija pa ne spada v krožni tok tistega individualnega kapitala, ki je ustvaril \(\KPEB'\); nasprotno, krožni tok ta produkt kot blago za prodajo izloča. \(\KPEB'\) je izrecno določen za tujo konsumpcijo. Zaradi tega najdemo pri tolmačih merkantilističnega sistema (ki mu je osnova obrazec \(\KPED\KPEcrta\KPEB\KPEpike\KPEP\KPEpike\KPEB'\KPEcrta\KPED'\)) zelo razvlečene pridige o tem, da mora konsumirati posamezni kapitalist samo kot delavec, enako kakor mora kapitalistična dežela prepustiti drugim bolj zabitim deželam potrošnjo svojega blaga in proces potrošnje sploh, medtem ko si mora postaviti za svojo življenjsko nalogo produktivno potrošnjo. \KPEstran Te pridige pogosto po obliki in vsebini spominjajo na podobne asketske opomine cerkvenih očetov.
\medskip
\hrule
\medskip

Proces kroženja kapitala je torej enotnost cirkulacije in produkcije, vključuje obedve. Kolikor sta obe fazi \(\KPED\KPEcrta\KPEB\), \(\KPEB'\KPEcrta\KPED'\) cirkulacijska postopka, je cirkulacija kapitala del splošne cirkulacije blaga. Kot funkcionalno določena odseka, stadija v krožnem toku kapitala, ki ne pripada le cirkulacijskemu področju, ampak tudi produkcijskemu, opravlja kapital znotraj splošne blagovne cirkulacije svoj lastni krožni tok. Splošna blagovna cirkulacija mu v prvem stadiju služi, da privzame obliko, v kateri lahko deluje kot produktivni kapital; v drugi, da se znebi blagovne oblike, v kateri ne more obnoviti svojega krožnega toka; hkrati pa, da mu omogoči, da loči svoj lastni kapitalski krožni tok od cirkulacije presežne vrednosti, ki mu je prirastla.

Krožni tok denarnega kapitala je torej najbolj enostranska, zato najbolj prepričljiva in najbolj značilna pojavna oblika krožnega toka industrijskega kapitala, katerega cilj in gonilni nagib, večanje vrednosti, pridobivanje denarja in akumulacija, bije v oči (kupovati zaradi dražje prodaje). S tem, da je prva faza \(\KPED\KPEcrta\KPEB\), se pokaže tudi, da izvirajo sestavni deli produktivnega kapitala z blagovnega trga, kakor tudi, da na splošno brez cirkulacije, trgovine, ni kapitalističnega produkcijskega procesa. Krožni tok denarnega kapitala ni le produkcija blaga; mogoč je le s cirkulacijo, jo predpostavlja. To vidimo že po tem, ker nastopa cirkulaciji pripadajoča oblika \(\KPED\) kot prva in čista oblika založene kapitalske vrednosti, ne pa tudi v drugih dveh oblikah krožnega toka.

Krožni tok denarnega kapitala je vedno splošni izraz industrijskega kapitala, če le vključuje povečevanje založene vrednosti. V \(\KPEP\KPEpike\KPEP\) se kaže denarni izraz kapitala le kot cena produkcijskih elementov, torej le kot vrednost, izražena v računskem denarju, in se v tej obliki izraža v knjigovodstvu.

\(\KPED\KPEcrta\KPED'\) postane posebna oblika krožnega toka industrijskega kapitala, kadar se na novo nastopajoči kapital najprej \KPEstran založi kot denar in potem v isti obliki odtegne, bodisi pri prehodu iz ene poslovne panoge v drugo, bodisi ko se industrijski kapital umakne iz posla. To terja, da deluje presežna vrednost, v začetku založena v denarni obliki, kot kapital, najbolj jasno pa se to pokaže, če deluje ta presežna vrednost v kakem drugem podjetju, ne v tistem, iz katerega izvira. \(\KPED\KPEcrta\KPED'\) je lahko prvi krožni tok kakega kapitala; lahko je zadnji; lahko se šteje za obliko celotnega družbenega kapitala; \(\KPED\KPEcrta\KPED'\) je oblika kapitala, ki se naloži na novo, bodisi kot v denarni obliki na novo akumulirani kapital ali pa kot stari kapital, ki se v celoti spremeni v denar, da bi prešel iz ene produkcijske panoge v drugo.

Kot oblika, ki jo vedno vsebujejo vsi krožni toki, opravlja denarni kapital ta krožni tok ravno za tisti del kapitala, ki ustvarja presežno vrednost, za variabilni kapital. Normalna oblika, v kateri se zalaga mezda, je plačilo v denarju; ta proces se mora v krajših obdobjih stalno obnavljati, ker živi delavec iz rok v usta. Nasproti delavcu mora zato nastopati kapitalist trajno kot denarni kapitalist, njegov kapital pa kot denarni kapital. Tu ne more priti, kakor pri nakupu produkcijskih sredstev in prodaji produktivnega blaga, do neposredne ali posredne izravnave (tako da nastopa večja količina denarnega kapitala dejansko le v obliki blaga, denar le v obliki računskega denarja, v gotovini pa končno le za izravnavo računov). Po drugi strani zapravi kapitalist del iz variabilnega kapitala izhajajoče presežne vrednosti za svojo zasebno potrošnjo, ki sodi v trgovino na drobno in ki se ne glede na vse ovinke plačuje v gotovini, v denarni obliki presežne vrednosti. Ali je ta del presežne vrednosti velik ali majhen, ni prav nič pomembno. Variabilni kapital se neprenehoma pojavlja na novo kot denarni kapital (\(\KPED\KPEcrta\KPEB\)), naložen v mezdo, \(\KPEd\) pa kot presežna vrednost, ki se izdaja za zadovoljitev zasebnih potreb kapitalista. Torej \(\KPED\) kot založena variabilna vrednost kapitala in \(\KPEd\) kot njen prirastek, oba pa ohranjena nujno v denarni obliki, da bi se v njej lahko potrošila.

Obrazec \KPEstran \( \KPED \KPEcrta \KPEB \KPEpike \KPEP \KPEpike \KPEB' \KPEcrta \KPED' \) z rezultatom \( \KPED' \textrm{ = } \KPED \textrm{ + } \KPEd \) vključuje v svoji obliki prevaro, ima varljiv značaj, ki izvira iz tega, ker nastopata založena in povečana vrednost v svoji ekvivalentni obliki, v denarju. Poudarek ni na večanju vrednosti, ampak na \emph{denarni obliki} tega procesa, na tem, da se potegne na koncu v denarni obliki iz cirkulacije več vrednosti, kakor je je bilo v začetku vanjo založene, torej na povečanju količine zlata in srebra, ki pripada kapitalistu. Tako imenovani monetarni sistem je zgolj izraz vsebinsko neopredeljene oblike \( \KPED \KPEcrta \KPEB \KPEcrta \KPED' \), gibanja, ki poteka izključno le v cirkulaciji in ki zato lahko pojasni oba akta, (1) \( \KPED \KPEcrta \KPEB \) in (2) \( \KPEB \KPEcrta \KPED' \), le tako, da se proda \( \KPEB \) v drugem aktu nad svojo vrednostjo, kar pomeni, da potegne iz cirkulacije več denarja, kot ga je njegov nakup vrgel vanjo. Nasprotno temu pa je \( \KPED \KPEcrta \KPEB \KPEpike \KPEP \KPEpike \KPEB' \KPEcrta \KPED' \), vzeta kot edina oblika, podlaga razvitejšega merkantilnega sistema, v katerem je nujen element ne le cirkulacija blaga, ampak tudi produkcija blaga.

Varljivi značaj oblike \( \KPED \KPEcrta \KPEB \KPEpike \KPEP \KPEpike \KPEB' \KPEcrta \KPED' \) in njej ustrezajoča varljiva razlaga se pokaže takoj, kakor hitro jo obravnavamo kot enkratno, ne kot tekočo, nenehno se obnavljajočo; kakor hitro je torej ne štejemo za eno od oblik krožnega toka, ampak za njegovo edino obliko. Vendar pa opozarja sama na druge oblike.

Prvič. Ves ta krožni tok sam predpostavlja, da ima produkcijski proces kapitalistični značaj in da je zato njegova osnova ta produkcijski proces obenem s posebno družbeno ureditvijo, ki izhaja iz njega. \( \KPED \KPEcrta \KPEB \textrm{ = } \KPED \KPEcrta \KPEBrazcepDsPs \); \( \KPED \KPEcrta \KPEDs \) pa predpostavlja mezdnega delavca in s tem produkcijska sredstva kot del produktivnega kapitala, zaradi tega predpostavlja proces dela in večanja vrednosti, produkcijski proces že kot funkcijo kapitala.

Drugič. Če se \( \KPED \KPEpike \KPED' \) ponovi, je vrnitev k denarni obliki prav tako prehodna kakor denarna oblika v prvem stadiju. \( \KPED \KPEcrta \KPEB \) izgine zato, da napravi prostor \( \KPEP \). Nepretrgano ponovno zalaganje v denarju kakor tudi njegovo nepretrgano \KPEstran vračanje v denarni obliki sta tudi sama samo prehodna člena v krožnem toku.

Tretjič:
\[
\begin{tikzpicture}[baseline=(current bounding box.center), remember picture]

    \matrix[matrix of math nodes, inner sep=0pt, column sep=0pt]{
      \node (D1) {\KPED}; & 
      \KPEcrta & 
      \KPEB & 
      \KPEpike & 
      \node (P1) {\KPEP}; & 
      \KPEpike & 
      \node (B'1) {\KPEB'}; & 
      \KPEcrta & 
      \KPED'\textrm{. } & 
      \node (D2) {\KPED}; &
      \KPEcrta & 
      \KPEB & 
      \KPEpike & 
      \node (P2) {\KPEP}; & 
      \KPEpike & 
      \node (B'2) {\KPEB'}; & 
      \KPEcrta & 
      \KPED'\textrm{. } &
      \node (D3) {\KPED}; & 
      \KPEcrta & 
      \KPEB & 
      \KPEpike & 
      \node (P3) {\KPEP}; & 
      \KPEpike &
      \textrm{ itd.}\\
    };

    % \draw (P1.north) to[out=90,in=90] (P2.north);
    % \draw (P2.north) to[out=90,in=90] (P3.north);
    \draw [overbrace style] (P1.north) -- (P2.north);
    \draw [overbrace style] (P2.north) -- (P3.north);
    \draw [underbrace style] (D1.south) -- (D2.south);
    \draw [underbrace style] (D2.south) -- (D3.south);
    \draw [underbrace style, decoration={raise=0.5em}] (B'1.south) -- (B'2.south);

\end{tikzpicture}
\]

Že pri drugi ponovitvi krožnega toka se pojavi krožni tok \( \KPEP \KPEpike \KPEB' \KPEcrta \KPED'\textrm{. } \KPED \KPEcrta \KPEB \KPEpike \KPEP \), še preden se konča drugi krožni tok \( \KPED \); in na vse nadaljnje krožne toke lahko tako gledamo z vidika oblike \( \KPEP \KPEpike \KPEB' \KPEcrta \KPED' \KPEpike \KPEP \), tako da je \( \KPED \KPEcrta \KPEB \) kot prva faza prvega krožnega toka le prehodna priprava stalno ponavljajočega se krožnega toka produktivnega kapitala, kakor je to v resnici pri industrijskem kapitalu, ki se prvič naloži v obliki denarnega kapitala.

Na drugi strani: prvi krožni tok \( \KPEB' \KPEcrta \KPED'\textrm{. } \KPED \KPEcrta \KPEB \KPEpike \KPEP \KPEpike \KPEB' \) (okrajšano \( \KPEB' \KPEpike \KPEB' \)), krožni tok blagovnega kapitala, je opravljen, preden se sklene drugi krožni tok \( \KPEP \). Tako vsebuje prva oblika že obe drugi, in tako izgine denarna oblika, razen kolikor ni zgolj vrednostni izraz, ampak je vrednostni izraz v obliki ekvivalenta, v denarju.

Končno. Če vzamemo posamezni na novo nastopajoči kapital, ki prvič opravlja krožni tok \( \KPED \KPEcrta \KPEB \KPEpike \KPEP \KPEpike \KPEB' \KPEcrta \KPED' \), potem je \( \KPED \KPEcrta \KPEB \) pripravljalna faza, predhodnica prvega produkcijskega procesa, ki ga opravlja ta posamezni kapital. Ta faza \( \KPED \KPEcrta \KPEB \) torej ni njegova predpostavka, ampak jo nasprotno produkcijski proces določa oziroma pogojuje. Vendar velja to le za posamezni kapital. Krožni tok denarnega kapitala je splošna oblika krožnega toka industrijskega kapitala, če gre za kapitalistični produkcijski način, torej v okviru družbene ureditve, ki jo določa kapitalistična produkcija. Kapitalistični produkcijski proces je torej predpostavljen kot prius; če ne v prvem krožnem toku denarnega kapitala na novo naloženega industrijskega kapitala, pa zunaj njega; trajni obstoj tega produkcijskega procesa zahteva nenehno obnavljanje krožnega toka \( \KPEP \KPEpike \KPEP \). V prvem stadiju \( \KPED \KPEcrta \KPEBrazcepDsPs \) se kaže ta predpostavka že s tem, ker predpostavlja po eni strani obstoj razreda mezdnih delavcev, \KPEstran po drugi pa, ker predpostavlja, da je to, kar je za kupca produkcijskih sredstev prvi stadij \( \KPED \KPEcrta \KPEB \), za njihovega prodajalca \( \KPEB' \KPEcrta \KPED' \), ker predpostavlja torej obstoj blagovnega kapitala v \( \KPEB' \), zato blago s\^amo kot rezultat kapitalistične produkcije in s tem delovanje produktivnega kapitala.

\end{document}
